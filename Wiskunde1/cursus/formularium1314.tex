\documentclass[10pt]{article}
\usepackage{a4wide}
\usepackage{multicol}
\usepackage{palatino}
\usepackage{mathpazo}
\usepackage{listings}
\usepackage{array}

\lstset{language=scilab,basicstyle=\tt}


\newcommand{\en}{\wedge}
\newcommand{\of}{\vee}
\newcommand{\niet}{\neg}
\newcommand{\alsdan}{\rightarrow}
\newcommand{\asa}{\leftrightarrow}

\newcommand{\mysection}[1]{\vspace{1.em} \noindent {\bf {\large #1}}}
\newcommand{\mym}[1]{${\displaystyle #1}$}
\newcommand{\di}{\displaystyle}
\newcommand{\D}{{\rm D}}

\renewcommand{\arraystretch}{1.5} % extra verticale ruimte in tabellen

\pagestyle{empty}


\begin{document}

\noindent
{\bf {\huge Formularium Wiskunde}}

\vspace{1.5em}

\mysection{Nulpunten kwadratische functie} \\

\begin{center}
  \begin{tabular}{rc}
    \bf Voorschrift & $y(x)=ax^2+bx+c$ \\[2mm]
    \bf Nulpunten & $\di x_{1,2}=\frac{-b\pm \sqrt{b^2-4ac}}{2a}$ \\
  \end{tabular}
\end{center}


\mysection{Exponenti\"ele \& logaritmische functies}
\[
  \log_g(x)=y\Leftrightarrow g^y=x
\]

\noindent {\bf Rekenregels}
\[
  \begin{array}{r@{\;=\;}l}
    g^{x+y}              & g^x\cdot g^y \\[2mm]
    \left(g^x\right)^y  & \left(g^y\right)^x=g^{x\cdot y} \\[2mm]
    g^{-x}               & \di \frac1{g^x} \\[2mm]
    \log_g(x \cdot y)   & \log_g(x)+\log_g(y) \\[2mm]
    \log_g(x^y)          & y \cdot \log_g(x) \\[2mm]
    \di \log_g\left(\frac1x\right) & -\log_g(x)
  \end{array}
\]


\mysection{Meetkunde} \\
\begin{center}
  \begin{tabular}{rc}
    \bf Omtrek cirkel & $2 \pi R$ \\[2mm]
    \bf Oppervlakte cirkel & $\pi R^2$ \\
  \end{tabular}
\end{center}


\mysection{Propositielogica}\\

\begin{center}
  \begin{tabular}{cc}
  \sc en & $\en$ \\  
  \sc of & $\of$ \\ 
  \sc niet & $\niet$ \\ 
  \end{tabular} 
\end{center}

\clearpage

\noindent
{\bf {\huge Formularium Scilab}}

\vspace{0.5em}

\begin{multicols}{2}
\begin{center}
  \begin{tabular}{p{4.cm}l}
    \multicolumn{2}{c}{\large\bfseries Rekenen in Scilab} \\
    Vierkantswortel & \verb'sqrt(x)' \\
    Macht & \verb'x^y' \\ Absolute waarde & \verb'abs(x)' \\
    $\ln(x)$ & \verb'log(x)' \\
    \mym{e^x} &\verb'exp(x)' \\
    $\pi$ & \verb'%pi' \\
    $\mathrm{true}, \mathrm{false}$ & \verb'%T, %F'
  \end{tabular}
\end{center}
\begin{center}
  \begin{tabular}{|c|c|}
    \multicolumn{2}{c}{\large\bfseries Logische operatoren} \\
    \hline
    {\sc en} & \verb'&' \\
    \hline
    {\sc of} & \verb'|' \\
    \hline
    $=$ & \verb'==' \\
    \hline
    $\neq$ & \verb'~=' \\
    \hline
    $\leq$ & \verb'<=' \\
    \hline
    $\geq$ & \verb'>=' \\ 
    \hline
  \end{tabular}
\end{center}
\end{multicols}


\mysection{Vectoren}\\
\begin{tabular}{|p{7.3cm}|p{5.5cm}|}
  \hline
  definitie vector    &   \verb+V=[1,1.5,2,2.5,3]=1:0.5:3+ \\
  vectorelement       &   \verb+V(2)+\\
  vector met $4$ elementen, alle gelijk aan $0$ resp. $1$ & \verb+zeros(4,1)+, \verb+ones(4,1)+ \\
  lengte van de vector $V$ & \texttt{length(V)}\\
  \hline
\end{tabular}

\vspace{0.5cm}

\paragraph{Functiedefinitie}
\begin{center}
\begin{minipage}{.6\linewidth}
\begin{lstlisting}
function y=mijnFunctie(x)
  y=2*x
endfunction
\end{lstlisting}
\end{minipage}
\end{center}


\paragraph{Tekenen}
\begin{center}
\begin{minipage}{.6\linewidth}
\begin{lstlisting}
clf; xgrid; x=0:10; plot(x,mijnFunctie)
\end{lstlisting}
\end{minipage}
\end{center}


\paragraph{Lussen}
\begin{center}
\begin{minipage}{.6\linewidth}
\begin{lstlisting}
for i = 1:n ... end
while i < n ... end
\end{lstlisting}
\end{minipage}
\end{center}

\paragraph{Voorwaardelijk statement}
\begin{center}
\begin{minipage}{.6\linewidth}
\begin{lstlisting}
if x == 2 then ...
elseif ... then ...
else ...
end
\end{lstlisting}
\end{minipage}
\end{center}

\vfill
\begin{center}  Academiejaar 2013--2014 \end{center}

%\end{multicols}
\end{document}
