\section{Oefeningen}
\begin{oef}
   Bereken volgende uitdrukkingen in SciLab: 
\begin{enumerate}
\item  $\dfrac{4\cdot \pi-7}4 \cdot 2+10$
\item $\dfrac{ -7^3}8 $
\item $| - 4| $
\item mod(7, 3), mod($ -15$, 4) 
\item floor($-2.45$) 
\item $500\cdot 1.05^\frac{12}2\cdot \dfrac{1-1.05^\frac{-13}2}{1-1.05^\frac{-1}2} $
\end{enumerate}
   \begin{opl}
\begin{enumerate}
\item  12.783185	
\item - 42.875
\item 4
\item 1, -3 
\item -3
\item 7555.9475
\end{enumerate}
   \end{opl}
\end{oef}

\begin{oef}
Definieer de functie $opp(z)$ die de totale oppervlakte berekent 
van een kubus met zijde $z$. Bereken nu de oppervlakte 
als $z$ gelijk is aan \num{2.5} of \num{3.125}.
\begin{opl}
\begin{lstlisting}
function r = opp(z)
  r = z * z
end

opp(2.5)
opp(3.125)
\end{lstlisting}
\end{opl}
\end{oef}



\begin{oef}
Een rol behangpapier is \SI{10}{\meter} lang. Stel de veranderlijke~$t$
gelijk aan de hoeveelheid behangpapier die je nodig hebt, uitgedrukt in meter.
Stel \verb/aantalrollen/ het aantal rollen dat je moet kopen om aan de
hoeveelheid $t$ te voldoen. Teken de grafiek van de functie \verb/aantalrollen/.
\begin{opl}
\begin{lstlisting}
function r = aantalrollen(t)
  r = ceil(t / 10)
end

plot(0:100, aantalrollen)
\end{lstlisting}
\end{opl}
\end{oef}


\begin{oef}
De studentenkring organiseert een voetbalwedstrijd. De materiaalmeester heeft \SI{100}{\meter} lint bij om het voetbalterrein af te spannen. 
\begin{enumerate}
\item Teken de grafiek van de functie \verb/breedte/ die de breedte $b$ van het terrein in functie van de lengte $l$ weergeeft.
\item Teken de grafiek van de functie \verb/opp/ die de oppervlakte $o$ van het terrein in functie van de lengte $l$ weergeeft. Voor welke lengte is de oppervlakte maximaal?
\end{enumerate}
\begin{opl}
\begin{lstlisting}
function b = breedte(l)
  b = 50 - l
end

function r = opp(l)
  r = breedte(l) * l
end

plot(0:50, opp)
\end{lstlisting}
Maximumoppervlakte bij \verb'l' = 25.
\end{opl}
\end{oef}

\begin{oef}
(Moeilijker) Er is een misdaad gebeurd. De politie moet twee oppervlakten afspannen. Ze gebruikt daarvoor \SI{60}{\meter} lint. De eerste oppervlakte is cirkelvormig. De tweede oppervlakte is een vierkante oppervlakte die langs \'e\'en zijde aan een muur grenst. Langs de muur hoeft geen lint gespannen te worden. 
\begin{enumerate}
\item Definieer de functie \verb/oppervlakte/ die de afgespannen oppervlakte $opp$ toont in functie van zijde $z$ van het vierkant.
\item Teken deze functie.
\item Wat zijn de afmetingen van cirkel en vierkant opdat de afgespannen oppervlakte minimaal is?
\end{enumerate}
\begin{opl}
\begin{lstlisting}
function r = straal(z)
  r = (60 - 3 * z) / (2 * %pi)
end

function r = oppervlakte(z)
  r = z * z + %pi * straal(z)^2
end

plot(0:50, oppervlakte)
\end{lstlisting}
Bij minimale oppervlakte: $z = 8.35$ en $r = 5.56$.
\end{opl}
\end{oef}

\begin{oef}
Teken de functies $y = 5^x$ en $y = \frac15^x$
op hetzelfde scherm. 
Zorg voor aangepast bereik van de co\"ordinaatassen. Verzorg ook de titel van de grafiek.
\end{oef} 


\begin{oef}
Herschrijf de functie van listing~\ref{fac} waarbij je te werk gaat zoals in listing~\ref{lege_vector}.
\end{oef}

\begin{oef}
Schrijf de functie  \verb/sommeetk(a,r,n)/, die de som berekent van opeenvolgende elementen van een meetkundige rij.
De veranderlijke $a$ is de eerste term van de som, $v$ is de reden van de meetkundige rij en de som 
bevat $n$ termen. Geef als output  de som en de laatste term van de som.
(meervoudige output). Gebruik eerst de \verb/for/-lus , daarna  de \verb/while/-lus.
\begin{opl}
\begin{samepage}
Met een {\tt for}-lus:
\begin{lstlisting}
function [som, term] = sommeetk(a, r, n)
  som = 0
  term = a
  for k = 1:n
    som = som + term
    term = term * r
  end
  term = term / r
endfunction
\end{lstlisting}
\end{samepage}
\begin{samepage}
Met een {\tt while}-lus:
\begin{lstlisting}
function [som, term] = sommeetk(a, r, n)
  som = 0
  term = a
  k = 1
  while k <= n,
    som = som + term
    term = term * r
    k = k + 1
  end
  term = term / r
endfunction
\end{lstlisting}
\end{samepage}
\end{opl}
\end{oef}


\begin{oef}
Definieer in Scilab volgende matrices en vectoren:
\begin{enumerate}
  \item $\displaystyle A=\begin{bmatrix}1&2&3\\4&5&6\\7&8&9\end{bmatrix}$
  \item $B=\begin{bmatrix}1&2&3&4&\ldots&19&20\end{bmatrix}$
  \item $C=\begin{bmatrix}1\\2\\3\\ \ldots \\19\\20\end{bmatrix}$
  \item $D=\begin{bmatrix}2\\4\\6\\\ldots\\38\\40\end{bmatrix}$
\end{enumerate}
\begin{opl}
\begin{lstlisting}
A = [1,2,3;4,5,6;7,8,9]
B = [1:20]
C = B'
D = C * 2
\end{lstlisting}
\end{opl}
\end{oef}


\begin{oef}
Bepaal het element van $A$ op de tweede rij, derde kolom.
\begin{opl}
\begin{lstlisting}
A(2,3)
\end{lstlisting}
\end{opl}
\end{oef}

\begin{oef}
Definieer de submatrix $E$ die bestaat uit de eerste 5 kolommen van $B$.
\begin{opl}
\begin{lstlisting}
E = B(:,1:5)
\end{lstlisting}
\end{opl}
\end{oef}

\begin{oef}
Voeg de matrices $C$ en $D$ samen tot een matrix met 2 kolommen en 20 rijen.
\begin{opl}
\begin{lstlisting}
[C,D]
\end{lstlisting}
\end{opl}
\end{oef}

\begin{oef}
Schrijf de functie \texttt{vermenigvuldigMatrix(M,a)} die elk element van de matrix $M$ vermenigvuldigt met $a$.
\begin{opl}
\begin{lstlisting}
function R = vermenigvuldigMatrix(M, a)
  R = M * a
end
\end{lstlisting}
\end{opl}
\end{oef}

\begin{oef}
Schrijf de functie \texttt{kolomNaarRij(V)} die de kolomvector $V$ herschrijft tot een rijvector.
\begin{opl}
\begin{lstlisting}
function R = kolomNaarRij(V)
  R = V'
end
\end{lstlisting}
\end{opl}
\end{oef}

\begin{oef}
Schrijf de functie \texttt{keerOm(V)} die de rijvector $V$ ``omkeert'': eerste element wordt het laatste enz.
\begin{opl}E\'en van de vele mogelijke oplossingen:
\begin{lstlisting}[caption={Een vector omkeren}, label=vectoromkeren]
function W=keerom(V)
  // schrijft vector V in omgekeerde volgorde
  W=V
  l=length(V)
  for i=1:l
    W(l+1-i)=V(i)
  end
endfunction
\end{lstlisting}
\end{opl}
\end{oef}

\begin{oef}
Schrijf de functie \texttt{maakMatrix(n)} die een vierkante matrix met $n$ elementen genereert.
Indien $n$ geen kwadraat is van een geheel getal, toont de functie een foutboodschap en geeft ze {\sc false} (\texttt{\%F}) terug.
\end{oef}
   
\begin{oef}
Schrijf de functie \func{aantal(K,r,T)}, die volgend probleem oplost:
\begin{quote}
Er staat nu \euro $K$  op een rekening met  rente $r$  per periode.
Elke periode wordt er \euro ${T}$ van de  rekening afgehaald. 
Na 1 periode haal je voor de eerste keer \euro ${T}$ af.
\begin{enumerate}
    \item Zoek hoeveel keren je het bedrag $T$ kan afhalen zonder dat de rekening in onder nul komt.
    \item Als output geef je het aantal keren dat je een bedrag 
          $T$ afhaalde en het bedrag dat nog op de rekening overblijft na 
          de laatste $T$ af te halen.
\end{enumerate}  
\end{quote}
\begin{opl}
\begin{lstlisting}
function K = intrest(K, r)
    K = K * (1 + r/100)
endfunction

function [n,K] = aantal(K, r, T)
    n = 0
    K2 = intrest(K, r) - T
    
    while K2 >= 0,
        K = K2
        K2 = intrest(K, r) - T
        n = n + 1
    end
endfunction
\end{lstlisting}
\end{opl}
\end{oef}

\begin{oef}
Schrijf de functie \func{posnegvector(n)} die een vector $V$ 
weergeeft met hoogstens $n$ elementen. In het programma wordt hoogstens $n$ keren 
een geheel getal opgevraagd (gebruik het commando \emph{a=input(`  \ldots')}):
indien dit getal positief is geef je  het element van de vector de waarde 1, 
indien het getal negatief is dan wordt het element van de vector gelijk aan -1.
De vector eindigt als de opgevraagde input =0.
\begin{opl}
\begin{lstlisting}
function R = posnegvector(n)
    R = []
    x = 1
    k = 0
    while k < n & x <> 0,
        x = input('Geef getal in: ')
        if x < 0 then
            x = -1
        end
        
        if x <> 0 then
          R = [R, x]
          k = k + 1
        end
    end
endfunction
\end{lstlisting}
\end{opl}
\end{oef}


\begin{oef}
Definieer de samengestelde functie \func{g(x)} met
    als voorschrift:
$g(x)=2*x-3$ als $x<2$ en $g(x)=(x-3)+2$ als $x>=2$. 
Teken die functie op het
interval $[-5,5]$.
\begin{opl}
\begin{lstlisting}
function y = g(x)
    if x < 2 then
        y = 2*x-3
    else
        y = x-3+2
    end
endfunction

plot(-5:5, g)
\end{lstlisting}
\end{opl}
\end{oef}

\begin{oef}
Schrijf de functie \func{zoekdelers(x,v),} met $x$ een 
geheel getal kleiner dan  $10000$, en $V$, een vector met een 
aantal priemgetallen. De output van de functie is een nieuwe 
vector $W$ die alleen die priemgetallen uit $V$ bevat die deler 
zijn van $x$.\\
\textbf{Uitbreiding} De output is nu een matrix met 2 rijen. De 
eerste rij is de vector $W$ zoals hierboven vermeld. De tweede 
rij geeft voor elke priemdeler uit $W$ aan, hoe dikwijls die voorkomt in 
de ontbinding van $x$.
\begin{opl}
\begin{lstlisting}
function [x, m] = multipliciteit(x, k)
    m = 0
    
    while modulo(x, k) == 0,
        x = x / k
        m = m + 1
    end
endfunction

function [ps, ms] = zoekdelers(x, v)
    ps = []
    ms = []    
    
    for p = v
        [x, m] = multipliciteit(x, p)
        
        if m > 0 then
            ps = [ ps, p ]
            ms = [ ms, m ]
        end
    end
endfunction
\end{lstlisting}
\end{opl}
\end{oef}


\begin{oef}
Het algoritme van Euclides dient  om de grootste gemene delers van 2 gehele getallen
te vinden. Zoek het algoritme op op internet. Schrijf nu de functie 
\func{ggd(D,d)} die dit algoritme toepast en als output de grootste gemene deler van de gehele getallen \verb+D+ en \verb+d+ weergeeft.

\begin{opl}
\begin{lstlisting}
function [y, x] = swap(x, y)
endfunction

function r = ggd(D,d)
    while d <> 0
        D = modulo(D, d)
        [D, d] = swap(D, d)
    end
    
    r = D
endfunction
\end{lstlisting}
\end{opl}
\end{oef}

\begin{oef}
Schrijf de functie \func{reverse(V)}, met output de vector $W$ 
waarbij de elementen van $V$ in omgekeerde volgorde voorkomen.\\
\textbf{Uitbreiding  \func{reverse(V,n})}. De elementen van $V$ 
moeten niet alleen in omgekeerde volgorde voorkomen maar elk element 
moet $n$ keren herhaald worden.

\begin{opl}
\begin{lstlisting}
function W = reverse(V, n)
    W = []
    
    for x = V
        for k = 1:n
            W = [x, W]
        end
    end
endfunction
\end{lstlisting}
\end{opl}
\end{oef}

\begin{oef}
Schrijf drie verschillende functies die de matrix 
\begin{equation*}
\begin{bmatrix}
 1&2  &3  \\
 4& 5 &6 \\
 7& 8 &9
\end{bmatrix}
\end{equation*}
genereren. Input is de integer \verb+n+. Als \verb+n+ geen kwadraat is, genereert de functie een foutmelding.
\end{oef}

\begin{oef}
Kerstmis 2003 kochten we een kerstboom van \SI{1}{\meter}. Nadien plantten we hem in de tuin. Hij groeit jaarlijks \SI{30}{\centi\meter}. Met Kerstmis zetten we hem elk jaar weer binnen. Onze living is \SI{2,30}{\meter} hoog. Tot welk jaar kunnen wij onze kerstboom gebruiken zonder er een stukje af te moeten snijden? Schrijf de functie \verb/y=kerstboom()/ die volgende boodschap doet verschijnen op het scherm: ``Met kerstmis in het jaar y zult u een nieuwe kerstboom moeten kopen.'' Vervang \verb/y/ door het berekende jaartal.
\begin{opl}
\begin{lstlisting}
function y = kerstboom()
    y = 2003
    lengte = 1
    
    while lengte <= 2.30,
        y = y + 1
        lengte = lengte + 0.3
    end
    
    printf("Met kerstmis in het jaar %d zult u " +
           "een nieuwe kerstboom moeten kopen.", y)
endfunction
\end{lstlisting}
\end{opl}
\end{oef}


\begin{oef}
\begin{itemize}
  \item Schrijf de functie \verb/EuroNaarEurocent(V)/. Input is de vector \verb/V/ met twee elementen (aantal euro, aantal eurocent). Output is het totaal aantal eurocent.
  \item Schrijf de functie \verb/EurocentNaarEuro(x)/ die het omgekeerde doet van de vorige functie.
  \item Schrijf de functie \verb/GeefTerug(V,W)/. Input is het bedrag dat overhandigd werd, genoteerd door de vector \verb/V/ (euros en eurocenten), en het bedrag dat moet betaald worden, eveneens als vector.
\end{itemize}
\begin{opl}
\begin{lstlisting}
function R = EuroNaarEurocent(V)
    R = V(1) * 100 + V(2)
endfunction

function R = EurocentNaarEuro(x)
    R = [ floor(x / 100), modulo(x, 100) ]
endfunction

function R = GeefTerug(V,W)
    x = EuroNaarEurocent(V)
    y = EuroNaarEurocent(W)
    R = EurocentNaarEuro(x - y)
endfunction
\end{lstlisting}
\end{opl}
\end{oef}

\begin{oef}
Schrijf de functie \verb/MaxSom(V)/. Input is de vector \verb/V/. Output is de positie van de grootste som
van 3 opeenvolgende elementen (begin- en eindindex) \'en de waarde van die som. Test de functie ook met negatieve getallen!
\begin{opl}
\begin{lstlisting}
function [maxFrom,maxTo,maxSum] = MaxSom(V)
    maxFrom = 1
    maxSum = sum(V(1:3))
    s = maxSum
    [h, w] = size(V)

    for i = 2:w-2
        s = s - V(i-1) + V(i+2)
        
        if s > maxSum then
            maxSum = s
            maxFrom = i
        end
    end
 
    maxTo = maxFrom + 2   
endfunction
\end{lstlisting}
\end{opl}
\end{oef}

\begin{oef}
\begin{itemize}
  \item Schrijf de functie \verb/gemiddelde(V)/: output is het gemiddelde van de vector \verb/V/.
  \item Schrijf de functie \verb/aantalElementenGroterDanGemiddelde(V)/: input vector \verb/V/, output aantal elementen van \verb/V/ die groter zijn dan het gemiddelde van de elementen van \verb/V/.
  \item Schrijf de functie \verb/elementenGroterDanGemiddelde(V)/. Output is de vector \verb/W/ die alle elementen bevat die groter zijn dan het gemiddelde van \verb/V/. Breid daarna uit naar een matrix.
\end{itemize}
\begin{opl}
\begin{lstlisting}
function R = gemiddelde(M)
    [h, w] = size(M)
    
    R = sum(M) / (h*w)
endfunction

function n = aantalElementenGroterDanGemiddelde(V)
    g = gemiddelde(V)
    n = 0
    
    for col = V
        for x = col'
            if x > g then
                n = n + 1
            end
        end
    end
endfunction

function R = elementenGroterDanGemiddelde(M)
    g = gemiddelde(M)
    R = []
    
    for col = M
        for x = col'
            if x > g then
                R = [R, x]
            end
        end
    end
endfunction
\end{lstlisting}
\end{opl}
\end{oef}

\begin{oef}
\begin{itemize}
  \item Schrijf de functie \verb/equals(V,W)/ die controleert of de vectoren \verb/V/ en \verb/W/ gelijk zijn aan mekaar.
  \item Schrijf de functie \verb/equalsMatrix(M,N)/ die controleert of de matrices \verb/M/ en \verb/N/ gelijk zijn aan mekaar.
\end{itemize}
\begin{opl}
\begin{lstlisting}
function R = equalsMatrix(M, N)
    [h,w] = size(M)
    [h2,w2] = size(N)
    
    if w <> w2 | h <> h2 then
        R = %f
    else
        R = %t
        
        for i = 1:h
            for j = 1:w
                R = R & M(i,j) == N(i,j)
            end
        end
    end
endfunction

function R = equals(V, W)
    equalsMatrix(V,W)
endfunction
\end{lstlisting}
\end{opl}
\end{oef}


%%% Local Variables: 
%%% mode: latex
%%% TeX-master: "../cursusTW1"
%%% End: 
