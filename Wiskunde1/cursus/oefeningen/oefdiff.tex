%%%%%%%%%%%%%%%%%%%%%%%%%%%%%%%%%%%%%%%%%%%%%%%%%%%%%%%%%%%%%%%%%%%%%%%%%
% Laatste aanpassing:           
%
% 9/9/02 [Jan]: figuren.eps in pdf veranderd, file bewaard als unix-type.
%   Oefening weggelaten (gegeven f zoek f' en omgekeerd), omwille van
%   het ontbreken van beide .eps figuren. Enkele overmatige bold,underline
%   e.d. weggelaten.
%
% 12/06/02 door Roos            
%   
%   formulering van enkele mini-maxi-problemen verbeterd,  
%   nieuwe oefeningen toegevoegd, volgorde veranderd
%                                  
%                                   
%%%%%%%%%%%%%%%%%%%%%%%%%%%%%%%%%%%%%%%%%%%%%%%%%%%%%%%%%%%%%%%%%%%%%%%%%%

%\chapter{Oefeningen afgeleiden}
\section{Oefeningen}
\emph{Herhaling}


\subsection{Definitie en rekenregels}
\begin{enumerate}
    \item  In figuur \ref{fig:combinaties} zijn de grafieken gegeven van
    functies (a tot en met d).
    A, B, C en D zijn de grafieken van de afgeleide functies van
    a, b, c en d. Maak de juiste koppeling door in enkele goedgekozen punten
    \emph{het teken en de grootte} van de afgeleide functie te bepalen.
    \begin{figure}[tbp]
            \centering
        \includegraphics[bb= 20 250 400 780,clip,width=\textwidth]{figuren/functies_en_afgeleiden.pdf}
        \caption{Combineer de juiste afgeleide en functie}
        \label{fig:combinaties}
    \end{figure}

    \item Zelfde vraag voor figuur~\ref{fig:combinaties2}
    \begin{figure}[tbp]
            \centering
        \includegraphics[bb= 30 80 400 600,clip,width=\textwidth]{figuren/uitvoer.pdf}
        \caption{Combineer de juiste afgeleide en functie}
        \label{fig:combinaties2}
    \end{figure}


    \item  Bepaal zoals in de cursus (\ldots een raaklijn is een
    snijlijn die maar 1 snijpunt mag hebben) de
    afgeleide functie van de kwadratische functie, eerst eens in een 
    concreet punt vb.\ $(1,0)$ voor $f(x)$, daarna in een willekeurig 
    punt $(a,f(a))$
    \begin{eqnarray*}
        &&f(x)=-3x^{2}+4x-1 \\
        &&g(x)=5x^{2}+7x-2
    \end{eqnarray*}

    \item  Gegeven de vraagfunctie $3\cdot q+4\cdot p=10$.
    \begin{enumerate}
        \item  Geef het voorschrift van de opbrengstfunctie $O(q)$

        \item  Bereken de gemiddelde
    opbrengsttoe- of afname als $q=2$ en $\Delta q=0.2$.

        \item  Doe hetzelfde als  $q=0.75$ en $\Delta q=0.2$
    \end{enumerate}
\newpage
    \item  Bereken de afgeleide functie van de volgende functies. Geef
    de resultaten zoveel mogelijk met ontbonden vormen.\\
    {\setlength{\extrarowheight}{10pt}
    \begin{tabular}{llll}
        (a) & $\displaystyle{y=\frac{x^{4}-1}{x^{4}}}$
            & (k) & $\displaystyle{y=\frac{\sqrt {x+2}}{x}}$ \\

        (b) & $\displaystyle{y=\sqrt{(x+2)(x-2)}}$
            & (l) & $\displaystyle{y=\frac{4}{\sqrt{x}}-\frac{x^{2}}{2}}$\\

        (c) & $\displaystyle{y=(x^{2}-3x+5)\cdot(x^{2}+x+1)}$
            & (m) & $\displaystyle{y=\frac{x^2+3x-7}{6x-x^2}}$ \\

        (d) & $\displaystyle{y=\frac{2x}{1+x^{2}}}$
            & (n) & $\displaystyle{y=\frac{x^3+1}{x}}$ \\

        (e) & $\displaystyle{y=\sqrt{x^{3}+5x-1}}$
            & (o) & $\displaystyle{y=\frac{x^3+1}{x}}$ \\

        (f) & $\displaystyle{y=\frac{(x+2)^{3}}{(2x-1)^{2}}}$
            & (p) & $\displaystyle{y=\frac{x^2}{x-4}}$ \\

        (g) & $\displaystyle{y=\sqrt{9x^2+2x}}$
            & (q) & $\displaystyle{y=\sqrt{9x^2+2x}}$ \\

        (h) & $\displaystyle{y=-(x^2+1)^2(2x^3-8)}$
            & (r) & $\displaystyle{y=\sqrt{x+1}}$ \\

        (i) & $\displaystyle{y=(4x+1)^3}$
            & (s) & $\displaystyle{y=(2x-3)^2}$ \\

        (j) & $\displaystyle{y=(5x^3-2x)(2x-8)}$
            & (t) & $\displaystyle{y=(2x^2-65)(x+1)}$ \\ 
	\end{tabular}}
\end{enumerate}




\subsection{Raaklijnen}
\begin{enumerate}
    \item  Bereken de vergelijking van de raaklijn aan de grafiek van
    $f(x)$ in het punt met eerste co\"{o}rdinaat $a$ als
    \begin{enumerate}
        \item  $\displaystyle{f(x)=\frac{1}{x}}$ en $a=3$

        \item  $\displaystyle{f(x)=\frac{x^{2}+1}{x-1}}$ en $a=2$

        \item  $\displaystyle{f(x)=\frac{4x}{x^{2}+1}}$ en $a=-1$
    \end{enumerate}
    
     \item \label{vr:vglraaklijn} Bepaal de vergelijking van de 
     raaklijn aan de kromme
    $f(x)$ in het  punt $(x_0,y_0)$:
    \begin{enumerate}
        \item $f(x)=x^2+2x+3;~x_{0}=1$
        \item $f(x)=\displaystyle{\frac{1}{x+2}};~x_{0}=3$
        \item $y=\displaystyle{\frac8{x^2+x+2}};~x_{0}=2$
    \end{enumerate}

    \item  Gegeven $\displaystyle{g(x)=\frac{-2x^{2}+10x+4}{x}}$
    \begin{enumerate}
        \item  Zoek  de punten waarin de helling van de functie $g(x)$
        gelijk is aan $-6$.

        \item  Geef de vergelijking van de raaklijn in \'{e}\'{e}n
        van de gevonden punten.

        \item  Zoek de vergelijking van de rechte die lOPOrecht staat
        op die raaklijn in het gevonden raakpunt. (We noemen deze
        rechte de \emph{normaal})
    \end{enumerate}
    
    \item Zelfde vraag als \ref{vr:vglraaklijn}, maar zoek ook de lOPOrechte
    op de raaklijn:
    \begin{enumerate}
        \item $\displaystyle{f(x)=\sqrt{3x^2+x-1}};~x_{0}=-2$
        \item $\displaystyle{f(x)=\frac1{\sqrt{2x+1}}};~x_{0}=4$
    \end{enumerate}

    \item  Bepaal het snijpunt van de raaklijnen in ($\frac{1}{2}$,2)
    en (2,$\frac{1}{2}$) aan de grafiek van $y=\frac{1}{x}$.
    
    \item In welke punten is de raaklijn aan
    \begin{enumerate}
        \item  de functie $f(x)=x^{2}$ evenwijdig met de rechte
        $y=3x-1$?

        \item  de functie $\displaystyle{g(x)=\frac{x+1}{x-2}}$ evenwijdig met de
        tweede bissectrice?

    \end{enumerate}
    
    \item Gegeven $f(x)=x^2+4$. Zoek het punt van de
    grafiek waar de raaklijn door het punt $(0,0)$ gaat.

    \item We zeggen dat twee krommen raken aan elkaar in
    $p(x_0,y_0)$, als ze in dit punt dezelfde raaklijn hebben.
    Onderzoek of de volgende paren krommen elkaar raken.
    \begin{enumerate}
    \item $f(x)=x^3$ en $g(x)=x^2+3x-2$
    \item $f(x)=x^3$ en $g(x)=x^2+x-1$
    \end{enumerate}

    \item  Bepaal $a$ in $f(x)=x^{2}+2a\cdot x+a^{2}$ zo dat de
    raaklijn aan de grafiek in het punt met eerste co\"{o}rdinaat $2$
    lOPOrecht staat op de rechte $y=-x+3$.

    \item  Bepaal de vergelijking van de parabool die voldoet aan
    beide volgende voorwaarden:
    \begin{enumerate}
        \item  De rechte $y=-2x+1$ is een raaklijn aan de functie in
        het punt (0,1).

        \item  De rico van de raaklijn in het punt $x=2$ is gelijk aan
        $3$.

    \end{enumerate}

     \item Bepaal $b$ en $c$ zodat de parabool $y=x^2+bx+c$ raakt
    aan de rechte $y=-5x-3$ in $(-1,2)$.
    
    \item  Gegeven de functie $f(x)=\frac{a\cdot x}{x+b}$. Zoek $a$ 
    en $b$ zodat de raaklijn in het raakpunt $(2,4)$ lOPOrecht staat op 
    de recht $y=\frac{1}{2}\cdot x +3$ in het raakpunt.
    
    \item Zoek het voorschrift van een 3-de graadsfunctie $g(x)$ die 
    voldoet aan
    \begin{itemize}
        \item  de functie gaat door de oorsprong.
    
        \item  de raaklijnen in $x=1$ en $x=\frac{-1}{9}$  zijn horizontaal.
    
        \item  De rechte $y=-x$ staat lOPOrecht op de raaklijn in het 
        punt met $x=0$
    \end{itemize}    

    \item Toon aan dat de driehoek gevormd door de X-as, de Y-as en
    een raaklijn aan de grafiek van $f(x)=\frac{1}{x}$ steeds
    dezelfde oppervlakte heeft.

    \end{enumerate}



    \subsection{Extreme waarden}

    \begin{enumerate}
        \item  Het verband tussen prijs en productie is bepaald door
        het volgend voorschrift: $P(q)=9-0.001\cdot q^{2}$. Ga het
        stijgen en dalen na van deze functie op het domein.

        \item  Onderzoek het stijgen en dalen van volgende functies.

        \noindent
    {\setlength{\extrarowheight}{10pt}
    \begin{tabular}{llll}
        (a) & $f(x)=x^{2}\cdot(x-3)$ &
            (f) & $\displaystyle{k(x)=\frac{x-2}{x^2}}$ \\

        (b) & $g(x)=x^{3}+x^{2}-x+1$ &
            (g) & $\displaystyle{l(x)=\frac{x^2-10x+21}{2x-15}}$ \\

        (c) & $\displaystyle{h(x)=\frac{x+1}{x-1}}$ &
            (h) & $\displaystyle{p(x)=\frac{x^2-5}{2x-4}}$ \\

        (d) & $i(x)=x-\sqrt{4x-16}$ &
            (i) & $q(x)=x^6+x^4$ \\

        (e) & $\displaystyle{j(x)=\frac{x^{2}+1}{2\cdot x}}$ &
            (j) & $r(x)=x^4-18x^2+4$ \\
    \end{tabular}}

        \item  Het elektriciteitsverbruik in een gezin voldoet aan het
        volgende voor\-schrift
        \begin{displaymath}
            E(t)=4-2\cdot (\frac{t-13}{7})^{4}   \mbox{met } 6\leq t \leq20
        \end{displaymath}
    $t$ is de tijd tussen 6 uur 's morgens en 20 uur 's avonds. Door
    gebruik van zon\-ne-\-e\-nergie kan het elektriciteitsverbruik verminderen
    met een bedrag
    \begin{displaymath}
        Z(t)=1-(\frac{t-13}{7})^{2}
    \end{displaymath}
    \begin{enumerate}
        \item  Bereken het maximale verbruik aan energie zonder gebruik
        van zonne-energie.

        \item   Bereken het maximale verbruik aan energie met gebruik
        van zonne-energie. Geef de grafieken weer van die functies met
        Euler.
	
    \end{enumerate}

    \item Toon aan dat de richtingsco\"{e}ffici\"{e}nt van de raaklijn
    aan de grafiek van de functie $f$ met vergelijking 
    $f(x)=-3x^{4}-2x^{2}+4\cdot x$
    afneemt als het raakpunt $x$ naar rechts schuift.

\end{enumerate}

\subsection{Maximum-minimum problemen}

\begin{enumerate}
    \item  Een rechthoek heeft als oppervlakte 16 $\mathrm{cm^{2}}$. Bepaal de
    afmetingen zodat de diagonaal zo kort mogelijk is.

    \item  Een balkvormige doos zonder deksel heeft als inhoud  4
    $\mathrm{dm^{3}}$. Het grondvlak is vierkantig.
    Bepaal de afmetingen zo dat \emph{de oppervlakte} van de doos
    minimaal wordt.
    
     \item Aan de vier hoeken van een rechthoekig stuk karton, van 30~cm
    op 50~cm, snijdt men gelijke vierkanten weg. Van de rest maakt men
    een doos zonder deksel. Voor welke afmetingen van de doos
    is de inhoud van deze doos maximaal?

    \item Het bedrukte gedeelte van een blad is 200~cm$^2$. Bepaal het
    voordeligste formaat als er links en rechts 1~cm, onder en boven
    2~cm moet blijven.

    \item De manager van een winkelketen wil de \emph{rechthoekige}
    parking  van zijn winkel
    (oppervlakte $50~m^2$) omheinen om er reclamepanelen op te
    bevestigen. Drie zijden van de omheining wil hij maken van hout
    (kostprijs \euros{50}/m); de vierde zijde maakt hij uit cementen
    blokken (kostprijs \euros{100}/m). Zoek de afmetingen van de
    omheining die het minst kost.
    
    \item De lengte van een balk is driemaal de breedte van de balk.
    De totale oppervlakte is 200 $\mathrm{cm^{2}}$. Zoek de afmetingen
    zodat de inhoud maximaal is. 
    \item In een rechthoekige driehoek moet de som van de schuine
    zijde en de hoogte van de opstaande rechthoekszijde steeds gelijk zijn aan
    8~cm. Hoe verandert de oppervlakte van die driehoek en wanneer is
    ze maximaal?

    \item  Een handelaar verkoopt wijn aan \euros{3} de fles. Om grote
    bestellingen aan te moedigen, beslist de handelaar een reductie toe te
    kennen voor bestellingen  van meer dan 100 flessen. Voor iedere fles
    boven de 100 wordt de prijs per fles voor de hele bestelling met
    \euros{0.01} verlaagd. Bepaal bij welke hoeveelheid de inkomsten van de
    handelaar maximaal zijn.
    
    \item Tot enige tijd geleden kostte een hamburger in een
    hamburgertent op een festivalweide \euros{2} per stuk. De uitbater
    verkocht toen gemiddeld 10\,000 hamburgers per festivaldag.
    Nadat men de prijs verhoogde tot \euros{2.40}, zakte het
    aantal verkochte hamburgers tot 8\,000 per dag.
    \begin{enumerate}
        \item Welke prijs van de hamburgers maximaliseert de
        opbrengst
        van de uitbater, in de veronderstelling dat een verhoging van de prijs
        een lineaire vermindering van de vraag impliceert, zoals
        aangegeven in de opgave.
        \item Veronderstel dat de uitbater gemiddeld \euros{1\,000}
        vaste kosten heeft per festivaldag, en \euros{0.60} variabele kosten per
        hamburger. Welke prijs \emph{maximaliseert} nu de winst?
    \end{enumerate}
    
    \item Sabena vervoert iedere maand 8\,000 passagiers naar Londen.
    Een ticket kost \euros{100}. Omwille van verliezen in het bedrijf wil
    men de prijs van het ticket verhogen. Men verwacht echter dat, per
    verhoging van \euros{5}, het aantal passagiers daalt met 100
    personen. Bereken de prijs die Sabena de grootst mogelijke 
    \emph{opbrengst} oplevert.
    
   \item  Om 9 uur bevindt schip B zich op 104 km ten oosten van
    schip A. Schip B vaart westwaarts met een snelheid van 16 km per
    uur en schip A vaart zuidwaarts aan 24 km per uur.
    \begin{enumerate}
        \item  Indien beide schepen zo verder varen, wanneer zullen
        zij dan het dichtst bij elkaar zijn?

        \item  Hoe ver zijn ze dan van elkaar verwijderd?
    \end{enumerate}

    \item  Een muur is 3~m hoog en ligt op een afstand van 9~m
    rond een gebouw. Bepaal de lengte van de kortste ladder, die het gebouw
    kan raken, als de ladder buiten de muur op de grond staat. Tip1:
    zoek eigenschap van gelijkvormige driehoeken. Tip 2: $f(x)$
    en $f(x)^{2}$ hebben hun maximum in hetzelfde punt.
    
    \item Welke gelijkbenige driehoek, met omtrek 25~cm, heeft de
    grootste oppervlakte? (HINT: Onderzoek waar het kwadraat van
    de oppervlakte maximaal is.)
    
   \item  Zoek de vergelijking van de rechte door het punt (3,4)
   zodat de oppervlakte van de driehoek afgesneden in het eerste kwadrant 
    minimaal is. Neem als veranderlijken a en b respectievelijk 
    de x-coordinaat en de y-coordinaat van de snijpunten met de assen.
    
    \item De dwarsdoorsnede van een goot heeft de vorm van een 
     gelijkbenige trapezium. De goot is gemaakt uit een stuk zink dat 
     60 cm breed is. De opening van de goot bovenaan (de grote basis 
     van het trapezium) is 30 cm. Voor 
     welke afmetingen van de goot ($x$ breedte van de bodem van de goot,
     $y$ opstaande wand)
     is de oppervlakte van de 
     dwarsdoorsnede maximaal? Gebruik Euler om de afgeleide en de 
     nulpunten te zoeken.
    
    \item  Een rechthoek, met omtrek 18 cm wentelt rond een
    rechthoekszijde. Er ontstaat een cilinder. Voor welke afmetingen
    van de rechthoek is de inhoud van de cilinder maximaal?

     \item Er moet een pijpleiding 
      gelegd worden vanaf positie A in zee naar een punt B op 
      de kustlijn. We veronderstellen de kustlijn 
      rechtlijnig. De lOPOrechte afstand van A naar de kustlijn (C) 
      is 8 km en de afstand langs de kust van C naar B is 20 km. De 
      kostprijs voor de pijpleiding onder water is 17 (geld)eenheden per 
      $km$ en over land slechts 15 eenheden per $km$. Waar moet 
      de pijpleiding aan land komen om de kostprijs zo minimaal 
      mogelijk te houden? Bereken ook de minimale kost in 
      geldeenheden. Noem het punt waar de pijpleiding aan land 
      komt D (D ligt tussen B en C) en stel de afstand tussen C en D gelijk aan 
      de veranderlijke $x$. Gebruik Euler voor het rekenwerk.
   

\end{enumerate}