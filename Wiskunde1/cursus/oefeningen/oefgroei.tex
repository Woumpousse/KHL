%%%%%%%%%%%%%%%%%%%%%%%%%%%%%%%%%
% 6 sept 2013 [Jan]: kleine verbeteringen (komma's, eenheden, getallen)
% november 2012 [Greetje]: oplossingen, fout bij oef 1 en kleine wijziging in oef 2
% mei 2011 [Jan]: oefening Fukushima Jodium 131
%
% 9/4/11 [Greetje] Grondige herziening
% 
%Aanpassing greetje   %
% 10/09/01  
% verbeteringen roos    %
% 
% Laatste aanpassing:           %
% 17/08/02 door Roos
%   verbeteringen schooljaar 2001-2002
%  nieuwe oefeningen test en ex.Roby %
% 12/07/02 door Roos        %
%%%%%%%%%%%%%%%%%%%%%%%%%%%%%%%%%

% \chapter{Oefeningen op exponenti\"{e}le groei}
\section{Oefeningen}


\begin{oef}    
In tabel~\ref{tbl:6processen} vind je de functiewaarden van enkele functies. 
     Ga na of de functies overeenkomen met een
      exponentieel of een lineair proces of geen van
      beide.
      \begin{table}[htb]
                \centering
                \caption{6 verschillende groeiprocessen}
          \begin{tabular}{ccccccc}
         \toprule
         $t$ & $f(t)$ & $g(t)$ & $h(t)$ & $k(t)$ & $m(t)$ & $w(t)$ \\
         \midrule

         1 & 10 & 105 & 12 & 1701 & 5 & \num{29.7}  \\

         2 & 17 & 118 & \num{13.2} & 567 & 25 & \num{27.1}  \\

         3 & 24 & 131 & \num{14.52} & 189 & 125 & \num{24.5}  \\

         4 & 31 & 146 & \num{15.972} & 63 & 625 & \num{21.9}  \\

         5 & 38 & 163 & \num{17.5692} & 21 & 3125 & \num{19.3}  \\
         \bottomrule

     \end{tabular}
          \label{tbl:6processen}
      \end{table}
      \begin{opl}
      \begin{itemize}
      \item $f$: lineair groeiproces met toename gelijk aan 7
      \item $g$: geen exponentiële en geen lineaire groei
      \item $h$: exponentieel groeiproces met groeifactor gelijk aan 1,1
      \item $k$: exponentieel groeiproces met groeifactor gelijk aan $\frac13$
      \item $m$: exponentieel groeiproces met groeifactor gelijk aan 5
      \item $w$: lineair groeiproces met toename gelijk aan $-2,6$
      \end{itemize}
      \end{opl}
\end{oef}

\begin{oef}
 
    
        Het aantal microben in een proefopstelling verdubbelt in 6
     uren tijd.
     \begin{enumerate}
         \item  Wat is de groeifactor (i) per 6 uur? (ii) per dag? (iii) per uur?

         \item  Veronderstel dat het aantal microben nu 100 bedraagt. Bereken zonder rekenmachine en zonder functievoorschrift, indien mogelijk, wanneer het er (i) 400 zijn, (ii) 25,  (iii) 1000.

         \item  Geef het functievoorschrift voor dit groeiproces met als periode
          (i) \'e\'en dag, (ii)  6 uur, (iii) \'{e}\'{e}n uur. Neem 100 als beginwaarde.
          
          \item Maak in Scilab de grafiek van \'e\'en van de functievoorschriften die je hierboven vond. Lees op de grafiek af wanneer er 1000 microben zijn.
     \end{enumerate}
     \begin{opl}
     \begin{enumerate}
     \item $g_6=2$; $g_d=2^4=16$; $g_u=2^\frac{1}{6}=1,122462$
     \item (i) 12u; (ii) 12 uur geleden; (iii) tussen 18 en 24u
     \item $g_6(t)=100\cdot 2^t$; $g_d(t)=100\cdot 16^t$; $g_u(t)=100\cdot \left(2^\frac16 \right)^t$
     \end{enumerate}
     \end{opl}

      \end{oef}

\begin{oef}
  In 2011 waren er ongeveer 7 miljard mensen. Per jaar is
      de bevolkingstoename ongeveer \SI{1,1}{\percent}.
      \begin{enumerate}
          \item  Wat is de groeifactor per jaar? Per decennium? Per semester?
          
          \item Stel de vergelijking op van de groeifunctie die het aantal mensen $M(t)$ op tijdstip $t$ met $t$ uitgedrukt in jaren, weergeeft. 
          
          \item  Hoeveel mensen verwacht men in 2050 volgens dit model?

          \item Maak een grafiek van de functie in Scilab. Lees op de grafiek af wanneer  de bevolking 8 miljard zal bedragen.

      \end{enumerate}
      \begin{opl}
      \begin{enumerate}
      \item $g_j=1,011$; $g_d=1,011^{10}$; $g_s=1,011^\frac{1}{2}$
      \item $M(t)=7\cdot 1,011^t$ in miljard aantal en $t$ het aantal jaren verstreken sinds 2011
      \item $M(39)=7\cdot 1,011^{39}=10,724896$
      \end{enumerate}
      \end{opl}

      \end{oef}



\begin{oef}
 Bij een dieptetoename van \SI{32}{\meter} in de aarde vermeerdert
    de temperatuur met \SI{1}{\celsius}.
    \begin{enumerate}
    \item Veronderstel dat op een diepte van \SI{25}{\meter} een temperatuur
    van \SI{10}{\celsius} heerst.
        \begin{enumerate}
        \item Op welke diepte heerst een temperatuur van
        \SI{15}{\celsius}?
        \item Welke temperatuur heerst er op een diepte van \SI{985}{\meter}?
        \end{enumerate}
    \item Veronderstel dat op de grond (\SI{0}{\meter}) een temperatuur
    van \SI{-5}{\celsius} heerst.
        \begin{enumerate}
        \item Op welke diepte heerst een temperatuur van
        \SI{0}{\celsius}?
        \item Welke temperatuur heerst er op een diepte van \SI{800}{\meter}?
        \end{enumerate}
    \end{enumerate}
\begin{opl}
\begin{enumerate}
\item 
$T_1(t)=10+\frac{t}{32}$ met $t$ het aantal meter grond dieper dan \SI{25}{\meter} onder de grond
\begin{enumerate}
\item Zoek $t$ zodat $T_1(t)=15$: op diepte van \SI{185}{\meter} onder de grond bedraagt de temperatuur \SI{15}{\celsius}.
\item $T_1(960)=40$, dus \SI{40}{\celsius}
\end{enumerate}
\item $T_2(t)=-5+\frac{t}{32}$ met $t$ het aantal meter onder de grond
\begin{enumerate}
\item Zoek $t$ zodat $T_2(t)=0$ geeft $t=160$, dus \SI{160}{meter} onder de grond bedraagt de temperatuur \SI{0}{\celsius}
\item $T_2(800)=20$, dus \SI{800}{meter} onder de grond is het \SI{20}{\celsius}
\end{enumerate}
\end{enumerate}
\end{opl}
\end{oef}





\begin{oef}
 De hoeveelheid  radium halveert om de  1656 jaar.
    \begin{enumerate}
        \item Hoeveel vermindert de hoeveelheid radium
            procentueel per jaar?
        \item Schrijf de functie  die de hoeveelheid radium op tijdstip
            $t$ weergeeft, als de initi\"ele hoeveelheid radium $y_0$
            bedraagt, waarbij $t$ uitgedrukt is in jaren.
        \item Hoeveel gram radium blijft er over van \SI{1}{\gram} na 20
            jaar?
    \end{enumerate}
\begin{opl}
\begin{enumerate}
\item groeifactor per 1656 jaar: $\frac12$; groeifactor per jaar: $\left(\frac{1}{2}\right)^\frac{1}{1656}=0,9995815$ zodat de procentuele afname gelijk is aan \SI{0,0418480}{\percent}.
\item $R(t)=y_0\cdot \left(\frac{1}{2}^\frac{1}{1656}\right)^t$
\item $R(20)=\left(\frac{1}{2}^\frac{1}{1656}\right)^{20}=0,9916636$, dus er blijft \SI{0.9916636}{\gram} over.
\end{enumerate}
\end{opl}
       \end{oef}


\begin{oef}
Een auto verliest jaarlijks \SI{18}{\percent} van zijn waarde.
\begin{enumerate}
  \item Hoeveel \% van de oorspronkelijke waarde blijft er over na 2, 3, 6, 10 jaar?
  \item Wat zal de waarde zijn van een auto binnen 6 jaar, als de auto nu \euros{12\,500} bedraagt? 
\end{enumerate}
\begin{opl}
Het dalend groeiproces wordt beschreven door de exponenti\"ele functie
\[
  P(t) = (1 - \frac{18}{100})^t
\]
\begin{enumerate}
  \item Waardes
        \[
          \begin{array}{cc}
            t & P(t) \\
            \toprule
            2 & 0,67 \\
            2 & 0,55 \\
            6 & 0,30 \\
            10 & 0,14 \\
          \end{array}
        \]
  \item $12.500 \cdot P(6) = 3800$
\end{enumerate}
\end{opl}
\end{oef}

\begin{oef}
Stel dat je eerste over-over-over-grootvader met Belgische nationaliteit in 1830 precies \'e\'en eurocent bij een
bank zou hebben uitgezet tegen een samengestelde intrest van \SI{5}{\percent}, hoeveel zou je dan nu ontvangen?
\begin{opl}
Groeifunctie: $B(t)=1,05^t$: bedrag in eurocent, $t$ jaar na 1830\\
Nu is het 2012, dus bereken $B(182)=7185,42$\\
In 2012 zou je \euros{71,85} ontvangen.
\end{opl}
   
        
\end{oef}






%%% Local Variables: 
%%% mode: latex
%%% TeX-master: "../cursusTW1"
%%% End: 
