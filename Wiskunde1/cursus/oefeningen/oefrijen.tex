%%%%%%%%%%%%%%%%%%%%%%%%%%%%%%%%%%%%%%%%%%%%%%%%%%%%%%%%%%%%%%%%%%%%
% Laatste aanpassing:
% augustus 11 [Jan]: eenheden, getallen consistent gemaakt
%
%5/5/11 [Greetje]: grote kuis
%
% 9/9/02 [Jan]: inconsistentie met euro's weggewerkt.
%           
% 17/08/02 Roos
%   volgorde van de oefeningen volgens moeilijkheidsgraad of aard.
%    foute opgave "groei" verwijderd.      
%   
%   examenvragen 2002 op financi\"{e}le algebra toegevoegd.
%
% 01/09/01 door Greetje          
%%%%%%%%%%%%%%%%%%%%%%%%%%%%%%%%%


\section{Oefeningen}
\subsection{Rekenkundige en meetkundige rijen}
\begin{enumerate}
    \item  Een rekenkundige rij heeft als eerste term $4$, als tweede term $6$.
    \begin{enumerate}
        \item Bereken het verschil $v$. 
        \item Bereken de twintigste term.

        \item  Bereken de som van de eerste 40 termen.

        \item  Een som van opeenvolgende termen van die rij begint
        bij de zesde term. Hoeveel termen moet
        men optellen om als som 264 te bekomen?
    \end{enumerate}

    \item  Een rolletje toiletpapier heeft een diameter van \SI{16}{\centi\meter}. De
    diameter van het binnenste kartonnen rolletje is \SI{4}{\centi\meter}. We nemen
    aan dat het papier een dikte heeft van \SI{1}{\milli\meter}. Noem $t_{1}$ de
    lengte van de eerste winding, $t_{2}$ de lengte van de
    tweede winding enz.
    \begin{enumerate}
        \item  Hoeveel windingen bevat het rolletje
        toiletpapier?

        \item  Hoeveel meter papier is er op de rol? (omtrek
        cirkel: $2 \cdot \pi \cdot R$)
    \end{enumerate}

    \item  Gegeven de meetkundige rij $t_{n}$ met $t_{4}=64$ en $t_{7}=-1$.
    \begin{enumerate}
        \item  Bereken de reden $r$.
        \item Bereken de elfde term $t_{11}$.

        \item  Bereken de som van de eerste 10 termen van deze rij.
    \end{enumerate}

    \item  Een veerkrachtig balletje laat men vallen van op een hoogte
    van \SI{100}{\centi\meter}. De hoogte na elke bots is \'{e}\'{e}n tiende
    minder dan de vorige hoogte.
    \begin{enumerate}
        \item  Welke soort rij vormen de opeenvolgende
        hoogten? 
        \item Geef de formule om een willekeurig element van de
    rij te berekenen uit het eerste element $h_{0}=100$?

        \item  Welke hoogte bereikt het balletje na zes keren
        botsen?

        \item  Welke afstand heeft het balletje afgelegd als
        het de tiende keer net de grond raakt? 
  \end{enumerate}

    \item  Neem een vierkant $V_{1}$ met zijde \SI{1}{\meter}. De middens van de
    zijden bepalen opnieuw een vierkant $V_{2}$. De middens van dit
    vierkant bepalen een derde vierkant $V_{3}$ enz.
    \begin{enumerate}
        \item  Bereken vier elementen van
    de rij $O_{n}$ van de oppervlakte van de
        vierkanten $V_{n}$. Welk soort rij vormen die?

        \item  Geef de formule
        voor een algemeen element van de rij $O_{n}$. 
        \item Bereken de
        waarde van $O_{10}$.
    \end{enumerate}

    \item  Bij een orgel worden pijpen met dezelfde klankkleur in een
    register samengebracht. Elk register bevat 56 pijpen met een
    verschillende toonhoogte. De toonhoogte van een pijp wordt bepaald
    door haar lengte. Hoe langer de pijp, hoe lager de toon. De
    lengtes van aangrenzende pijpen $L_{i}$ staan in de verhouding
    $\frac{\sqrt[12]{2}}{1}$.
    \begin{enumerate}
        \item  Welk soort rij is $L_{i}$, als $L_{1}$ de lengte
        van de langste pijp is? Bepaal de parameters van de rij.

        \item  Als de langste pijp \SI{263}{\centi\meter} is\footnote{Men noemt dit
        een \emph{8-voet register}}, bereken dan $L_{2}$,
        $L_{5}$ en $L_{50}$.

    \item Veronderstel dat voor een ander register de kortste pijp \SI{1,4}{\centi\meter} is,
    bereken dan $L_{55}$, $L_{50}$ en $L_{10}$.
    Rond af op \'{e}\'{e}n  \si{\centi\meter}.
    \end{enumerate}

    \item Een bal rolt van een helling die \SI{100}{\meter} lang is. Tijdens
    de eerste seconde legt hij \SI{1}{\meter} af, tijdens de tweede seconde
    \SI{3}{\meter}, tijdens de derde seconde \SI{5}{\meter} enz. We noemen  $A_n$ de afstand
    afgelegd tijdens de $n$-de seconde. 
    \begin{enumerate}
    \item Bepaal enkele elementen van de rij $A_n$. Welk soort rij bekom je?
    \item Hoe ver rolt de bal in 6 seconden?
    \item In hoeveel tijd legt de bal de hellende weg af?
    \end{enumerate}

    \item Toen, volgens de legende, Sissa Dahir het schaakspel had
    uitgevonden, was koning Hindoe Shiras zo enthousiast, dat hij
    Sissa zelf de beloning liet bepalen. De knecht dacht even na
    en vroeg het volgende: ``Sire, geef me \'e\'en graankorrel
    voor het eerste vakje van het schaakspel, 2 voor het tweede, 4
    voor het derde, 8 voor het vierde enz.\ tot het laatste.''
    Hoeveel ton graan vroeg de knecht? (20 graankorrels wegen
    ongeveer \SI{1}{\gram}.)

   
\end{enumerate}

%%\newpage
\subsection{Samengestelde intrest}
\begin{enumerate}
    \item  Wim is nu 7 jaar oud en wil op zijn twaalfde
    verjaardag een
    fiets met 10 versnellingen als geschenk. Zijn ouders willen nu
    reeds een bedrag op een spaarrekening zetten van \euros{400} tegen een
    jaarlijkse rente van \SI{7}{\percent}.
    \begin{enumerate}
        \item   Over welk bedrag beschikt Wim op zijn
    twaalfde verjaardag?

        \item  Wanneer zal het bedrag op Wim's spaarrekening verdubbeld
        zijn?

    \end{enumerate}
  

    \item  Een vader wil nu een bedrag van \euros{10\,000} verdelen over zijn
    2 kinderen van respectievelijk 10 en \num{14,5} jaar oud. De vader wil
    dat de verdeling zo verloopt, dat elk kind later op zijn
    \'{e}\'{e}nentwintigste verjaardag \emph{hetzelfde} kapitaal ontvangt.
    De jaarlijkse rente op beide rekeningen is dezelfde en bedraagt \SI{7}{\percent}.
    \begin{enumerate}
        \item   Welk bedrag wordt nu voor elk kind belegd?
        \item  Over welk bedrag zal ieder kind beschikken op zijn
        \'{e}\'{e}nentwintigste verjaardag?
    \end{enumerate}


    \item  Jef heeft de gewoonte om regelmatig uitstel van betaling te
    vragen in de winkel van zijn beste vriend Karel. Op 1 november
    2000 heeft Jef volgende schulden bij Karel:
    \euros{1000} te betalen op 1 november 2004, en nog eens een bedrag
    van \euros{3000} te betalen op 1 november 2009. 
    De jaarlijkse rente is gelijk aan  \SI{4}{\percent}.
    \begin{enumerate}
        \item  Jef heeft geld gewonnen met de Lotto en wil dat geld
        gebruiken om alle schulden samen af te betalen op 1 november
        2002. Hoeveel moet Jef betalen aan Karel op 1
        november 2002?

        \item  Jef heeft niet gewonnen met de Lotto en heeft op 1 november 2004  onvoldoende geld
       gespaard om zijn schuld af te lossen. Hij vraagt uitstel
        en wil alle schulden samen terugbetalen op 1 november 2010.
        Hoeveel moet Jef op 1 november 2010 betalen aan Karel?

    \end{enumerate}

    \item Na hoeveel tijd is een kapitaal, dat tegen samengestelde intrest
    is uitgezet:
    \begin{enumerate}
    \item aan \SI{6}{\percent} verdubbeld?
    \item aan \SI{4,5}{\percent} verdrievoudigd?
    \end{enumerate}
    
    \item  Luk en Jos beleggen beiden hetzelfde bedrag op
    verschillende rekeningen. Luk stort op een rekening met een
    jaarlijkse rente van \SI{6}{\percent}, terwijl Jos een semestri\"{e}le
    rentevoet van \SI{4}{\percent} krijgt.
    Wanneer zal het kapitaal van Jos het drievoud zijn van dat van Luk?

    \item David krijgt bij zijn verjaardag \euros{25} als
    geschenk. Hij zet het geld op een spaarboekje aan een intrest
    van \SI{3}{\percent} per jaar. Zijn vriend Maarten krijgt op dezelfde dag
    \euros{20}, maar een bevriend bankier raadt hem aan het te
    beleggen in een kasbon van onbepaalde duur aan \SI{3,5}{\percent} intrest
    per jaar. Wanneer zullen beiden evenveel geld bezitten?

    \item De prijs van een huis vermeerdert maandelijks met \SI{0,5}{\percent}.
    Hoeveel zou een nieuw huis 10 jaar geleden gekost hebben, als
    het nu \euros{200\,000} kost? Hoeveel zal het huis binnen 10
    jaar kosten?

    \item Een multi-miljonair heeft roerend goed belegd in
    Argentini\"e (\euros{1\,000\,000}) en in Brazili\"e
    (\euros{600\,000}). Omwille van inflatie (\SI{11}{\percent} in
    Argentini\"e en \SI{18}{\percent} in Brazili\"e) daalt de waarde echter snel.
    Wanneer zal de waarde van het roerend goed in beide landen 
    evenveel waard zijn?

   \item Een kapitaal van \euros{3000} wordt uitgezet tegen samengestelde intrest
   en groeit aan tot \euros{\num{4432.37}} na 8 jaar. Na hoeveel tijd is dit bedrag
   \euros{5000} geworden ?

    \item Als men  een lening terugbetaalt na 7 jaar  (\'{e}\'{e}nmalige aflossing),
    betaalt men
    \euros{1400} meer terug dan indien men deze lening na 5 jaar zou 
     terug betalen.
    De werkelijke rentevoet  is \SI{4}{\percent}. Hoe groot is het
    ontleende bedrag? ($700\,000$)

    \item Een postbediende heeft een bedrag geleend en verbindt zich ertoe
    \euros{2500} te betalen na 2 jaar en \euros{5000} na 4 jaar. Hoeveel
    bedraagt de lening met een jaarlijkse rente van \SI{4}{\percent}?

    
\end{enumerate}



%\newpage
\subsection{Financi\"{e}le algebra}

\begin{enumerate}
    \item   Een vader stort elke semester \euros{500} op een rekening
    voor zijn zoon. De eerste storting gebeurt op zijn twaalfde verjaardag,
    de laatste op zijn achttiende verjaardag. De jaarlijkse rente is \SI{5}{\percent}.
    \begin{enumerate}
	\item Welk bedrag ontvangt de zoon na de laatste storting?

	\item Welk bedrag had de vader \'{e}\'{e}nmalig op die rekening
    kunnen plaatsen, toen de zoon 12 jaar was, zodat hij op zijn
    achttiende net hetzelfde bedrag zou ontvangen?
    \end{enumerate}

    \item Jef stort elk jaar een vast bedrag op een
    rekening, de eerste maal als hij 15 jaar wordt en de laatste maal
    als hij 30 wordt. Als hij 35 jaar wordt, wil hij over een bedrag
    van \euros{10\,000} beschikken. De jaarlijkse rente over die periode is \SI{6}{\percent}.
    Hoeveel moet hij elk jaar storten?



    \item  Een twintigjarige stort jaarlijks \euros{250} op een rekening die
   \SI{10,5}{\percent}  opbrengt. De eerste storting gebeurt op zijn
   twintigste verjaardag en de laatste op zijn
   vierendertigste verjaardag. Van dan af gebeuren er geen
   stortingen meer en blijft het geld gewoon op de rekening staan.
   \begin{enumerate}
       \item  Wat is het saldo van de rekening op zijn
   veertigste verjaardag? 

        \item  Op zijn veertigste verjaardag geeft de
        zoon een groot feest. Daarna blijft er nog \euros{1\,000} op zijn
        rekening staan. Met dit bedrag wil de zoon de volgende jaren
        op reis gaan, telkens rond zijn verjaardag, en elke keer wil
        hij eenzelfde bedrag uitgeven. Welk bedrag kan de zoon
        aan elk van die 5 reizen besteden?
        \item  Welk bedrag staat er nog op zijn rekening na
    zijn drie\"{e}nveertigste verjaardag? 
    \end{enumerate} 
    
   
    \item De ouders van Lieve sparen sinds haar geboorte
    \euros{50} per maand en zullen dat doen tot en met haar
    achttiende verjaardag, aan een jaarlijkse rente van
    \SI{5,2}{\percent}. Als Lieve 19 is studeert ze verder. Hoeveel geld (steeds 
    een vast bedrag) van
    het gespaarde kan ze jaarlijks besteden (jaarlijkse rente \SI{3}{\percent}) als ze  voorziet om 8 jaar te studeren? Ze haalt het geld steeds op haar verjaardag af, te beginnen bij haar negentiende verjaardag.

    \item  Iemand wil een bedrag lenen van \euros{4000}. Hij is bereid
   om elk jaar, de eerste maal na \'{e}\'{e}n jaar, \euros{200}
   af te betalen.
Hoeveel volledige jaren moet die persoon \euros{200} afbetalen,
       om tenminste de volledige schuld af te betalen. Werk
       nu met een jaarlijkse rente van \SI{4}{\percent}.


    \item Men wil een kapitaal van \euros{25\,000} vormen door jaarlijks
    een bedrag van \euros{2500} te beleggen aan \SI{4,5}{\percent}. Bepaal het aantal
    keren dat dit bedrag moet gestort worden indien het vereiste
    kapitaal bekomen moet zijn:
    \begin{enumerate}
    \item bij de laatste storting?
    \item 3 jaar na de laatste storting?
    \end{enumerate}

    \item  Er zijn 2 kopers A en B voor eenzelfde huis. Koper A stelt 
    voor:  een voorschot van \euros{27\,500}, en daarna 20 jaarlijkse 
    stortingen van
    \euros{2000}, de eerste na 6 maanden. Koper B heeft volgend 
    voorstel: een voorschot van 
    \euros{33\,500}, en daarna 15 jaarlijkse stortingen van elk
    \euros{3250}, de eerste over 1 jaar. Welke koper is voor de eigenaar
    het meest interessant? De jaarlijkse rente bedraagt \SI{5}{\percent}.


    \item  Iemand stort de eerste van elke maand \euros{50} op een
    rekening, te beginnen vanaf 1 juli 2002.  Vanaf 1 december 2002
    voorziet hij een aantal grote aankopen (kerstaankopen) en
    spaart een aantal maanden niets. Vanaf 1 juli 2003 kan hij opnieuw
    om de 2 maanden \euros{75} op diezelfde rekening zetten. De
    laatste storting gebeurt op 1 mei 2004. De jaarlijkse rente is \SI{7}{\percent}.
    \begin{enumerate}
        \item  Bereken hoeveel geld er op 1 mei 2004 op de rekening
        staat. 

        \item  Wat is de waarde van alle stortingen op 1 juli 2002?
    \end{enumerate}


      \item   Jan stort op 1 juli 1999 een bedrag van \euros{10\,000} en op 1
    juli 2003 stort hij nogmaals een bedrag van \euros{2000} op
    diezelfde rekening. Zijn vriend Jef begint te sparen vanaf 1
    januari 2000 en stort elke semester eenzelfde bedrag T.  Jef beweert
    dat hij op 1 juli 2008, na zijn laatste storting, evenveel op
    zijn rekening zal hebben als zijn vriend Jan op datzelfde tijdstip.
    De jaarlijkse rente is \SI{5}{\percent}.  Bereken het bedrag T dat Jef elke semester stort.
    
     \item Je spaart maandelijks \euros{100} op een spaarrekening
    (jaarlijkse rente \SI{4,5}{\percent}). Je vriend Mathieu
    spaart \euros{75} per maand, met een jaarlijkse intrest van
    \SI{6,5}{\percent}.
    \begin{enumerate}
        \item Welke zijn de maandelijkse intresten (in \%)?
        \item Wanneer zal hij meer gespaard hebben dan jou?
    \end{enumerate}

    \item Miriam kan een auto (waarde \euros{15\,000}) leasen aan
    \euros{300} per maand (jaarlijkse rente \SI{8}{\percent}). Hoe lang mag ze de auto
    maximaal leasen om niet boven de waarde van een effectieve
    aankoop te gaan? Ze doet de eerste aflossing bij ingebruikname.

    \item Joris is zopas aangeworven
    bij KBC Bank\&Verzekeringen, maar denkt nu reeds aan zijn
    pensioen. KBC zelf stelt een pensioenfonds voor aan \SI{6}{\percent} jaarlijkse
    intrest. Joris denkt gedurende 40 jaar te werken vooraleer op
    pensioen te gaan en tijdens die periode maandelijks een bepaald
    bedrag te sparen.
    \begin{enumerate}
	\item Eens op pensioen, zou hij graag gedurende 20 jaar
	jaarlijks \euros{7000} afnemen van het gespaarde bedrag (de
	intrest blijft \SI{6}{\percent}, opname telkens in het begin van ieder jaar).
	Hoeveel moet hij in totaal gespaard hebben?
	\item Hoeveel moet hij maandelijks sparen zolang hij werkt (inlage
	telkens op het einde van iedere maand)?
    \end{enumerate}

    \item Met Kerstmis vierde je je 20ste verjaardag. Van je
    vrienden kreeg je een Lotto-biljet cadeau. En jawel, je won de
    superpot van \euros{1,4 miljoen}. Je geeft een feestje en het
    resterende bedrag (\euros{1,24 miljoen}) zet je op 6 januari op een
    spaarboekje (jaarlijkse rente \SI{6}{\percent}). Er zijn twee mogelijkheden:
    \begin{enumerate}
    \item (staat los van vraag $a$) 
    Je bent een feestneus. Elk jaar op 6 januari geef je een feest(je) voor je vrienden, de eerste keer na je 21ste verjaardag, de laatste keer na je 40ste verjaardag. Hoeveel mag dat feestje kosten als alle geld op moet zijn?
    \item  Je bent een investeerder. Op je 40ste begin je het geld te
    investeren in het familiebedrijfje: elk jaar  op 6 januari 
    investeer je een bedrag van
    \euros{250\,000}. Hoeveel keer kan je die
    investering (van \euros{250\,000}) doen?
    \end{enumerate}
    
    \item Iemand wint het groot lot. Hij schenkt onmiddellijk aan 
    familie en vrienden \euros{500\,000}. Gedurende tien jaar gaat hij 
    om de twee maanden op reis voor een bedrag van \euros{20\,000} (eerste keer: 2 maand na de schenking; laatste keer: 10 jaar na de schenking). Na 
    deze tien jaar komt hij echter tot inzicht dat geld niet gelukkig 
    maakt en wordt prompt een boeddhistische monnik. Vijftien jaar 
    later sterft de oude monnik en laat heel zijn erfenis na aan het 
    klooster. Het blijkt om \euros{2\,000\,000} te gaan. Hoeveel geld 
    had de man oorspronkelijk gewonnen? De jaarlijkse rente bleef de 
    hele tijd constant \SI{4}{\percent}.

    \item Vandaag vier je je 20ste verjaardag. Van je vrienden
    kreeg je een tombolabiljet `Win for Life' cadeau en het blijkt
    dat je de hoofdprijs gewonnen hebt. Te beginnen met vandaag, stort
    de Nationale Loterij \emph{om de 2 jaar} \euros{12\,500} op je rekening. Je
    besluit dit geld telkens op je spaarboekje te zetten en op je
    84ste verjaardag (de dag na de storting van dat jaar) uit te keren
    aan je kleinkinderen. De jaarlijkse rente bedraagt \SI{6}{\percent}.
    \begin{enumerate}
    \item Welk bedrag zullen je kleinkinderen erven?
    \item Na hoeveel stortingen staat er 1 miljoen \euro op je spaarboekje?
    \end{enumerate}
    
   \item  Ik koop een huis. De bank stelt me voor een lening aan te gaan met looptijd 20 jaar en aflossingsbedrag \euros{537,50} per maand. De eerste aflossing vindt plaats 1 maand na de start van de lening. De jaarlijkse rente bedraagt \SI{5,34}{\percent}. Hoeveel geld heb ik geleend? (\euros{80\,000}).

\item Een klant koopt een PC van \euros{1000} op 1/6/2009 op afbetaling. Bij aankoop betaalt hij een voorschot van \euros{150}. Op 1/7/2009 betaalt hij de eerste inning van de 44 aflossingen. De laatste inning betaalt hij op 1/2/2013. De jaarlijkse rente bedraagt \SI{15,58}{\percent}. Welk bedrag moet er maandelijks afgelost worden?


\end{enumerate}


%\end{document}
