%%%%%%%%%%%%%%%%%%%%%%%%%%%%%%%%%
% Laatste aanpassing:           %
% 10/09/01 door Greetje
%   2 grafieken ingevoegd
% 01/09/01 door Greetje          %
%%%%%%%%%%%%%%%%%%%%%%%%%%%%%%%%%

%\documentclass[10pt]{report}
%\usepackage[dutch]{babel}
%\usepackage[eurosym,right]{eurofont}
%\include{packages}
%\begin{document}

%\chapter{Oefeningen op functies en grafieken}
\section{Oefeningen}
\subsection{Herhalingsvragen}
\begin{enumerate}
    \item  Hoe zoek je het domein van een functie uit het voorschrift?

    \item  Wat heb je nodig om een tekenstudie te maken van een
    veeltermfunctie?

    \item  Dit is  de \emph{definitie} van INT($x$):
    \begin{quote}
        INT($x$) met $x$ een re\"{e}el getal is het grootste
            geheel getal \emph{dat kleiner is dan of gelijk aan $x$}
    \end{quote}
    Vergelijk met de functie ``floor'' in Mathcad. Bereken nu de INT van de
    getallen $-3.14$; $7.5$; $-2$; $5.99$. Zoek nu zelf een correcte definitie van \emph{absolute waarde} van
    een getal.
\end{enumerate}


\subsection{Opgaven}

\begin{enumerate}
    \item \label{opg:funca}
    \emph{Teken de grafieken van volgende functies}\\
    Om grafieken te tekenen zoek je enkele functiewaarden
    en verbind je die zinvol. Als je het verloop van de functie al
    min of meer kent, kan je je tevreden stellen met een minimaal
    aantal punten. Soms is het nodig het voorschrift te splitsen naar
    gelang het interval waarop je werkt.

   \begin{eqnarray}
       y & = & -2x^{2}+4x-1
       \label{eq:gr1}  \\
       f(x) & = & |x-2|
       \label{eq:gr2}  \\
       g(x) & = & \frac{1}{-2x+3}
       \label{eq:gr3}  \\
       y & = & \mbox{INT}(x)
       \label{eq:gr4}  \\
       h(x) & = & \sqrt{x^{2}-4}
       \label{eq:gr5}  \\
       k(z) & = & z^{3}-2z
       \label{eq:gr6}  \\
       y & = & \left \{
       \begin{array}{cc}
          -2  & \mbox{als $x\geq -1$}  \\
           -x^{2}-1 & \mbox{als $x<-1$}
       \end{array}
       \right.
             \label{eq:gr7}
   \end{eqnarray}

   \begin{enumerate}
        \item  Bekijk grafiek die hoort bij functievoorschrift~(\ref{eq:gr6}). Bereken
        de functiewaarden $k(2)$  en $k(-2)$; $k(3)$  en $k(-3)$. Wat valt er op?
    Formuleer je resultaat in een formule.

        \item  Probeer voor de functie~(\ref{eq:gr2}) een meervoudig
        voorschrift te vinden zonder absolute waarde.
    \end{enumerate}

    \item  In deze vraag vertrekken we van een gegeven grafiek en we
    zoeken een voorschrift dat erbij past. In figuur~\ref{fig:par}
    vind je de grafiek van een functie. Welke veelterm hoort vermoedelijk bij
    deze grafiek? Verklaar. Bepaal de belangrijke punten die de
    functie volledig determineren. Zoek het functievoorschrift.
    \begin{figure}[htb]
        \centering
        \setlength{\unitlength}{1mm}
        \begin{picture}(70,71)
        \put(0,25){\vector(1,0){70}} \put(15,10){\vector(0,1){60}}
        \put(65,27){$X$} \put(17,66){$Y$} \qbezier(15,55)(35,-25)(55,55)
        \put(15,25){\dashbox{2}(40,30){}} \put(15,15){\dashbox{2}(20,10){}}
        \put(34,27){$2$} \put(54,20){$4$} \put(8,14){$-1$}
        \put(11,54){$3$}
        \end{picture}
        \caption{Grafiek van een veeltermfunctie}
        \label{fig:par}
    \end{figure}

    \item  Bekijk figuur~\ref{fig:meervoudig}. Zoek een
    meervoudig voorschrift dat hoort bij deze functie.
    Kan je dit meervoudig voorschrift vervangen door
    \'{e}\'{e}n uitdrukking? Vergelijk met \'{e}\'{e}n van de
    grafieken uit opgave~\ref{opg:funca}.
    \begin{figure}[ht]
            \centering
        \setlength{\unitlength}{1mm}
    \begin{picture}(70,51)
    \put(0,5){\vector(1,0){70}} \put(50,0){\vector(0,1){50}}
    \put(30,5){\line(1,1){35}} \put(30,5){\line(-1,1){25}}
    \put(22,0){$(-2,0)$} \put(38,25){$(0,2)$} \put(30,5){\circle*{2}}
    \put(50,25){\circle*{2}} \put(65,7){$X$} \put(52,46){$Y$}
    \end{picture}
        \caption{Grafiek van een meervoudige functie}
        \label{fig:meervoudig}
    \end{figure}

    \item  Een rol behangpapier is $10$ meter lang. Stel de
    veranderlijke  $t$ gelijk
    aan de hoeveelheid behangpapier die je nodig hebt, uitgedrukt in
    meter. Stel $f(t)$ het aantal rollen dat je moet kopen om aan de
    hoeveelheid $t$ te voldoen. Teken de grafiek van $f(t)$
    met $t\in [0,65]$. Zoek indien mogelijk een voorschrift voor deze
    functie.

    \item Bepaal het domein en het beeld van de volgende functies.
    Zoek eventuele nulpunten. Waar is de functie positief en
    negatief? Controleer je antwoorden met Mathcad.

    \noindent
    {\setlength{\extrarowheight}{18pt}
    \begin{tabular}{p{6.cm}p{6.cm}}
    $f_1(x)=(3x-4)(8x^2-22x+15)$ & $\displaystyle{f_2(x)=\sqrt{\frac{x^2+14x+49}{x^2-1}}}$\\
    $f_3(x)=(2x+1)(x^2-4x-5)$    & $\displaystyle{f_4(x)=x+\frac{14}{x+4}}$\\
    $f_5(x)=(-7x^2+x)(7x^2-8x+1)$& $\displaystyle{f_6(x)=\frac{x^2-5x+6}{x^2-7x+12}}$ \\
    $f_7(x)=(x^2+6x+9)(x^2-1)$   & $\displaystyle{f_8(x)=\frac{x^2-8x+16}{1-\sqrt x}}$\\
    \end{tabular}}
    \begin{displaymath}
    f_9(x)=\left\{
        \begin{array}{l}
            \displaystyle{\frac3{4-x}~\mbox{als}~x<3}\\
            \displaystyle{\frac{2x}{\sqrt{x^2-5}}}
        \end{array} \right.
   \end{displaymath}

   \item Zoek het snijpunt van volgende paren van functies.

   \noindent
    {\setlength{\extrarowheight}{18pt}
   \begin{tabular}{p{3.5cm}cp{6.cm}}
        $f_a(x)=2x^2-5x-6$ &en& $f_1(x)=3x+4$ \\
        $f_b(x)=x^2-10x+9$ &en& $f_2(x)=x-9$\\
        $f_c(x)=x-2$ &en& $\displaystyle{f_3(x)=\frac{12+2x-x^2}{x-2}}$\\
        $f_d(x)=3x^2+9$ &en& $f_4(x)=2x^2-5x+3$\\
        $f_e(x)=30x^3-3x^2$ &en& $f_5(x)=16x^3+25x^2$\\
        $\displaystyle{f_f(x)=\frac{x^2-8}{x+2}}$ &en& $f_6(x)=x^2+4x-4$\\
        $\displaystyle{f_g(x)=\frac{x-10}{-x+5}}$ &en& $f_7(x)=x^2+x+2$
   \end{tabular}}
\end{enumerate}

