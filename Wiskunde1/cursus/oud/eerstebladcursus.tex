\documentclass[11pt]{article}
\usepackage[dutch]{babel}
\usepackage{boxedminipage}
\begin{document}
    \noindent
\newcommand{\groot}[1]{{\large{\bf #1}}}
\setlength{\fboxsep}{.7cm} \noindent
\begin{boxedminipage}{\textwidth}
\begin{center}{\Large{\bf Informatieblad}}\end{center}
\groot{Opleiding:} {\bf 1 Toegepaste Informatica} \\ \\
\groot{Opleidingsonderdeel:} {\bf Wiskunde} \\ \\
\groot{Titularissen:} {\bf G.~Jongen, J.~Van~Hee en R.~Wyseur}
\end{boxedminipage}

\vspace{0.7cm} \noindent
    Wat de inhoud is van deze cursus vind je in de inhoudstafel maar
    we geven graag wat \textbf{accenten} mee zodat je weet waar je
    v\'{o}\'{o}r  staat.\\

    \noindent
    In dit vak beklemtonen we
    \begin{itemize}
    \item \textbf{logisch redeneren en zelfcontrole} en niet zozeer rekenvaardigheden.
    De zelfcontrole moet effici\"{e}nt gebeuren, o.a.\ via gebruik van wiskundige
    software;

    \item \textbf{oefeningen}. Ze helpen je begrippen
    en algoritmen aangebracht in de theorie op een grondige manier
    te verwerken. Bovendien leer je problemen geformuleerd in woorden
    om te zetten naar wiskundige vergelijkingen;

    \item  zowel het \textbf{analytisch} berekenen (via exacte
    methoden) als  het \textbf{benaderend} rekenen (numerieke iteratieve methoden,
    progranmmeerbaar) als het \textbf{grafisch} illustreren
    (grafieken). Deze verschillende werkwijzen om een
    probleem aan te pakken vullen elkaar goed aan.

    \end{itemize}

    \noindent
    We verwachten van je dat je \textbf{actief} aan de lessen \textbf{meewerkt}:
    \begin{itemize}
      \item je \textbf{analyseert}, \textbf{concretiseert}
      en/of \textbf{illustreert} de gedoceerde wiskundige begrippen door
      oefeningen op te lossen tijdens de les in kleine groepjes;
        \item je verwerkt \textbf{zelfstandig nieuwe leerstof}
        via werkteksten met doelgerichte opdrachten;
        \item je probeert thuis \textbf{nieuwe oefeningen}
        op te lossen.
    \end{itemize}
    \begin{quote}
    Feedback wordt op eenvoudige vraag gegeven.
    \end{quote}


   \noindent Je behaalt je eindscore voor dit vak op volgende manier:
    \begin{enumerate}
    \item 30~\% van de punten staan op permanente evaluatie (PE). Deze
    punten verdien je \textbf{tijdens} het academiejaar. Meer
    informatie vind je in het ``Informatieblad Permanente
    Evaluatie'';
    \item  de overige 70~\% van de punten verdien je tijdens het eindexamen in juni.
    Alleen de leerstof die niet in de werkteksten aan bod gekomen
    is, wordt hier ondervraagd.
    \end{enumerate}
\newpage

\setlength{\fboxsep}{.7cm} \noindent
\begin{boxedminipage}{\textwidth}
\begin{center}{\Large{\bf Informatieblad Permanente Evaluatie}}\end{center}
\groot{Opleiding:} {\bf 1 Toegepaste Informatica} \\ \\
\groot{Opleidingsonderdeel:} {\bf Wiskunde} \\ \\
\groot{Titularissen:} {\bf G.~Jongen, J.~Van~Hee en R.~Wyseur}
\end{boxedminipage}

\vspace{0.7cm} \noindent Quoteringen voor permanente evaluatie
worden toegekend op:
    \begin{itemize}
        \item  een test in het eerste semester. Het
        doel is dat je kennismaakt met de manier van
        vragen stellen en quoteren.
        De nadruk ligt op kwalitatieve feedback en niet zozeer
        op punten;

        \item  het persoonlijk verwerken via een begeleidende werktekst
        van enkele numerieke methoden. Deze opdracht wordt in de
        loop van het eerste semester gegeven. De punten die je met dit
        werkje verdient, vormen het hoofdaandeel van de permanente evaluatie;

        \item enkel kleine opdrachten die in de loop van het
        academiejaar worden opgegeven;

        \item de manier waarop je omspringt met feedback die
        gegeven wordt bij tussenresultaten van de grote opdracht
        of eindresultaten van de kleine opdrachten;

        \item werkzaamheid in de klas.

      \end{itemize}

\noindent Volgende afspraken gelden hierbij:
\begin{itemize}
\item de opdrachten worden \textbf{tijdens de lessen} gegeven.
Je bent zelf verantwoordelijk of je al dan niet tijdig op de
hoogte bent van een gegeven opdracht. Enkel in geval van overmacht
(bijv.\ ziekte) kunnen we uitzonderingen voorzien;

\item je geeft de opdracht steeds \textbf{persoonlijk} aan je docent af
(dus niet via de correspondentiebakjes);

\item je levert je werkje af \textbf{op het afgesproken tijdstip} (dag en uur).
\end{itemize}

\begin{flushleft}
\noindent Deze tekst is ook te vinden op EXPORT ON FESTER in de
mappen \tt{jahee/wiskunde}, \tt{rowys/wiskunde} en
\tt{majog/1ti/wiskunde}.
\end{flushleft}


\end{document}

