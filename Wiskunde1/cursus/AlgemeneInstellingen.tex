\documentclass[DIV=calc,BCOR=1cm]{scrbook}

\usepackage{showlabels}
\usepackage[utf8]{inputenc}
\usepackage[dutch]{babel}
\usepackage{graphicx}
\usepackage{fancyvrb}
\usepackage[labelsep=space,hang,small,bf]{caption}
\usepackage{array}
\usepackage{enumerate}
\usepackage{answers} %Voor oefeningen en oplossingen
\usepackage{amsthm}
\usepackage{subfig}
\usepackage{amsmath}
\usepackage{textcomp} %voor euro met nieuw command \euros{}
\usepackage[utopia]{quotchap}
\usepackage{latexsym}
\usepackage{varioref}
\usepackage{amscd}
\usepackage{epsfig}
\usepackage{array}
%\usepackage{psboxit}
\usepackage{float}
\usepackage{boxedminipage}
\usepackage[scaled]{beramono}
\usepackage[T1]{fontenc}
\usepackage{xcolor}
%\linespread{1.02}
\usepackage[output-decimal-marker={,},quotient-mode = fraction,per-mode=symbol]{siunitx}
\usepackage{listings}
\usepackage{booktabs}
\usepackage{multirow} 
\usepackage{multicol} 
\usepackage{wrapfig}
\usepackage{microtype}
\usepackage{tikz}
\usepackage{pdfpages}
%\usepackage[adobe-utopia]{mathdesign}
\usepackage[bitstream-charter]{mathdesign}
%\usepackage[colorlinks,citecolor=spot,linkcolor=spot,urlcolor=spot]{hyperref} %Voor screenpdf
\usepackage{hyperref}


\usetikzlibrary{patterns,topaths,arrows,calc,decorations.markings,positioning}
\frenchspacing

% Kleuren. Voor het printen in enkel zwart en wit
% kan je kleuren herdefiniëren. Dit wordt dan automatisch
% aangepast, zowel in \textcolor commando's, als in
% TikZ tekening met \draw[kleur ...] ...
\definecolor{grijs}{rgb}{0.9,0.9,0.9}
\definecolor{donkergroen}{RGB}{1,74,0}
\definecolor{spot}{rgb}{0,0.1,0.6}
%\definecolor{chaptergrey}{RGB}{84,106,150}
%\definecolor{orange}{RGB}{0,0,0}

% Geen schreefloze in titels enz.
\setkomafont{sectioning}{\rmfamily\bfseries\boldmath}
\setkomafont{descriptionlabel}{\rmfamily\bfseries}
%\addtokomafont{disposition}{\color{spot}} %hoofdstuktitels in blauw voor screenpdf

%%%%%%%%%%%% splitsing %%%%%%%%%%%%%%
\hyphenation{Kar-naugh-af-beel-ding}





%%%%%% Tellers %%%%%%
\setcounter{secnumdepth}{2}
\setcounter{tocdepth}{2}

%%%%%% Environments %%%%%%
\newenvironment{bewijs}{\noindent{\textbf{Bewijs}} \\}{\proofend}
\newenvironment{stelling}{\noindent{\textbf{Stelling}} }{}
\newenvironment{definitie}{\noindent{\textbf{Definitie}} }{}

%%%%%%%%% Voor oefeningen en oplossingen %%%%%%%%%%%
\Newassociation{opl}{Oplossing}{ans}
\theoremstyle{definition} % want anders inhoud van oefeningen cursief
\newtheorem{oef}{Oefening}[chapter]
\Opensolutionfile{ans}[ans1]


%%%%%% commands %%%%%%
\newcommand{\euros}[1]{\texteuro~#1}

\newcommand{\dif}{\text{d}}
\newcommand{\Dif}[1]{\text{D} \left( #1 \right)}
\newcommand{\Dom}{\text{Dom}}
\newcommand{\proofend}{\begin{flushright} $\Box$ \end{flushright}}
\newcommand{\vraagis}{\stackrel{?}{=}}
\newcommand{\en}{\wedge}
\newcommand{\of}{\vee}
\newcommand{\niet}{\neg}
\newcommand{\alsdan}{\rightarrow}
\newcommand{\asa}{\leftrightarrow}
\newcommand{\verbspatie}{\vspace{0.2cm}}
\newcommand{\opdracht}[1]{\begin{quote}#1\marginpar[]{$\triangleleft$}\end{quote}}
\newcommand{\veld}[1]{{\tt #1}}
\newcommand{\knop}[1]{``{\it #1}''}
\newcommand{\func}[1]{{\tt #1}}
\newcommand{\D}{{\rm D}}
\newcommand{\mysection}[1]{\vspace{1.em} \noindent {\bf {\large #1}}}
\newcommand{\im}[1]{\begin{itemize} #1 \end{itemize}}
\newcommand{\mym}[1]{${\displaystyle #1}$}

\newcommand{\voorbeeld}{
\refstepcounter{vb}%
\subsubsection*{Voorbeeld~\thesection.\arabic{vb}}
}

\newcommand{\vbsection}[1]{\setcounter{vb}{0}\section{#1}}

%%%%%%% nodig voor aanmaak bibliografie %%%%%%%
\def\bbt#1{\bibitem{#1} \label{bb:#1}}
\setlength{\unitlength}{1pt}
\bibliographystyle{unsrt}

%\input kvmacros nodig voor Karnaugh

\lstset{language=Scilab,basicstyle=\ttfamily \footnotesize,backgroundcolor=\color{white},
frame=single,framerule=0.5pt,rulecolor=\color{gray},tabsize=2,numbers=none,showstringspaces=false}
\lstdefinestyle{inline} {basicstyle=\normalsize \ttfamily}