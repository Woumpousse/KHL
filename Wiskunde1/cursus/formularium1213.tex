\documentclass[10pt]{article}
\input kvmacros
\newcommand{\en}{\wedge}
\newcommand{\of}{\vee}
\newcommand{\niet}{\neg}
\newcommand{\alsdan}{\rightarrow}
\newcommand{\asa}{\leftrightarrow}


\textwidth16.cm%\oddsidemargin-.cm 
%\textheight23.cm
%\topmargin-.cm
\newcommand{\mysection}[1]{\vspace{1.em} \noindent {\bf {\large #1}}}
\newcommand{\im}[1]{\begin{itemize} #1 \end{itemize}}
\newcommand{\mym}[1]{${\displaystyle #1}$}
\newcommand{\D}{{\rm D}}

\usepackage{multicol}
\usepackage{palatino}
\usepackage{mathpazo}

\begin{document}
\renewcommand{\arraystretch}{1.5} % extra verticale ruimte in tabellen

\pagestyle{empty}
\noindent
{\bf {\huge Formularium Wiskunde}}

\vspace{1.5em}

%\begin{multicols}{1}
\mysection{Nulpunten kwadratische functie} \\

\noindent
{\bf Voorschrift:} \mym{ y(x)=ax^2+bx+c}\\ \\
{\bf Nulpunten:} \mym{x_{1,2}=\frac{-b\pm \sqrt{b^2-4ac}}{2a}}
\\

\mysection{Exponenti\"ele \& logaritmische functies} \\

\begin{tabular}{l}
\noindent \mym{\log_g(x)=y\Leftrightarrow g^y=x} \\
\end{tabular}\\

\noindent {\bf Rekenregels}

\begin{tabular}{l}
\mym{g^{x+y}=g^x\cdot g^y}\\
\mym{\left(g^x\right)^y=\left(g^y\right)^x=g^{x\cdot y}}\\
\mym{g^{-x}=\frac1{g^x}}\\
\mym{\log_g(x\cdot y)=\log_g(x)+\log_g(y)}\\
 \mym{\log_g(x^y)=y\log_g(x)}\\
\mym{\log_g\left(\frac1x\right)=-\log_g(x)}
\end{tabular}\\

\mysection{Meetkunde} \\

\noindent
\textbf{Omtrek cirkel:} $2\cdot \pi R$\\
\textbf{Oppervlakte cirkel: } $\pi R^2$
\\

\mysection{Propositielogica}\\

\begin{tabular}{cc}
en & $\en$ \\  
of & $\of$ \\ 
niet & $\niet$ \\ 
\end{tabular} 




%\end{multicols}

\newpage
%\topmargin-1.cm
%\textheight29.cm


\noindent
{\bf {\huge Formularium Scilab}}

\vspace{0.5em}

\begin{multicols}{2}
\mysection{Rekenen in Scilab}\\

\begin{tabular}{p{4.cm}l} Vierkantswortel: & \verb/sqrt(x)/\\ Macht: &
\verb/x^y/\\ Absolute waarde: & \verb/abs(x)/\\ $\ln(x)$&\verb/log(x)/\\
\mym{e^x}&\verb/exp(x)/\\ $\pi$&\verb/%pi/ \\
$true,~false$&\verb/%T, %F/
\end{tabular}

\mysection{Logische operatoren}\\

\begin{tabular}{|c|c|}
\hline

{\sc en, of} & \verb/*, + / of \verb/ &, |/ \\ \hline

$\neq,~=$ & \verb/~=, ==/ \\ \hline

$<,~>,~\leq$    & \verb/<, >, <=/ \\ \hline

\end{tabular}


\end{multicols}
\mysection{Vectoren}\\

\begin{tabular}{|p{7.3cm}|p{5.5cm}|}
\hline

definitie vector    &   \verb+V=[1,1.5,2,2.5,3]=1:0.5:3+ \\

vectorelement       &   \verb+V(2)+\\

vector met $4$ elementen, alle gelijk aan
$0$   resp. $1$              & \verb+zeros(4,1)+, \verb+ones(4,1)+ \\

lengte van de vector $V$ & \texttt{length(V)}\\

\hline
\end{tabular}

\vspace{0.5cm}

\mysection{Functies}\\

\begin{tabular}{p{4.cm}l}
Defini\"eren: &\verb/function y=mijnFunctie(x)/\\
            & \verb/y=2*x / \\
            & \verb/endfunction /\\
Tekenen:    & \verb/clf; xgrid; x=0:10; plot(x,mijnFunctie)/ \\

\end{tabular}\\

\mysection{Programmeren}\\

\begin{tabular}{p{4.cm}l}
Lussen & \verb/for y=1:x ... end/ of \verb/while y<x ... end/ \\
Voorwaardelijk statement & \verb/if x==2 ... elseif ... else ... end/ \\
\end{tabular}\\

%\vfill
\begin{center}  Academiejaar 2012--2013 \end{center}

%\end{multicols}
\end{document}
