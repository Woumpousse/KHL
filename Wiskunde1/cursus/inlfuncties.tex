%%%%%%%%%%%%%%%%%%%%%%%%%%%%%%%%%%%%%%%%%%%%%%%%%%%%%%%%%%%%%
% Laatste aanpassingen:           				            
% 
% 9/9/02 [Jan]: Verwijzingen naar Mathcad weggewerkt.       
%
% 04/08/02 door Roby                
%   tekeningen aangepast aan Euler  
%
% 01/09/01 door Greetje             
%%%%%%%%%%%%%%%%%%%%%%%%%%%%%%%%%%%%%%%%%%%%%%%%%%%%%%%%%%%%%

    \chapter{Functies: Overzicht}

    \begin{quote}
     \textit{ `Zo doe je dat,' zei de Koningin zeer beslist, `niemand
     kan twee dingen tegelijk doen, zoals je weet. Laten we om te
     beginnen eens bedenken hoe oud je bent -- hoe oud ben je?'}

     \textit{`Ik ben op de kop af zeven en een half.'}

     \textit{`Die kop hoef ik er niet bij,' merkte de Koningin op.
     `Ik geloof het zo ook wel. Nu zal ik \emph{jou} iets te geloven
     geven. Ik ben precies honderd en \'{e}\'{e}n jaar, vijf maanden
     en een dag.'}

     \textit{`\emph{Dat} geloof ik niet!' zei Alice.}

     \textit{`O nee?' zei de Koningin op medelijdende toon. `Probeer
     het nog eens: haal diep adem en doe je ogen dicht.'}

     \textit{Alice lachte. `Proberen heeft geen zin,' zei ze,
     `onmogelijke dingen \emph{kun} je niet geloven.'}

     \textit{`Je hebt vast te weinig geoefend,' zei de Koningin.
     `Toen ik zo oud was als jij, deed ik het elke dag een half uur.
     Het kwam zelfs voor dat ik nog voor het ontbijt maar liefst zes
     onmogelijke dingen geloofde.'}


          Uit `Achter de spiegel' -- Lewis Carroll
\end{quote}

\newpage
    \section{Definities}
    \subsection{Re\"{e}le functies}
     Een re\"{e}le functie $f$ is een \emph{verzameling} koppels re\"{e}le
    getallen $(x,y)$, waarbij men $x$ de onafhankelijk veranderlijke
    (argument) noemt en $y=f(x)$ de afhankelijk  veranderlijke
    (functiewaarde). Het verband
    tussen $x$ en $y$ kan weergeven worden door een \emph{voorschrift, een tabel
    of door een grafiek}. Elk van deze voorstellingen
    heeft zijn voordelen en levert andere inzichten. Een grafiek is
    \emph{globaal en overzichtelijk}, een voorschrift is
    \emph{analytisch en puntgericht}. Met Euler zullen we
    regelmatig grafische
    voorstellingen maken bij allerhande problemen.

    De begrippen domein, beeld, nulpunten, stijgend (dalend), maximum (minimum),
    positief of negatief kunnen zowel analytisch via het
    voorschrift als grafisch worden omschreven. We proberen de
    begrippen te verduidelijken aan de hand van een eenvoudig voorbeeld,
    namelijk de
    (veelterm)functie
    \begin{equation}
        y=f(x)=-3x^{2}+6x+9
        \label{eq:functievb}
    \end{equation}

    De grafiek~\ref{fig:para} van functie~(\ref{eq:functievb})
    is een \emph{parabool} en het
    voorschrift is een tweedegraads veelterm.
        \begin{figure}[tbp]
                    \includegraphics[width=11cm]{figuren/euler_inlfuncties/fig_parabool.pdf}
                  \caption{Grafiek van $f(x)=-3x^{2}+6x+9$}
            \label{fig:para}
        \end{figure}

    \subsection{Het domein van een functie}
    \begin{itemize}
        \item  We gebruiken de notatie $dom(f)$ voor het
        domein van een functie. Het is de verzameling van alle $x$-waarden waarvoor de
        functie goed gedefinieerd is. \emph{Concreet:} delen door 0
        en de vierkantswortel uit een negatief getal kan niet bij
        re\"{e}le getallen. Voor dit voorbeeld\footnote{Dit is
        trouwens algemeen geldig voor elke willekeurige \emph{veelterm}functie.} kan je eenvoudig inzien
        dat je geen enkele beperking hebt qua $x$-waarden die je mag
        invullen in de functie. We noteren: $dom(f)=\mathbb{R}$.

        \item Grafisch is $dom(f)$ de deelverzameling van de $x$-as die we
        vinden door alle punten van de grafiek te projecteren op
        de $x$-as.
    \end{itemize}

    \subsection{De beeldverzameling van een functie}
    \begin{itemize}
        \item  We gebruiken de notatie $bld(f)$ voor de
        beeldverzameling van een functie. Het beeld van een functie is de verzameling van alle
        functiewaarden $f(x)$, met $x$ behorend tot het domein.
    Dit betekent voor dit voorbeeld: zoek welke waarden de
        uitdrukking $ -3x^{2}+6x+9$ kan aannemen, als je elke
        mogelijke $x$ (uit het domein) zou invullen. Deze opdracht
        is meestal niet gemakkelijk uit te voeren.

        \item  $Bld(f)$ is een deelverzameling van de \emph{$y$-as}, die
        we vinden door alle grafiekpunten te projecteren op de $y$-as.
        Op grafiek~\ref{fig:para} lezen we af dat $bld(f)=]-\infty
        , 12]$
    \end{itemize}


    \subsection{Nulpunt van een functie}
    \begin{itemize}
        \item  Een nulpunt van een functie is een $x$-waarde
        waarvoor de functiewaarde gelijk is aan 0. Als $f(a)=0$, dan
        noemen we $a$ een nulpunt van de functie $f(x)$. Zo kan je
        narekenen voor voorbeeld~(\ref{eq:functievb}) dat $-1$ een
        nulpunt is, want $f(-1)=-3\cdot (-1)^{2}+6\cdot (-1)+9=0$.
    Een nulpunt
        bepalen betekent oplossingen zoeken van de
        \emph{vergelijking} $f(x)=0$. De methode om vergelijkingen op
        te lossen, beschrijven we in paragraaf~\ref{sec.vgl}.

        \item  Grafisch is een nulpunt de $x$-waarde of de eerste
        co\"{o}rdinaat van een \emph{snijpunt} van de grafiek met de $x$-as.
    Zo zie je op figuur~\ref{fig:para} dat
    functie~(\ref{eq:functievb}) twee nulpunten heeft.
    \end{itemize}


    \subsection{Het teken van een functie: + of $-$}
    \begin{itemize}
        \item  Hier zoeken we de $x$-waarden waarvoor de functiewaarden
        \emph{positief of negatief zijn.} Analytisch betekent dit dat
        we volgende \emph{ongelijkheid} moeten oplossen: $f(x)\geq 0$
        of $f(x) \leq 0$ .Voor functie~(\ref{eq:functievb}) is de verzameling van de
        $x$-waarden waarvoor de functiewaarden positief zijn gelijk
        aan $[-1,3]$. De methode om ongelijkheden op te lossen zien we
        verder (paragraaf~\ref{sec.ongel}).

        \item  Grafisch  de ongelijkheid  $f(x)\geq 0$ oplossen
        betekent de grafiekpunten die \emph{boven de $x$-as liggen}
    projecteren op de $x$-as. Voor dit voorbeeld wordt dat het interval $[-1,3]$.

    \end{itemize}


    \subsection{Stijgen en dalen van een functie}
    \begin{itemize}
        \item  Een functie is stijgend ($\nearrow$ ) op een interval $[a,b]$ als
    $\forall x, y \in [a,b]$, waarvoor $x<y$, geldt dat $f(x)<f(y)$ .\\
    Een functie is dalend ($\searrow$ ) op een interval $[a,b]$ als
    $\forall x, y \in [a,b]$, waarvoor $x<y$, geldt dat $f(x) >f(y)$

        \item  Grafisch zien we dat de functie stijgend
        (\emph{dalend}) is op een
        interval als de functiewaarden naar boven (\emph{onder})
        bewegen als de $x$-waarden naar \emph{rechts} bewegen.
    \end{itemize}


    \newpage
    \section{Algebra\"{i}sche functies}
Er zijn drie soorten algebra\"{i}sche functies: veeltermfuncties,
rationale functies en irrationale functies. Bij elk van de drie geven
we een voorbeeld en bespreken we kort het domein en de grafiek.

\subsection{Veeltermfuncties}
\subsubsection{Voorbeeld}
\begin{equation}
    f(x)=-3x^{3}+2x-6
    \label{eq:veeltermfunctie}
\end{equation}
Functie~(\ref{eq:veeltermfunctie}) noemen we een \emph{veeltermfunctie}.
De veranderlijke $x$ komt enkel voor als grondtal van machten met een
positieve exponent.

\subsubsection{Domein}
Zoals reeds vermeld is het domein van elke willekeurige
veeltermfunctie $\mathbb{R}$. Je mag met andere woorden elk re\"{e}el
getal $x$ invullen in het functievoorschrift.

\subsubsection{Grafiek}
De grafiek van een veeltermfunctie is een vloeiende ononderbroken
kromme die kan stijgen of dalen. Figuur~\ref{fig:veeltermfunctie}
toont de grafiek van functie~(\ref{eq:veeltermfunctie}).
\begin{figure}[tbp]
    \centering
    \includegraphics[width=11cm]{figuren/euler_inlfuncties/fig_derdegraads.pdf}
    \caption{Grafiek van $f(x)=-3x^{3}+2x-6$}
    \label{fig:veeltermfunctie}
\end{figure}


\subsection{Rationale functies}
\subsubsection{Voorbeeld}
\begin{equation}
    g(x)=\frac{2x^{2}+4}{3x-1}
    \label{eq:rationalefunctie}
\end{equation}
Functie~(\ref{eq:rationalefunctie}) noemen we een \emph{rationale
functie} of breukfunctie.
De veranderlijke $x$ komt ook voor in de noemer of als grondtal met
een negatieve exponent.

\subsubsection{Domein}
Delen door $0$ is niet toegestaan. Daarom moeten we alle
$x$-waarden die de noemer nul maken, uitsluiten uit het domein. We
formuleren het vaak wat bondiger als volgt: $dom(g) = \mathbb{R}
\setminus \{ \mbox{nulpunten van de noemer} \}$. Reken zelf na dat
de volgende voorwaarde geldig moet zijn: $x \neq \frac{1}{3}$.

\subsubsection{Grafiek}
Er kunnen `gaten' zijn in de grafiek. Rond sommige $x$-waarden (hier
$x=\frac{1}{3}$) kunnen
de functiewaarden naar $\pm \infty$ gaan. Figuur~\ref{fig:rationalefunctie}
toont de grafiek van functie~(\ref{eq:rationalefunctie}).
\begin{figure}[tbp]
    \centering
    \includegraphics[width=11cm]{figuren/euler_inlfuncties/fig_rationale.pdf}
    \caption{Grafiek van $\displaystyle{g(x)=\frac{2x^{2}+4}{3x-1}}$}
    \label{fig:rationalefunctie}
\end{figure}




\subsection{Irrationale functies}
\subsubsection{Voorbeeld}
\begin{equation}
    h(x)=\frac{\sqrt{2x+3}}{3x+1}
    \label{eq:irrationalefunctie}
\end{equation}
Functie~(\ref{eq:irrationalefunctie}) noemen we een \emph{irrationale
functie} of wortelfunctie.
De veranderlijke $x$ staat onder het wortelteken.

\subsubsection{Domein}
Naast eventuele nulpunten van de noemer, moet je ook de $x$-waarden
die de uitdrukking onder een (even) machtswortel negatief
maken, uitsluiten uit het domein. Noemer verschillend van nul levert:
$x \neq -\frac{1}{3}$. De uitdrukking onder de wortel zal positief of
nul zijn als $x\geq -\frac{3}{2}$. We besluiten:
\begin{displaymath}
    dom(h)=[-\frac{3}{2},+\infty) \backslash \{-\frac{1}{3}\}
\end{displaymath}


\subsubsection{Grafiek}
De grafiek bestaat meestal slechts op een deelinterval van de
re\"{e}le rechte. Figuur~\ref{fig:irrationalefunctie}
toont de grafiek van functie~(\ref{eq:irrationalefunctie}).
\begin{figure}[tbp]
    \centering
    \includegraphics[width=11cm]{figuren/euler_inlfuncties/fig_irrationale.pdf}
    \caption{Grafiek van $\displaystyle{h(x)=\frac{\sqrt{2x+3}}{3x+1}}$}
    \label{fig:irrationalefunctie}
\end{figure}




\newpage
\section{Vergelijkingen oplossen}\label{sec.vgl}
\subsection{Inleiding}
    We beginnen met de gemakkelijkste vergelijkingen, namelijk de
    veeltermvergelijkingen, daarna de rationale vergelijkingen. We
    geven ook een voorbeeld van een irrationale vergelijking.
    Algebra\"{i}sche vergelijkingen kunnen ontstaan bij het
    zoeken van nulpunten van functies of het bepalen van
    snijpunten van functies,\ldots




\subsection{Veelterm- en rationale vergelijkingen}
\subsubsection{Algemene werkwijze}
    \begin{enumerate}
        \item  Breng de vergelijking onder \emph{standaardvorm}
        $\frac{T(x)}{N(x)}=0$. De teller $T(x)$ en de noemer $N(x)$ zijn
        veeltermen.

        \item  \emph{Ontbind teller en noemer in factoren }. Enkele
        belangrijke rekenregels zijn: gemeenschappelijke
        factoren voorop zetten, merkwaardige producten
        $a^{2}-b^{2}=(a-b)(a+b)$, gebruik van de discriminant,\dots
        De formule voor de discriminant is:
    \begin{displaymath}
        Dis=b^{2}-4\cdot a\cdot c
    \end{displaymath}
    De nulpunten worden gegeven door
    \begin{displaymath}
        x_{1}=\frac{-b+\sqrt{b^{2}-4\cdot a\cdot c}}{2\cdot a}
    \end{displaymath}
    en
    \begin{displaymath}
        x_{2}=\frac{-b-\sqrt{b^{2}-4\cdot a\cdot c}}{2\cdot a}
    \end{displaymath}

        \item  \emph{Bestaansvoorwaarde:} De noemer moet verschillend van
        nul zijn.
    Dan stellen we de teller gelijk aan nul. We lossen de
    veeltermvergelijking op.

        \item \emph{ Controleer} dat de nulpunten van de teller \textbf{geen}
        nulpunten zijn van de noemer.
    \end{enumerate}


\subsubsection{Voorbeeld}
    \emph{Zoek de snijpunten van de 2 functies
    $f(x)=\frac{2x^{2}+5x-3}{x^{2}+x-2}$ en $g(x)=4x-2$.}

        De snijpunten van 2 functies $f(x)$ en $g(x)$ vinden we door het
        volgend stelsel op te lossen.
    \begin{displaymath}
        \left\{
        \begin{array}{c}
            y=f(x)  \\
            y=g(x)
        \end{array} \right.
    \end{displaymath}
    Dit stelsel lossen we op door gelijkstelling: $f(x)=g(x)$.
    \begin{enumerate}
        \item \emph{ Standaardvorm:}
        \begin{displaymath}
            \frac{2x^{2}+5x-3}{x^{2}+x-2}-(4x-2)=0
        \end{displaymath}

        \item  \emph{Ontbind in factoren} 
        \begin{eqnarray*}
            \frac{(2x-1)(x+3)}{(x+2)(x-1)}-\frac{2(2x-1)(x+2)(x-1)}{(x+2)(x-1)}  & = & 0  \\
            \frac{2x-1}{(x+2)(x-1)}\cdot (x+3-2x^{2}-2x+4) & = & 0  \\
            \frac{2x-1}{(x+2)(x-1)}\cdot (-2x^{2}-x+7) & = & 0
        \end{eqnarray*}

        \item  \emph{Bestaansvoorwaarden}: $x\neq -2$ en $x\neq 1$. Nu
        stellen we de teller gelijk nul. We lossen op:
    \begin{displaymath}
        (2x-1)(-2x^{2}-x+7)=0
    \end{displaymath}
    Discriminant van de kwadratische term is $1-4\cdot (-2)\cdot 7=57$
    \\ Nulpunten van de teller zijn:\\
    $x_{1}=\frac{1}{2}$ of $x_{2}=\frac{1-\sqrt{57}}{-4}=1.638$ of
    $x_{3}=\frac{1+\sqrt{57}}{-4}=-2.138$

        \item  \emph{Controleren} of de oplossingen voldoen aan de
        bestaansvoorwaarden. Alle nulpunten van de teller zijn
        verschillend van $-2$ en $1$.
    \end{enumerate}
    %\parindent
    \underline{Besluit}: De  $y$-co\"{o}rdinaten van de snijpunten van de functies
    $f(x)$ en $g(x)$ vinden we door de $x$-waarden  uit (3) in te vullen in \'{e}\'{e}n van de
    2 functies, hier best $g(x)$ (de functie met het eenvoudigste
    voorschrift). De snijpunten zijn:
    \begin{eqnarray*}
        a & = & (1.638, 4.552)  \\
        b & = &  (-2.138 , -10.552) \\
        c & = & (0.5,0)
    \end{eqnarray*}
    De snijpunten vind je terug op de grafiek~\ref{fig:snijp}.\\
    \begin{figure}[hbp]
            \centering
        \includegraphics[width=11cm]{figuren/euler_inlfuncties/fig_snijpunten_functies.pdf}
        \caption{Grafiek van de functies $f(x)$ en $g(x)$}
        \label{fig:snijp}
    \end{figure}



    \newpage
    \section{Oplossen van ongelijkheden}\label{sec.ongel}
    \subsection{Algemene werkwijze}
    \begin{enumerate}
        \item  We herleiden de ongelijkheid naar de \emph{standaardvorm}
        $\frac{T(x)}{N(x)}\geq 0$ of  $\frac{T(x)}{N(x)} \leq 0$ met
        teller en noemer veeltermen.

        \item  We \emph{ontbinden} teller en noemer in factoren. Gelijke
        factoren best niet wegdelen, want dan verdwijnen er nulpunten.

        \item  \emph{Tekenstudie} maken van de teller en de noemer
        afzonderlijk. Daarna het tekenverloop van de ganse
        breuk berekenen. \emph{Opmerking}: voor een tekenstudie
        hebben we opnieuw de nulpunten van de teller en noemer nodig.
    Die lezen we af uit de ontbinding in factoren.

    Regel: Rond
        een nulpunt verandert de functie meestal van teken tenzij het
        nulpunt een even aantal keren voorkomt in de uitdrukking. Deze regel kan
        je gemakkelijk op een grafiek controleren. Ofwel gaat de
        grafiek door de $x$-as (en is er een tekenverandering)
    ofwel raakt die aan de $x$-as (geen tekenverandering).

        \item  Uit het tekenverloop lezen we af voor welke
        verzameling de
        uitdrukking ofwel positief of negatief is.
    \end{enumerate}


    \subsection{Voorbeeld}
    \begin{quote}
         Zoek alle $x$-waarden waarvoor de grafiek van de functie $h(x)$
    onder de $x$-as ligt.
    \end{quote}
    \begin{displaymath}
        h(x)=\frac{5x+2}{2x^{3}-3x^{2}-2x+3}
    \end{displaymath}
    \textbf {Oplossing}
    \begin{enumerate}
        \item  \emph{De standaardvorm}: De ongelijkheid kan direct in
        standaardvorm geschreven worden.
        \begin{displaymath}
            \frac{5x+2}{2x^{3}-3x^{2}-2x+3}\leq 0
        \end{displaymath}

        \item \emph{ De ontbinding in factoren:}
        \begin{displaymath}
            \frac{5x+2}{(x-1)(2x-3)(x+1)}\leq 0
        \end{displaymath}

        \item  We maken een \emph{tekenstudie} van alle factoren van de
        uitdrukking. In de bovenste rij zetten we alle nulpunten in
        stijgende volgorde. De tekenstudie vind je in
        tabel~\ref{tbl:teken}

    %\begin{math}
            \begin{table}[htb]
            \centering
            \caption{Tekenstudie}
        \begin{tabular}{|c|c|c|c|c|c|c|c|c|c|}
        \hline
        $x$ &  & -1 &  & -0.4 &  & 1 &  & 1.5 &   \\
        \hline \hline
        $5x+2$ & - & - & - & 0 & + & + & + & + & +  \\
        \hline
        $x+1$ & - & 0 & + & + & + & + & + & + & +  \\
        \hline
        $x-1$ & - & - & - & - & - & 0 & + & + & +  \\
        \hline
        $2x-3$ & - & - & - & - & - & - & - & 0 & +  \\
        \hline \hline
        breuk & + & / & - & 0 & + & / & - & / & +  \\
        \hline
    \end{tabular}
        \label{tbl:teken}
    \end{table}

    %\end{math}



    De verticale strepen betekenen dat het bijhorende punt $x$ een nulpunt
    is van de noemer en niet behoort tot het domein van de uitdrukking.

        \item  We lezen de oplossing af uit de laatste rij van de tabel.
        De functie $h(x)$ is \emph{negatief} op 2 intervallen.
        Besluit: de functie $h(x)$ ligt onder de $x$-as op het
        gebied $]-1 , -0.4]\quad  \bigcup \quad  ]1,1.5[$. Dit
        oplossingsgebied lees je ook af op figuur~\ref{fig:ong}
    \end{enumerate}
    \begin{figure}[tbp]
            \centering
        \includegraphics[width=11cm]{figuren/euler_inlfuncties/fig_rationale_vgl.pdf}
        \caption{Oplossing van de rationale ongelijkheid}
        \label{fig:ong}
    \end{figure}
     Teken zelf met Euler de grafiek. Verander eens het interval op
     de $y$-as en je zal zien dat de functiewaarden rond $x=1$ en
     $x=1.5$ heel groot worden. De functiewaarden leunen aan bij een
     \emph{verticale asymptoot}.



\newpage
     \section{Irrationale vergelijkingen}

     \textbf{Voorbeeld}
     \begin{quote}
          We zoeken de snijpunten van de functie $f(x)=-\sqrt{1-x^{2}}$ en
     de eerste bissectrice.
     \end{quote}
     Op de grafiek~\ref{fig:irrat} zien we dat $f(x)$ een
     halve cirkel voorstelt met middelpunt (0,0) en straal 1. De
     halve cirkel ligt onder de $x$-as omwille van het minteken
     v\'{o}\'{o}r de
     wortel. De vergelijking van de eerste bissectrice is $y=x$.
    \begin{figure}[tbp]
            \centering
        \includegraphics[width=11cm]{figuren/euler_inlfuncties/fig_irrationale_vgl.pdf}
        \caption{Grafiek van de irrationale functie en de bissectrice}
        \label{fig:irrat}
    \end{figure}
     Bij het oplossen van irrationale vergelijkingen moeten we
     regelmatig voorwaarden stellen. Dit is
     de reden waarom wiskundige software soms foute oplossingen geeft. De
     vergelijking die we hier moeten oplossen is
     \begin{equation}
         -\sqrt{1-x^{2}}=x
         \label{eq:ir1}
     \end{equation}

     \begin{enumerate}
         \item \emph{ Bestaansvoorwaarden:} Alle uitdrukkingen onder
         de vierkantswortel moeten positief zijn. De noemer moet
         verschillend zijn van 0.
     \begin{eqnarray}
        1-x^{2}  & = & 0
         \label{eq:ir2}  \\
        (1-x)(1+x)   & = & 0
         \label{eq:ir3}
     \end{eqnarray}

     Het tekenverloop vind je in tabel~\ref{tbl:1-x^{2}}
     \begin{table}[h]
              \centering
              \caption{Tekenverloop}
         \begin{tabular}{|c|c|c|c|c|c|c|c|}
         \hline
         $x$ &$-\infty$ &  & -1 &  & 1 & & $\infty$   \\
         \hline \hline
         $1-x$ &+ & +  & + & + & 0 & -& -  \\
         \hline
         $1+x$ &- & - & 0 & + & + & +& +  \\
         \hline \hline
         $1-x^{2}$ & - &  - & 0 & + & 0 & -& -  \\
         \hline
     \end{tabular}
         \label{tbl:1-x^{2}}
     \end{table}

     Dit betekent dat $x$ moet behoren tot het interval $[-1,1]$.

         \item  Herleid de vergelijking zodat \'{e}\'{e}n  wortel
         alleen staat in \'{e}\'{e}n lid van de vergelijking.
     In deze vergelijking is geen herschikking nodig.

         \item  We mogen lid aan lid kwadrateren op voorwaarde dat
         beide leden hetzelfde teken hebben, want bij het kwadrateren
         wordt alles positief.\\ \emph{Voorbeeld}
         \begin{displaymath}
             -3\neq 3 \hspace {0.5cm} \mbox{maar} \hspace{0.5cm}
             (-3)^{2}=3^{2}
         \end{displaymath}
     Omdat het linkerlid van de vergelijking~\ref{eq:ir1}
     voor alle $x$-waarden
     \emph{negatief }is,
     eisen we dat het rechterlid ook negatief is. \\
     \emph{Kwadrateringsvoorwaarde KV}: $ x\leq 0$


         \item  Nu kwadrateren we beide leden. De
         vierkantswortels verdwijnen en we lossen de
         algebra\"{i}sche vergelijking op.
     \begin{eqnarray*}
         1-x^{2} & = & x^{2}  \\
         x^{2} & = & \frac{1}{2}
     \end{eqnarray*}
     Hieruit vinden we als voorlopige oplossingen
     $x_{1}=-\frac{1}{\sqrt{2}}$ en $x_{2}=+\frac{1}{\sqrt{2}}$.

         \item  \emph{Controle}: De oplossingen uit (4) moeten voldoen
         aan de bestaansvoorwaarden uit (1) en de KV uit (3). \\
     Aangezien $x_{2}=+\frac{1}{\sqrt{2}}$ niet voldoet aan de KV is er
     slechts \'{e}\'{e}n oplossing en bijgevolg slechts \'{e}\'{e}n snijpunt van de
     halve cirkel met de bissectrice. \\
     $(-\frac{1}{\sqrt{2}},-\frac{1}{\sqrt{2}})$ is het enige snijpunt.
     Het snijpunt kan je aflezen op figuur~\ref{fig:irrat}


     \end{enumerate}

%\end{document}
