%%%%%%%%%%%%%%%%%%%%%%%%%%%%%%
%
% 5 sept 2013 [Jan] SAP-code, nieuwe collega, academiejaar aangepast
%
%%%%%%%%%%%%%%%%%%%%%%%%%%%%%%%

\chapter*{Studiewijzer}


\section*{Cursusfiche}
\begin{description}
\item[Academiejaar]2013--2014
\item[Opleiding]Bachelor in de Toegepaste informatica
\item[Opleidingsonderdeel]Toegepaste wiskunde 1 -- MBI71A
\item[Studiepunten]3
\item[Contacturen]26  (1 uur les in gewoon klaslokaal en 1 uur les in PC-klas per week gedurende 13 lesweken)	
\item[Semester]	1 
\item[SBU per studiepunt]	25
\item[Lectoren]I. Hoornaert, G. Jongen en J. Van Hee
\item[Beoogde competenties]$\qquad$

\begin{itemize}
\item De student verwerft wiskundige basisvaardigheden die nodig zijn bij andere opleidingsonderdelen.
\item De student kan gestructureerd een probleem aanpakken en het volledig en met zorg afwerken
\item De student kan zelf wiskundige eigenschappen ontdekken via gebruik van grafieken, berekeningen en gebruik van wiskundige software. Hij kan die eigenschappen ook correct toepassen in concrete situaties.
\item De student kan een algoritme zinvol toepassen en de beperkingen ervan in concrete situaties onderzoeken en bespreken.
\item De student kan adequaat vragen stellen, inspelen op antwoorden en kan bondig en overzichtelijk (schematisch) conclusies formuleren.
\item De student kan positief inspelen op gegronde feedback om zo het eigen leerproces bij te sturen. 
\end{itemize}

\item [Leerinhoud]$\qquad$

\begin{itemize}
\item \emph{Inleiding tot de verzamelingenleer}: definitie, `element zijn van', bewerkingen
\item \emph{Relaties}: definitie, soorten relaties, inverse relatie
\item \emph{Eerstegraadsfuncties}: wiskundig model, grafiek, functievoorschrift, snijpunt van twee rechten
\item \emph{Lineaire Programmering}: maximum-minimum problemen met beperkingen
\item \emph{Exponenti\"{e}le groei}: verband met lineaire groei, groeifactor en procentuele groei
\item \emph{Exponenti\"{e}le en logaritmische functies}: verloop van de functies en oplossen van eenvoudige exponenti\"{e}le en logaritmische vergelijkingen
\item \emph{Propositielogica}: proposities, waarheidswaarde, waarheidstabellen;
\item \emph{Meetkunde}: software P.e.L., eigenschappen van meetkundige figuren zoals veelhoeken en  cirkels, macro's maken en meetkundige plaatsen zoeken
\item \emph{Extremumproblemen}:  dynamische en grafische oplossing in PeL (zonder afgeleiden) 
\item \emph{Grafieken tekenen}: software Scilab
\end{itemize}

\item [Werkvormen]$\qquad$

\begin{itemize}
\item \emph{Overdracht}: activerend hoorcollege
\item \emph{Inoefenend}: student verwerkt leerstof met begeleide oefeningen

\end{itemize}



\item [Studiemateriaal]	$\qquad$

\begin{itemize}
\item Cursus `Toegepaste wiskunde 1': I. Hoornaert, G.~Jongen en J. Van Hee,  KHLeuven departement Gezondheidszorg \& Technologie, 2013
\item Website Meetkunde: http://wiskunde.khleuven.be 
\item Elektronisch leerplatform Toledo 
\end{itemize}

\item [Evaluatie]	In de studiegids vind je de aanduidingen van examentijdstippen, vorm van het examen, aandeel van de examenonderdelen (zoals permanente evaluatie, opdrachten en contactexamen).   Meer info vind je in de cursusbeschrijving die volgt.

\end{description}

\newpage
\section*{Cursusbeschrijving}
\subsection*{Situering van het opleidingsonderdeel}

Het opleidingsonderdeel `Toegepaste wiskunde 1' ondersteunt andere opleidingsonderdelen. Op een systematische manier worden begrippen, technieken en procedures ge\"{i}ntro\-du\-ceerd \emph{in een eenvoudige wiskundige omgeving}. Diezelfde competenties duiken later elders in het curriculum op in een meer complexe omgeving.
 
Er worden wiskundige begrippen en rekenvaardigheden opgefrist en aangeleerd die essentieel zijn voor iedere informaticus in spe. Verder komen eenvoudige problemen aan bod. De oplossingsmethoden daarvan bestaan uit opeenvolgende  stappen: 
\begin{enumerate}
\item wat is het probleem?
\item welke elementen zijn gekend om het probleem op te lossen?
\item hoe zet je dat om naar wiskundige uitdrukkingen?
\item welke stappen moeten gezet worden om tot de oplossing te komen?
\end{enumerate}

Op die manier wordt \emph{het logisch denken en de zelfcontrole} aangescherpt. Vraagstukken spelen tenslotte een belangrijke rol gezien de analogie met de analyse van een informaticaprobleem. 


\subsection*{Studiepunten}

Aan het opleidingsonderdeel (OPO) `Toegepaste wiskunde 1' zijn 3 studiepunten toegekend. Dit betekent dat je $3 \cdot 25=75$ uren moet werken voor dit OPO. Na aftrek van examen en lestijden komt dit neer op ongeveer 2,5 uur studietijd thuis per week.

\subsection*{Organisatie van de lessen}

Je hebt 2 keer 1 uur les theorie en oefeningen per week, afwisselend in een gewoon lokaal of PC-klas. Dit betekent dat verschillende hoofdstukken aan bod komen in de loop van \'{e}\'{e}n week. 

De docent geeft een kort overzicht over de begrippen of de oplossingsmethode die in die les aan bod komt. Daarna ga je zelf aan de slag. Er wordt veel aandacht besteed aan de \emph{logische structuur} van je oplossing. Je moet het ook mondeling kunnen toelichten. 

In de PC-klassen leer je werken met de software. Nadien gebruik je die om begeleid maar toch \emph{zelfstandig} meetkundige problemen op te lossen en oefeningen te illustreren.

\subsection*{Evaluatie}

De 20 punten voor dit opleidingsonderdeel worden volledig toegekend tijdens het  \emph{contactexamen} in januari. 
Het contactexamen is mondeling/schriftelijk met schriftelijke voorbereiding en legt de nadruk op probleemoplossend vermogen. Je mag gebruik maken van het formularium  en de softwarepakketten P.e.L.\ en Scilab.  Een kritische bespreking van je resultaat is belangrijk.
 
De docent geeft vooraf voldoende informatie over de vorm, de eisen qua structuur en volledigheid en de inhoud van het contactexamen.

\subsubsection*{Tweede examenkans (augustus)}
Je kan een tweede examenkans benutten om toch een credit te halen voor dit opleidingsonderdeel, of om je resultaat van januari te verbeteren.

