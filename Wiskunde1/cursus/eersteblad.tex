%%%%%%%%%%%%%%%%%%%%%%%%%%%%%%%
% 14/9/05 [Jan]: aanpassingen, nieuwe vaknaam, Wim Bertels bijgevoegd,
%  academiejaar vervangen door semester, examens vervangen door examen,...
%
% 19/09/2004 [Jan]: aanpassing aan nieuw curriculum en verbeteren fouten
%
% 18 sept 03 [Jan]: foutjes verbeterd
%
% 18 sept 2002 [Jan]: bestand ge�ntegreerd in cursus, hoofding 
%	verzorgd met boxedminipage. Commentaren van Greetje weggelaten.
%
% aanpassingen van Greetje op 4/9/02 en 11/9
%
% Eerste blz. cursus wiskunde 1ti
% aangemaakt door Roos in september 2002
%%%%%%%%%%%%%%%%%%%%%%%%%%%%%%%

    \noindent
\setlength{\fboxsep}{.2cm} \noindent
\begin{boxedminipage}{\textwidth}
\begin{center}{\Large{\bf Informatieblad}}\end{center}
\textbf{Opleiding:} 1 Toegepaste informatica \\ \\
\textbf{Onderdeel:} Wiskunde - deel I \\ \\
\textbf{Titularissen:} W.~Bertels, J.~Van~Hee en R.~Wyseur
\end{boxedminipage}


\vspace{0.7cm} \noindent
    Wat de inhoud is van deze cursus vind je in de inhoudstafel maar
    we geven graag wat \textbf{accenten} mee zodat je weet waar je
    v\'{o}\'{o}r  staat.\\

    \noindent
    In dit vak beklemtonen we
    \begin{itemize}
    \item \textbf{logisch redeneren en zelfcontrole} en niet zozeer rekenvaardigheden.
    De zelfcontrole moet effici\"{e}nt gebeuren, o.a.\ via gebruik van wiskundige
    software;

    \item \textbf{oefeningen}. Ze helpen je begrippen
    en algoritmen aangebracht in de theorie op een grondige manier
    te verwerken. Bovendien leer je problemen geformuleerd in woorden
    om te zetten naar wiskundige vergelijkingen;

    \item  zowel het \textbf{analytisch} berekenen (via exacte
    methoden) als  het \textbf{benaderend} rekenen (numerieke iteratieve methoden,
    programmeerbaar) als het \textbf{grafisch} illustreren
    (grafieken). Deze verschillende werkwijzen om een
    probleem aan te pakken vullen elkaar goed aan.

    \end{itemize}

    \noindent
    We verwachten van je dat je \textbf{actief} aan de lessen \textbf{meewerkt}:
    \begin{itemize}
      \item je \textbf{analyseert}, \textbf{concretiseert}
      en/of \textbf{illustreert} de gedoceerde wiskundige begrippen door
      oefeningen op te lossen tijdens de les in kleine groepjes;
        \item je verwerkt \textbf{zelfstandig nieuwe leerstof}
        via werkteksten met doelgerichte opdrachten;
        \item je probeert thuis \textbf{nieuwe oefeningen}
        op te lossen.
    \end{itemize}
    \begin{quote}
    Feedback wordt op eenvoudige vraag gegeven. In de begeleidingsuren is er plaats voor een persoonlijke aanpak. Je kan hier individueel of samen met medestudenten vragen stellen, een moeilijke oefening bekijken, hulp vragen bij een minder eenvoudig stukje leerstof. Samen zoeken we hier naar die vorm van begeleiding waar je het meeste baat bij hebt. Soms betekent dit dat de docent je aanmoedigt en desnOPOs verplicht tot een gesprek. Durf in elk geval naar je docent stappen als je met vragen of opmerkingen zit. We zijn er om jou te helpen!
    \end{quote}


   \noindent Je behaalt je eindscore voor dit vak op volgende manier:
    \begin{enumerate}
    \item 30~\% van de punten staan op permanente evaluatie (PE). Deze
    punten verdien je \textbf{tijdens} het semester. Meer
    informatie vind je in het ``Informatieblad Permanente
    Evaluatie'';
    \item  de overige 70~\% van de punten verdien je tijdens het eindexamens in januari.
    Alleen de leerstof die niet in de werkteksten aan bod gekomen
    is, wordt hier ondervraagd.
    \end{enumerate}
\newpage

    \noindent
\setlength{\fboxsep}{.2cm} \noindent
\begin{boxedminipage}{\textwidth}
\begin{center}{\Large{\bf Permanente Evaluatie}}\end{center}
\textbf{Opleiding:} 1 Toegepaste informatica \\ \\
\textbf{Onderdeel:} Wiskunde - deel I\\ \\
\textbf{Titularissen:} W.~Bertels, J.~Van~Hee en R.~Wyseur
\end{boxedminipage}

\vspace{0.7cm} \noindent Quoteringen voor permanente evaluatie
worden toegekend op:
    \begin{itemize}
      
        \item  het persoonlijk verwerken via een begeleidende werktekst
        van een wiskundig onderwerp;

        \item enkele kleine opdrachten die in de loop van het
        semester worden opgegeven;

        \item de manier waarop je omspringt met feedback die
        gegeven wordt bij tussenresultaten van grote opdrachten 
        of eindresultaten van kleine opdrachten;

        \item werkzaamheid in de klas.

      \end{itemize}

\noindent Volgende afspraken gelden hierbij:
\begin{itemize}
\item de opdrachten worden \textbf{tijdens de lessen} gegeven.
Je bent zelf verantwoordelijk of je al dan niet tijdig op de
hoogte bent van een gegeven opdracht. Enkel in geval van overmacht
(bijv.\ ziekte) kunnen we uitzonderingen voorzien;

\item je geeft de opdracht steeds \textbf{persoonlijk} aan je docent af
(dus niet via de correspondentiebakjes);

\item als er een verslagje verwacht wordt, dan voldoet dit vanzelfsprekend aan de afgesproken regels die daaromtrent in de opleiding gelden (voorblad, duidelijke vermelding persoonlijke gegevens, goed taalgebruik,...);

\item je levert je werkje af \textbf{op het afgesproken tijdstip} (dag en uur).
\end{itemize}




