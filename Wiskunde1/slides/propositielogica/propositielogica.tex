\documentclass[dutch]{../khlslides}
\usepackage{graphicx}
\usepackage{pxfonts}
\usepackage{tikz}
\usepackage{calc}
\usepackage{fourier}

\usetikzlibrary{calc,shadows,decorations.markings}


\title{Propositielogica}
\logo{\includegraphics[height=0.5cm]{../KHL.jpg}}
\institute[KHL]{KHLeuven}

\newcommand{\AND}{\wedge}
\newcommand{\OR}{\vee}
\newcommand{\IMPLIES}{\rightarrow}
\newcommand{\IFF}{\leftrightarrow}
\newcommand{\NOT}{\neg}
\newcommand{\union}{\cup}
\newcommand{\intersect}{\cap}


\pgfkeys{
  /tikz/.cd,
  axis/.style={thin,-latex},
  plot/.style={domain=-5:5,thick}
}

\begin{document}

\maketitle

\begin{frame}
  \frametitle{Algebra\"ische Structuren}
  \structure{Wat is een Algebra\"ische Structuur?}
  \begin{itemize}
    \item Een verzameling $A$
    \item Operaties op elementen van $A$
  \end{itemize}
\end{frame}

\begin{frame}
  \frametitle{Voorbeeld \#1: Re\"ele Getallen}
  \begin{itemize}
    \item Verzameling: $\mathbb{R}$
    \item Elementen: re\"ele getallen
    \item Operaties
          \begin{center}
            \begin{tabular}{lc}
              \bf Naam & \bf Notatie \\
              \toprule
              Optelling & $x + y$ \\
              Aftrekking & $x - y$ \\
              Vermenigvuldiging & $x \cdot y$ \\
              Deling & $x / y$ \\
              Machtsverheffing & $x^y$
            \end{tabular}
          \end{center}
  \end{itemize}
\end{frame}

\begin{frame}
  \frametitle{Voorbeeld \#2: Verzamelingen}
  \begin{itemize}
    \item Verzameling: verzameling der verzamelingen
    \item Elementen: verzamelingen
    \item Operaties
          \begin{center}
            \begin{tabular}{lc}
              \bf Naam & \bf Notatie \\
              \toprule
              Unie & $A \union B$ \\
              Doorsnede & $A \intersect B$ \\
              Verschil & $A - B$ \\
              Cartesisch product & $A \times B$
            \end{tabular}
          \end{center}
  \end{itemize}
\end{frame}

\begin{frame}
  \frametitle{Voorbeeld \#3: Booleanse Algebra}
  \begin{itemize}
    \item Verzameling: $\left\{ \; \mathbf{true}, \mathbf{false} \; \right\}$
    \item Elementen: $\mathbf{true}$ en $\mathbf{false}$
    \item Kortere notatie: $1$ en $0$, respectievelijk
    \item Operaties
          \begin{center}
            \begin{tabular}{lcl}
              \bf Naam & \bf Notatie & \bf Uitspraak \\
              \toprule
              Conjunctie & $p \AND  q$ & $p$ en $q$ \\
              Disjunctie & $p \OR  q$ & $p$ of $q$ \\
              Negatie & $\NOT p$ & niet $p$ \\ 
              Implicatie & $p \IMPLIES  q$ & als $p$ dan $q$ \\
              Equivalentie & $p \IFF  q$ & $p$ als en slechts als $q$
            \end{tabular}
          \end{center}
          \vskip2mm
    \item Booleaanse uitdrukkingen heten proposities
  \end{itemize}
\end{frame}

\begin{frame}
  \frametitle{Waarheidstabellen}
  \begin{center}
    \Large\bf Conjunctie
  \end{center}
  \Large
  \[
    \begin{array}{ccc}
      p & q & p \AND q \\
      \toprule
      0 & 0 & 0 \\
      0 & 1 & 0 \\
      1 & 0 & 0 \\
      1 & 1 & 1 \\
    \end{array}
  \]
\end{frame}

\begin{frame}
  \frametitle{Waarheidstabellen}
  \begin{center}
    \Large\bf Disjunctie
  \end{center}
  \Large
  \[
    \begin{array}{ccc}
      p & q & p \OR q \\
      \toprule
      0 & 0 & 0 \\
      0 & 1 & 1 \\
      1 & 0 & 1 \\
      1 & 1 & 1 \\
    \end{array}
  \]
\end{frame}

\begin{frame}
  \frametitle{Waarheidstabellen}
  \begin{center}
    \Large\bf Negatie
  \end{center}
  \Large
  \[
    \begin{array}{ccc}
      p & \NOT p \\
      \toprule
      0 & 1 \\
      1 & 0 \\
    \end{array}
  \]
\end{frame}

\begin{frame}
  \frametitle{Waarheidstabellen}
  \begin{center}
    \Large\bf Implicatie
  \end{center}
  \Large
  \[
    \begin{array}{ccc}
      p & q & p \IMPLIES q \\
      \toprule
      0 & 0 & 1 \\
      0 & 1 & 1 \\
      1 & 0 & 0 \\
      1 & 1 & 1 \\
    \end{array}
  \]
\end{frame}

\begin{frame}
  \frametitle{Waarheidstabellen}
  \begin{center}
    \Large\bf Equivalentie
  \end{center}
  \Large
  \[
    \begin{array}{ccc}
      p & q & p \IFF q \\
      \toprule
      0 & 0 & 1 \\
      0 & 1 & 0 \\
      1 & 0 & 0 \\
      1 & 1 & 1 \\
    \end{array}
  \]
\end{frame}


\begin{frame}
  \frametitle{Oefening Waarheidstabellen}
  \large
  \[
    p \IMPLIES (\NOT q \IMPLIES p \OR q \OR r)
  \]
  
  \[
    \begin{array}{cccc@{\quad}c@{\quad}c@{\quad}c}
      p & q & r & \NOT q & p \AND q \AND r & \NOT q \IMPLIES p \OR q \OR r & p \IMPLIES (\NOT q \IMPLIES p \OR q \OR r) \\
      \toprule
      0 & 0 & 0 & \only<2->{1} & \only<10->{0} & \only<12->{0} & \only<13->{1} \\
      0 & 0 & 1 & \only<3->{1} & \only<10->{1} & \only<12->{1} & \only<13->{1} \\
      0 & 1 & 0 & \only<4->{0} & \only<10->{1} & \only<11->{1} & \only<13->{1} \\
      0 & 1 & 1 & \only<5->{0} & \only<10->{1} & \only<11->{1} & \only<13->{1} \\
      1 & 0 & 0 & \only<6->{1} & \only<10->{1} & \only<12->{1} & \only<13->{1} \\
      1 & 0 & 1 & \only<7->{1} & \only<10->{1} & \only<12->{1} & \only<13->{1} \\
      1 & 1 & 0 & \only<8->{0} & \only<10->{1} & \only<11->{1} & \only<13->{1} \\
      1 & 1 & 1 & \only<9->{0} & \only<10->{1} & \only<11->{1} & \only<13->{1} \\
    \end{array}
  \]
\end{frame}

\begin{frame}
  \frametitle{Tautologie\"en}
  \structure{Definitie}
  \begin{center}
    Proposities die altijd waar zijn, ongeacht de waarden van de variabelen
  \end{center}
  \vskip4mm
  \structure{Voorbeelden}
  \[
  \begin{array}{r@{\;}c@{\;}l}
      p & \OR & \NOT p \\[2mm]
      p & \IMPLIES & p \\[2mm]
      \NOT\NOT p & \IFF & p \\[2mm]
      p \AND q & \IFF & q \AND p \\[2mm]
    \end{array}
  \]
\end{frame}

\begin{frame}
  \frametitle{Logische Equivalenties}
  \structure{Definitie}
  \begin{center}
    Tautologie van de vorm $P \IFF Q$
  \end{center}
  \vskip4mm
  \structure{Voorbeelden}
  \[
  \begin{array}{r@{\;}c@{\;}l}
      \NOT\NOT p & \IFF & p \\[2mm]
      p \AND q & \IFF & q \AND p \\[2mm]
      p \OR q & \IFF & q \OR p \\[2mm]
      p \OR (q \OR r) & \IFF & (p \OR q) \OR r \\[2mm]
    \end{array}
  \]
\end{frame}

\begin{frame}
  \frametitle{Logische Equivalenties}
  \Large
  \begin{overprint}
    \onslide<1-2>
    \[
      p \IMPLIES (\NOT q \IMPLIES p \OR q \OR r)
    \]

    \onslide<3-4>
    \[
      \NOT p \OR (\NOT q \IMPLIES p \OR q \OR r)
    \]

    \onslide<5-6>
    \[
      \NOT p \OR (q \OR p \OR q \OR r)
    \]

    \onslide<7-8>
    \[
      (\NOT p \OR p) \OR q \OR q \OR r
    \]

    \onslide<9-10>
    \[
      1 \OR q \OR q \OR r
    \]

    \onslide<11>
    \[
      1
    \]
  \end{overprint}
  \vskip2cm
  \begin{overprint}
    \onslide<2-3>
    Logische equivalentie:
    \[
      \begin{array}{rcl}
        (P \IMPLIES Q) & \IFF & (\NOT P \OR Q)
      \end{array}
    \]
    We kiezen
    \[
      \begin{array}{rcl}
        P & = & p \\
        Q & = & \NOT q \IMPLIES p \OR q \OR r \\
      \end{array}
    \]

    \onslide<4-5>
    Logische equivalentie:
    \[
      \begin{array}{rcl}
        (P \IMPLIES Q) & \IFF & (\NOT P \OR Q)
      \end{array}
    \]
    We kiezen
    \[
      \begin{array}{rcl}
        P & = & \NOT q \\
        Q & = & p \OR q \OR r \\
      \end{array}
    \]

    \onslide<6-7>
    Logische equivalenties:
    \[
      \begin{array}{rcl}
        (P \OR (Q \OR R) & \IFF & ((P \OR Q) \OR R) \\
        P \OR Q & \IFF & Q \OR P
      \end{array}
    \]

    \onslide<8-9>
    Logische equivalentie:
    \[
      \begin{array}{rcl}
        (P \OR \NOT P) & \IFF & 1
      \end{array}
    \]

    \onslide<10-11>
    Logische equivalentie:
    \[
      \begin{array}{rcl}
        (1 \OR P) & \IFF & 1
      \end{array}
    \]
  \end{overprint}
\end{frame}

\begin{frame}
  \frametitle{Oefeningen}
  \begin{itemize}
    \item 3.1 (1, 2, 3)
    \item 3.2
    \item 3.15 (6, 10, 11, 12)
    \item 3.6
    \item Rest: thuis
  \end{itemize}
\end{frame}


\end{document}



%%% Local Variables: 
%%% mode: latex
%%% TeX-master: t
%%% End: 
