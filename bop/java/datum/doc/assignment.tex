\documentclass[a4paper]{article}
\usepackage{listings}

\lstdefinelanguage{MyJava}[]{java}{
  escapechar=@,%
  basicstyle={\ttfamily},%
  keywordstyle={\ttfamily\bfseries},%
  commentstyle={\it},%
}

\newcommand{\codefigure}[5][]{\begin{figure}[#5]\lstinputlisting[language=MyJava,caption=#3,label=#4,#1]{#2}\end{figure}}


\begin{document}
\section{Opgave}
\begin{itemize}
  \item Maak een nieuw project aan, genaamd {\tt Datum}.
  \item Importeer het bestand {\tt DateTests.java} door het vanuit Explorer te droppen op de {\tt src} folder in Eclipse.
  \item Open dit bestand.
  \item Plaats je muis over de fout op de eerste lijn (niet klikken!), wacht tot het menu verschijnt en kies ``Fix Project Setup''. Kies de optie om JUnit 4 aan de build path toe te voegen.
  \item Maak een nieuwe klasse {\tt Date}.
  \item Schrijf velden en methodes om de tests te doen slagen. Bekijk hiervoor de tests, lees de foutmelding en tracht af te leiden waar je klasse {\tt Date} aan moet voldoen.
  \item Soms krijg je te maken met foutieve invoer (bv. 13de maand). Gebruik voor deze gevallen {\tt throw new RuntimeException();}. Bijvoorbeeld:
\begin{verbatim}
if ( month < 1 || month > 12 ) {
    throw new RuntimeException();
}
\end{verbatim}
\end{itemize}


\end{document}