\documentclass[a4paper]{article}
\usepackage{a4wide}

\pagestyle{empty}

\begin{document}
\begin{center}
  \Huge Opgave Breuk
\end{center}
Schrijf een klasse {\tt Breuk} met de volgende functionaliteit:
\begin{itemize}
  \item Een breuk bestaat uit een teller en een noemer.
  \item Een breuk is onwijzigbaar: eenmaal aangemaakt kan je de teller noch de noemer wijzigen.
        Elke operatie op breuken geeft een nieuw object terug.
  \item Het doel van de opgave is om alle tests te doen slagen.
  \item Ter herinnering, de rekenregels:
        \[
          \begin{array}{lrcl}
            \textrm{optelling} & \displaystyle \frac ab + \frac cd & = & \displaystyle \frac{a \cdot d+b \cdot c}{c \cdot d} \\ \\
            \textrm{vermenigvuldiging} & \displaystyle \frac ab \cdot \frac cd & = & \displaystyle \frac{a \cdot c}{b \cdot d} \\ \\
            \textrm{invertering} & \displaystyle \left(\frac ab\right)^{-1} & = & \displaystyle \frac{b}{a} \\ \\
            \textrm{deling} & \displaystyle \frac ab / \frac cd & = & \displaystyle \frac ab \cdot \frac dc \\ \\
          \end{array}
        \]
  \item Valideer telkens de invoer! Bij foute invoer, schrijf je
        \begin{center} \tt
          throw new RuntimeException();
        \end{center}
  \item Een breuk vereenvoudigen doe je als volgt:
        \[
          \frac ab = \frac{a/\mathrm{ggd}(a,b)}{b/\mathrm{ggd}(a,b)}
        \]
        waarbij $\mathrm{ggd}$ staat voor grootste gemene deler. Deze functie
        werd reeds voor jullie ge\"implementeerd in de hulpklasse {\tt Util}.
\end{itemize}


\end{document}