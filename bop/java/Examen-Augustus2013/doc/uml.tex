\documentclass{standalone}
\usepackage{khluml}

\usetikzlibrary{calc}

\newcommand{\Student}{
  \umlclass{Student}{
       \umlmember[access=private]{naam:String}
       \umlmember[access=private]{voornaam:String}
       \umlmember[access=private]{code:int}
     }{
       \umlmember[access=public]{Student(voornaam:String, naam:String, code:int)}
       \umlmember[access=public]{getNaam():String}
       \umlmember[access=public]{getVoornaam():String}
       \umlmember[access=public]{getCode():int}
       \umlmember[access=private]{setNaam(naam:String)}
       \umlmember[access=private]{setVoornaam(voornaam:String)}
       \umlmember[access=private]{setCode(code:int)}
     }
}

\newcommand{\OPO}{
  \umlclass{OPO}{
    \umlmember[access=private]{code:int}
    \umlmember[access=private]{naam:String}
    \umlmember[access=private]{studiePunten:int}
    \umlmember[access=private]{engels:boolean}
  }{
    \umlmember[access=public]{OPO(code:int, naam:String, sp:int, engels:boolean)}
    \umlmember[access=public]{OPO(code:int, naam:String, sp:int)}
    \umlmember[access=public]{getCode():int}
    \umlmember[access=public]{getNaam():String}
    \umlmember[access=public]{getStudiePunten():int}
    \umlmember[access=public]{isEngels():boolean}
    \umlmember[access=private]{setCode(code:Code)}
    \umlmember[access=private]{setNaam(naam:String)}
    \umlmember[access=private]{setStudiePunten(sp:int)}
    \umlmember[access=private]{setEngels(engels:boolean)}
  }
}

\newcommand{\Inschrijving}{
  \umlclass{Inschrijving}{
    \umlmember[access=private]{datum:Datum}
  }{
    \umlmember[access=public]{Inschrijving(opo:OPO, datum:Datum)}
    \umlmember[access=public]{getOPO():OPO}
    \umlmember[access=public]{getDatum():Datum}
    \umlmember[access=private]{setOPO(opo:OPO)}
    \umlmember[access=private]{setDatum(datum:Datum)}
  }
}

\newcommand{\Jaarprogramma}{
  \umlclass{Jaarprogramma}{
  }{
    \umlmember[access=public]{JaarProgramma(student:Student)}
    \umlmember[access=public]{getStudent():student}
    \umlmember[access=public]{voegInschrijvingToe(inschrijving:Inschrijving)}
    \umlmember[access=public]{voegInschrijvingToe(dag:int,maand:int,jaar:int,opo:OPO)}
    \umlmember[access=public]{geefTotaalAantalStudiePunten():int}
    \umlmember[access=public]{geefFactuur():String}
    \umlmember[access=private]{setStudent(student:Student)}
  }
}


\begin{document}

\begin{tikzpicture}
  \node[/uml/class] (student) {
    \Student
  };

  \node[/uml/class,anchor=west] (opo) at ($ (student.east) + (3,0) $) {
    \OPO
  };

  \node[/uml/class,anchor=north] (inschrijving) at ($ (opo.south) + (0,-2) $) {
    \Inschrijving
  };

  \path let \p1 = (student.south), \p2 = (inschrijving.west) in
        node[/uml/class] (jaarprogramma) at (\x1, \y2) {
    \Jaarprogramma
  };
  
  \draw[/uml/association] (inschrijving.north) -- (opo.south)
                          node [at end,left,anchor=north east] {1}
                          node [at end,right,anchor=north west] {-opo};
  \draw[/uml/association] (jaarprogramma.north) -- (student.south)
                          node [at end,left,anchor=north east] {1}
                          node [at end,right,anchor=north west] {-student};
  \draw[/uml/association] (jaarprogramma.east) -- (inschrijving.west)
                          node [at end,left,anchor=north east] {$0\dots *$}
                          node [at end,right,anchor=south east] {-inschrijvingen};
\end{tikzpicture}

\end{document}
