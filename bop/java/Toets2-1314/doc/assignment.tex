%\documentclass[a4paper]{khltoets}
\documentclass[a4paper,solution]{khltoets}


\begin{document}

\HEADER{\large Beginselen van Objectgericht Programmeren}{Test 1}{6}

\BEGINASSIGNMENT
\textbf{Belangrijk}: deze opgave bevat uit meerdere delen.
Lees per blad eerst de \emph{volledige} opgave voordat je begint aan je ant\-woord.
Soms wordt immers slechts een vereenvoudigde implementatie gevraagd en staan  de details pas vermeld op het einde.
De enige toegelaten container is de array, je mag dus
geen {\tt ArrayList}s gebruiken noch andere containerklassen.
\vskip4mm
Een persoon heeft de volgende kenmerken:
\begin{itemize}
  \item Een persoon heeft een voornaam.
  \item Een persoon heeft een familienaam.
\end{itemize}
Schrijf een klasse {\tt Person} die de volgende functionaliteit aanbiedt:
\begin{itemize}
  \item We moeten {\tt Person}-objecten kunnen aanmaken.
        Bij aanmaak van een {\tt Person}-object moeten zowel voor- als familienaam
        gegeven worden. Indien \'e\'en van beide (of beide) {\tt null} is,
        moet er een fout gemeld worden: dit gebeurt met {\tt abort();}.
  \item Schrijf de nodige getters en setters om de voor- en familienaam
        op te vragen en te wijzigen. Bij wijzigen moet er gecheckt worden
        dat de nieuwe gegevens geldig zijn (d.i.\ niet {\tt null}).
\end{itemize}
\ENDASSIGNMENT

\BEGINSOLUTION
\INLINECODE[basicstyle=\small]{Person.java}
\ENDSOLUTION


\BEGINASSIGNMENT
Een film heeft de volgende kenmerken:
\begin{itemize}
  \item Een film heeft een titel.
  \item Een film heeft een regisseur.
  \item In een film spelen acteurs mee.
  \item Een film kan een sequel zijn op een andere film.
  \item Een film is gelijkaardig aan andere films.
\end{itemize}
Schrijf de klasse {\tt Movie} die de volgende functionaliteit aanbiedt:
\begin{itemize}
  \item Een eerste constructor heeft als parameters de titel en de regisseur. Voor de andere
        waarden worden dan redelijke standaardwaarden ingevuld.
  \item Een tweede constructor heeft als parameters de titel, de regisseur en de voorloper van de film.
        Voor de andere waarden worden dan redelijke standaardwaarden ingevuld.
  \item Een derde constructor heeft per veld een overeenkomstige parameter.
  \item Schrijf een methode {\tt addSimilarMovie} die een {\tt Movie} object verwacht:
        \begin{center} \tt public void addSimilarMovie(Movie movie) \{ ... \} \end{center}
        Een methodeoproep {\tt m1.addSimilarMovie(m2)} doet twee dingen:
        \begin{itemize}
          \item {\tt m2} wordt toegevoegd aan de lijst van gelijkaardige films van {\tt m1};
          \item {\tt m1} wordt toegevoegd aan de lijst van gelijkaardige films van {\tt m2}.
        \end{itemize}
        Met andere woorden, indien een film X gelijkaardig is aan een film Y, moet
        film Y ook gelijkaardig zijn aan film X.
  \item Schrijf de methodeheader (dus niet het stuk tussen accolades) op voor een {\tt addActor} methode die een acteur
        toevoegt aan de lijst acteurs.
\end{itemize}
Voor deze klasse hoef je de argumenten niet te checken op geldigheid. Vergeet niet dat je gebruik kunt maken van de klasse
{\tt Person} uit het vorige deel van deze vraag.
\ENDASSIGNMENT

\BEGINSOLUTION
\INLINECODE[basicstyle=\small]{Movie.java}
\ENDSOLUTION


\clearpage

\BEGINASSIGNMENT
Je krijgt de volgende extra code:
\INLINECODE[basicstyle=\small]{code.java}
Teken een klassendiagramma voor {\tt Person}, {\tt Movie}, {\tt Media} en {\tt MovieCollection}.
\ENDASSIGNMENT


\BEGINASSIGNMENT
Sergio Leone is mijn favoriete regisseur. Hij maakte o.a.\ de Dollars Trilogy, bestaande uit de films
``A Fistful of Dollars'', ``For a Few Dollars More'' en ``The Good, the Bad and the Ugly''.
Clint Eastwood is de hoofdrolspeler in de drie films. Lee Van Cleef speelt
in de tweede en derde film, terwijl Eli Wallach enkel in de laatste een rol heeft.
De drie films heb ik op DVD. The Good, the Bad and the Ugly heb ik ook op BluRay.

Schrijf de code op die de nodige klassen instantieert en methodes oproept om deze
informatie te modelleren in Java. Maak ook een UML objectdiagramma.
\ENDASSIGNMENT

\BEGINSOLUTION
\INLINECODE[basicstyle=\small]{modelling.java}
\ENDSOLUTION


\end{document}
