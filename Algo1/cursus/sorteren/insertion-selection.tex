\section{Selection Sort en Insertion Sort}
Selection Sort en Insertion Sort bespreken we samen doordat ze dezelfde basisstructuur
vertonen maar een licht verschillende uitwerking hebben.

\subsection{Werking}
Zoals elk sorteeralgoritme verwachten Selection Sort en Insertion Sort een rij getallen waarvan we willen
dat ze gesorteerd worden. Laten we deze rij \inlinecode{input} noemen.
We beschouwen ook een lege rij, \inlinecode{output} genaamd.
We gaan nu \'e\'en na \'e\'en de elementen
van \inlinecode{input} naar \inlinecode{output} verhuizen tot \inlinecode{input} leeg is.
We willen dan dat \inlinecode{output} alle getallen in gesorteerde volgorde bevat.

\begin{center}
  \begin{tabular}{r|c|c|c|c|c|c}
    & begin & & & & & einde \\
    \hline
    {\tt input} & $\square\square\square\square\square$ &
                  $\square\square\square\square$ &
                  $\square\square\square$ &
                  $\square\square$ &
                  $\square$ \\
    {\tt output} & &
                   $\square$ &
                   $\square\square$ &
                   $\square\square\square$ &
                   $\square\square\square\square$ &
                   $\square\square\square\square\square$                   
  \end{tabular}
\end{center}

De vraag is nu uiteraard hoe dit stapsgewijs elementen verhuizen
van \inlinecode{input} naar \inlinecode{output} juist moet gebeuren.
Een enkele verhuisoperatie bestaat uit twee stappen:
\begin{itemize}
  \item We moeten eerst een element uit \inlinecode{input} selecteren dat we gaan verhuizen.
  \item We moeten dit element correct plaatsen in \inlinecode{output}.
\end{itemize}
Nu zijn er twee mogelijkheden:
\begin{itemize}
  \item We houden de eerste stap (het selecteren) simpel, maar dan wordt de tweede stap (het plaatsen) moeilijker.
  \item We kunnen ervoor zorgen dat het plaatsen eenvoudig gaat, maar dan is het selecteren ingewikkelder. 
\end{itemize}
Beide mogelijkheden stemmen overeen met een sorteeralgoritme:
Insertion Sort komt overeen met de eerste optie, Selection Sort met de tweede optie.
We bespreken nu concreet hoe beiden te werk gaat door middel van een voorbeeld.

\begin{example}
We willen de rij $[5, 1, 2, 4, 3]$ sorteren. We beginnen dus met
\inlinecode{input = [5, 1, 2, 4, 3]} en \inlinecode{output = []}.
We moeten nu \'e\'en na \'e\'en elk element uit \inlinecode{input}
naar \inlinecode{output} verhuizen.
\begin{itemize}
  \item Insertion Sort neemt een willekeurig element uit \inlinecode{input}
        (voor het gemak is dit gewoon telkens het eerste) en gaat
        dan \inlinecode{output} af op zoek naar de juiste positie voor dit element.
  \item Selection Sort gaat voor elke verhuisstap zoeken we naar het kleinste element uit \inlinecode{input} en
        voegt dit toe aan het einde van \inlinecode{output}.
\end{itemize}
\begin{center}
  \newcommand{\HL}[1]{{\bfseries #1}}
  \begin{tabular}{l|l|l}
    & \multicolumn{1}{c}{\bf Insertion Sort} & \multicolumn{1}{c}{\bf Selection Sort} \\
     \hline\hline
       {\tt input} & {\tt [\HL5, 1, 2, 4, 3]} & {\tt [5, \HL1, 2, 4, 3]} \\
       {\tt output} & {\tt []} & {\tt []}
\\ \hline
      {\tt input} & {\tt [\HL1, 2, 4, 3]} & {\tt [5, \HL2, 4, 3]} \\
      {\tt output} & {\tt [5]} & {\tt [1]}
\\ \hline
      {\tt input} & {\tt [\HL2, 4, 3]} & {\tt [5, 4, \HL3]} \\
      {\tt output} & {\tt [1, 5]} & {\tt [1, 2]}
\\ \hline
      {\tt input} & {\tt [\HL4, 3]}& {\tt [5, \HL4]} \\
      {\tt output} & {\tt [1, 2, 5]} & {\tt [1, 2, 3]}
\\ \hline
      {\tt input} & {\tt [\HL3]}& {\tt [\HL5]} \\
      {\tt output} & {\tt [1, 2, 4, 5]} & {\tt [1, 2, 3, 4]}
\\ \hline
      {\tt input} & {\tt []} & {\tt []} \\
      {\tt output} & {\tt [1, 2, 3, 4, 5]} & {\tt [1, 2, 3, 4, 5]}
  \end{tabular}
\end{center}
Het uitgekozen element uit \inlinecode{input} staat telkens in het vet. Bij Insertion Sort
is dit het eerste element uit de rij, bij Selection Sort is dit het kleinste.
\end{example}

We kunnen de werking van beide algoritmes als volgt schematisch voorstellen:
\begin{center}
  \begin{tabular}{r||c|c}
    & {\bf Insertion Sort} & {\bf Selection Sort} \\
    \hline \hline
    {\bfseries Selectie uit {\tt\bfseries input}} & Eerste element (simpel) & Minimum element (complex) \\
    \hline
    {\bfseries Plaatsing in {\tt\bfseries output}} & Invoegen op juiste positie (complex) & Op het einde (simpel) \\
  \end{tabular}
\end{center}

We willen nu een \emph{in-place} versie ontwikkelen van beide algoritmen. Hiermee bedoelen we
dat al het sorteerwerk moet gebeuren in de oorspronkelijke gegeven rij zelf; met andere woorden,
we willen enkel gebruik maken van de rij \inlinecode{input} zonder een aparte rij
\inlinecode{output} nodig te hebben.
Dit kan door conceptueel \inlinecode{input} en \inlinecode{output} te herbergen in
\inlinecode{input} zelf. We kunnen dit als volgt visualiseren:

\begin{center}
  \newcommand{\IN}[1]{{\color{red}#1}}
  \newcommand{\OUT}[1]{{\color{green}#1}}
  \begin{tabular}{l|c}
    \multicolumn{1}{c}{{\bf oorspronkelijk}} & {\bf in-place} \\
    \hline\hline
    \begin{tabular}{rl}
      {\tt input} & \IN{$\square\square\square\square\square$} \\
      {\tt output} & \OUT{$$}
    \end{tabular}
    &
    \begin{tabular}{rl}
      {\tt input} & \IN{$\square\square\square\square\square$}\OUT{$$}
    \end{tabular} \\
\hline
    \begin{tabular}{rl}
      {\tt input} & \IN{$\square\square\square\square$} \\
      {\tt output} & \OUT{$\square$}
    \end{tabular}
    &
    \begin{tabular}{rl}
      {\tt input} & \OUT{$\square$}\IN{$\square\square\square\square$}
    \end{tabular} \\
\hline
    \begin{tabular}{rl}
      {\tt input} & \IN{$\square\square\square$} \\
      {\tt output} & \OUT{$\square\square$}
    \end{tabular}
    &
    \begin{tabular}{rl}
      {\tt input} & \OUT{$\square\square$}\IN{$\square\square\square$}
    \end{tabular} \\
\hline
    \begin{tabular}{rl}
      {\tt input} & \IN{$\square\square$} \\
      {\tt output} & \OUT{$\square\square\square$}
    \end{tabular}
    &
    \begin{tabular}{rl}
      {\tt input} & \OUT{$\square\square\square$}\IN{$\square\square$}
    \end{tabular} \\
\hline
    \begin{tabular}{rl}
      {\tt input} & \IN{$\square$} \\
      {\tt output} & \OUT{$\square\square\square\square$}
    \end{tabular}
    &
    \begin{tabular}{rl}
      {\tt input} & \OUT{$\square\square\square\square$}\IN{$\square$}
    \end{tabular} \\
\hline
    \begin{tabular}{rl}
      {\tt input} & \IN{$$} \\
      {\tt output} & \OUT{$\square\square\square\square\square$}
    \end{tabular}
    &
    \begin{tabular}{rl}
      {\tt input} & \OUT{$\square\square\square\square\square$}\IN{$$}
    \end{tabular} \\
  \end{tabular}
\end{center}

De rij \inlinecode{input} moet dus denkbeeldig opgesplitst worden
in een linkse en rechtse deelrij (aangegeven door groen en rood).
De linkerdeelrij moet te allen tijde gesorteerd zijn, de rechterdeelrij
stelt ``het overschot'' over dat nog moet gesorteerd worden.
We lichten dit toe in de volgende paragrafen.

\subsection{Insertion Sort} \label{sec:sorteren:insertion-sort}
We illustreren de werking van Insertion Sort in detail door middel van een uitgebreid voorbeeld.

\begin{example} \label{example:sorteer:insertion}
\newcommand{\IN}[1]{{\color{red}#1}}
\newcommand{\OUT}[1]{{\color{green}#1}}
\newcommand{\swap}{\stackrel{\curvearrowleftright}{,}}
We beschouwen de rij $[6,4,2,1,3,9]$. Zoals eerder uitgelegd splitsen we deze
rij in twee delen: het gesorteerde linkerstuk en het ongesorteerde rechterstuk.
Vermits we nog niets hebben gedaan, is het linkerstuk leeg, en het rechterstuk
komt overeen met de ganse rij.
\[
  [ \IN6, \IN4, \IN2, \IN1, \IN3, \IN9 ]
\]
Insertion Sort neemt altijd het eerste element van het ongesorteerde stuk
(in ons voorbeeld is dit $6$) en voegt dit in de ongesorteerde deelrij.
Vermits deze laatste leeg is, is dit triviaal:
\[
  [ \OUT6, \IN4, \IN2, \IN1, \IN3, \IN9 ]
\]
Vervolgens is het element 4 aan de beurt. We moeten dit op de juiste plaats
invoegen in de rij [\OUT6]. We zien dat 4 links van rechts hoort te staan.
We bereiken dit door 6 en 4 van plaats te verwisselen.
\[
  [ \OUT6 \swap \IN4, \IN2, \IN1, \IN3, \IN9 ] \rightarrow [ \OUT4, \OUT6, \IN2, \IN1, \IN3, \IN9 ]
\]
Nu pakken we 2 aan. Het is duidelijk dat 2 helemaal links hoort te staan.
Dit betekent dat we 4 en 6 een positie naar rechts moeten verschuiven,
terwijl 2 naar links moet verplaatst worden. We doen dit stapsgewijs door middel van een reeks verwisselingen.
\[
  [ \OUT4, \OUT6 \swap \IN2, \IN1, \IN3, \IN9 ]
  \rightarrow
  [ \OUT4 \swap \IN2, \OUT6, \IN1, \IN3, \IN9 ]
  \rightarrow
  [ \OUT2, \OUT4, \OUT6, \IN1, \IN3, \IN9 ]
\]
Nu is het de beurt aan 1. We moeten dit getal weer helemaal naar links verschuiven,
terwijl 2, 4 en 6 een stap naar rechts moeten verplaatst worden.
\[
  [ \OUT2, \OUT4, \OUT6 \swap \IN1, \IN3, \IN9 ]
  \rightarrow
  [ \OUT2, \OUT4 \swap \IN1, \OUT6, \IN3, \IN9 ]
  \rightarrow
  [ \OUT2 \swap \IN1, \OUT4, \OUT6, \IN3, \IN9 ]
  \rightarrow
  [ \OUT1, \OUT2, \OUT4, \OUT6, \IN3, \IN9 ]
\]
Het element 3 moet slechts 2 stappen naar links.
\[
  [ \OUT1, \OUT2, \OUT4, \OUT6 \swap \IN3, \IN9 ]
  \rightarrow
  [ \OUT1, \OUT2, \OUT4 \swap \IN3, \OUT6, \IN9 ]
  \rightarrow
  [ \OUT1, \OUT2, \OUT3, \OUT4, \OUT6, \IN9 ]
\]
We zien dat 9 reeds juist staat, we zijn dus klaar met sorteren.
\end{example}

\codefigure[numbers=left]{sorteren/insertion.js}{Insertion Sort}{code:sorting:insertion-sort}

Een implementatie van dit algoritme vind je in \cref{code:sorting:insertion-sort}.
\begin{itemize}
  \item De functie \inlinecode{voegIn} gedefinieerd op
        \crefrange{line:sorteer:insertion:voegin}{line:sorteer:insertion:voegin-end}
        voegt het element met index \inlinecode{i} in in de linkerdeelrij die
        gaat van \inlinecode{rij[0]} tot \inlinecode{rij[i-1]}. Visueel komt dit overeen met
        {
          \newcommand{\IN}[1]{{\color{red}#1}}
          \newcommand{\OUT}[1]{{\color{green}#1}}
          \[
            \begin{array}{cccccccc}
              \OUT\square & \OUT\square & \OUT\square & \OUT\cdots & \OUT\square &
                \IN\square & \IN\cdots &  \IN\square \\
              \uparrow & \uparrow & \uparrow && \uparrow & \uparrow && \uparrow\\
              {\tt 0} & {\tt 1} & {\tt 2} && {\tt i-1} & {\tt i} && \makebox[5pt][l]{\tt rij.length-1} \\
            \end{array}
          \]
        }
  \item De \inlinecode{while}-lus op
        \crefrange{line:sorteer:insertion:voegin:while}{line:sorteer:insertion:voegin:while-end}
        verschuift het toe te voegen element stapsgewijs door middel van een reeks verwisselingen.
  \item De lusconditie op \cref{line:sorteer:insertion:voegin:while} bestaat uit twee delen:
        \begin{itemize}
          \item \inlinecode{i-1 >= 0} zorgt ervoor dat we het element niet verschuiven tot
                voorbij het begin van de rij. De \inlinecode{-1} is nodig omdat
                we telkens de elementen met indices \inlinecode{i-1} en \inlinecode{i} verschuiven
                (\cref{line:sorteer:insertion:voegin:swap});
                indien we zouden toelaten dat \inlinecode{i == 0}, dan zouden
                we de elementen met indices \inlinecode{-1} en \inlinecode{0} willen verwisselen,
                wat uiteraard niet gaat. We moeten dus voorkomen dat \inlinecode{i} tot \inlinecode{0} gaat.
          \item \inlinecode{rij[i-1] > rij[i]} checkt of de elementen wel verwisseld moeten worden.
                Indien \inlinecode{rij[i-1] <= rij[i]}, heeft het in te voegen elementen
                zijn juiste plaats in de rij bereikt en moet het stapsgewijs verschuiven stopgezet worden.
        \end{itemize}
  \item De functie \inlinecode{insertionSort}
        (\crefrange{line:sorteer:insertion:insertion-sort}{line:sorteer:insertion:insertion-sort-end})
        voert het eigenlijke sorteren uit.
  \item De \inlinecode{for}-lus op \crefrange{line:sorteer:insertion:for}{line:sorteer:insertion:for-end}
        gaat de rij van links naar rechts af. Elk element wordt \'e\'en na \'e\'en ingevoegd
        door middel van een oproep naar \inlinecode{voegIn} op \cref{line:sorteer:insertion:voegin-inv}.
\end{itemize}

\begin{example}
We geven een schematisch overzicht van de verschillende stappen dat de
lus op \crefrange{line:sorteer:insertion:for}{line:sorteer:insertion:for-end}
in \cref{code:sorting:insertion-sort} doorloopt. We gebruiken de rij $[6,3,7,5,1,2,4,8,9]$ als voorbeeld.
Deze tabel geeft weer wat we zouden bekomen indien we een \inlinecode{alert} zouden toevoegen
juist na \cref{line:sorteer:insertion:voegin-inv} die de huidige waarden van \inlinecode{i}
en \inlinecode{rij} zou afdrukken.
\begin{center}
  \newcommand{\IN}[1]{{\color{red}#1}}
  \newcommand{\OUT}[1]{{\color{green}#1}}
  \begin{tabular}{cc}
    {\tt\bfseries i} & {\tt\bfseries rij} \\
    \hline
    {\it voor de lus} & $[\OUT6,\IN3,\IN7,\IN5,\IN1,\IN2,\IN4,\IN8,\IN9]$ \\ \hline
    {\tt 1} & $[\OUT3,\OUT6,\IN7,\IN5,\IN1,\IN2,\IN4,\IN8,\IN9]$ \\ \hline
    {\tt 2} & $[\OUT3,\OUT6,\OUT7,\IN5,\IN1,\IN2,\IN4,\IN8,\IN9]$ \\ \hline
    {\tt 3} & $[\OUT3,\OUT5,\OUT6,\OUT7,\IN1,\IN2,\IN4,\IN8,\IN9]$ \\ \hline
    {\tt 4} & $[\OUT1,\OUT3,\OUT5,\OUT6,\OUT7,\IN2,\IN4,\IN8,\IN9]$ \\ \hline
    {\tt 5} & $[\OUT1,\OUT2,\OUT3,\OUT5,\OUT6,\OUT7,\IN4,\IN8,\IN9]$ \\ \hline
    {\tt 6} & $[\OUT1,\OUT2,\OUT3,\OUT4,\OUT5,\OUT6,\OUT7,\IN8,\IN9]$ \\ \hline
    {\tt 7} & $[\OUT1,\OUT2,\OUT3,\OUT4,\OUT5,\OUT6,\OUT7,\OUT8,\IN9]$ \\ \hline
    {\tt 8} & $[\OUT1,\OUT2,\OUT3,\OUT4,\OUT5,\OUT6,\OUT7,\OUT8,\OUT9]$
  \end{tabular}
\end{center}
\end{example}

\begin{exercise}
Op \cref{line:sorteer:insertion:for} van \cref{code:sorting:insertion-sort} staat als
initialisering van de iteratievariabele \inlinecode{i=1}. Wat gebeurt er als
we \inlinecode{i=0} zouden schrijven?
\begin{solution}
Het resultaat blijft hetzelfde. Het geval waarbij \inlinecode{i} gelijk is aan
\inlinecode{0} komt overeen met het invoegen van een element in een lege rij.
Vermits hiervoor in ons geval niets hoeft te gebeuren, kunnen we deze stap gewoon overslaan.
In \cref{example:sorteer:insertion} kwamen we dit ook tegen: het invoegen
van 6 benodigde geen verwisselingen.
\end{solution}
\end{exercise}

\begin{exercise}
De \inlinecode{while}-conditie op \cref{line:sorteer:insertion:voegin:while} van
\cref{code:sorting:insertion-sort} is een conjunctie\footnote{Conjunctie: een EN-operatie,
geschreven \inlinecode{&&} in JavaScript. Een OF-operatie (\inlinecode{||}) heet een disjunctie.}
van twee voorwaarden. Kunnen we deze twee voorwaarden van plaats verwisselen;
met andere woorden, kunnen we \inlinecode{rij[i-1] > rij[i] && i-1 >= 0} schrijven?
\begin{solution}
Dit doe je best niet vermits je het risico loopt om, indien \inlinecode{i == 0},
\inlinecode{rij[-1]} met \inlinecode{rij[0]} te vergelijken. 
Een conjunctie (\inlinecode{&&}) wordt van links naar rechts uitgevoerd:
indien de linkervoorwaarde faalt, zal de rechtervoorwaarde niet ge\"evalueerd worden.
Eerst nakijken of \inlinecode{i-1 >= 0} verhindert dus dat \inlinecode{rij[i-1] > rij[i]}
wordt uitgevoerd indien \inlinecode{i} niet groter is dan \inlinecode{0}.

Terzijde: hoewel het gedrag
van zulke code wel gedefinieerd is in JavaScript (\inlinecode{rij[-1]} zal de waarde \inlinecode{undefined}
opleveren) en ironisch genoeg de code uiteindelijk zelfs zal werken (je kan zelfs
\inlinecode{i-1 >= 0} weglaten!), moet je JavaScript al h\'e\'el goed kennen
om op zulk gedrag te kunnen steunen teneinde correct werkende algoritmen te schrijven.
De bedoeling van deze cursus om de algoritmen te begrijpen en te kunnen implementeren,
en niet de gekheden en frivoliteiten van JavaScript te leren benutten;
indien je code steunt op zulke taalspecifieke details, zal het dan ook als fout beschouwd worden.
We willen dus dat je code weldegelijk \emph{eerst} checkt dat \inlinecode{i-1 >= 0}.
\end{solution}
\end{exercise}

\subsection{Selection Sort} \label{sec:sorteren:selection-sort}
Zoals met Insertion Sort beginnen we met een uitgebreid voorbeeld.

\begin{example} \label{example:sorteer:selection}
\newcommand{\IN}[1]{{\color{red}#1}}
\newcommand{\OUT}[1]{{\color{green}#1}}
\newcommand{\MIN}[1]{\stackrel{\downarrow}{\IN{\mathbf{#1}}}}
\newcommand{\HERE}[1]{\stackrel{\downarrow}{#1}}
\newcommand{\SWAP}[2]{%
\begin{tikzpicture}[overlay, remember picture]
  \draw[<->] ($ (#1.north) + (1pt,0pt) $) --
                ++(1pt,5pt)
                -- ($ (#2.north) + (-2pt,5pt) $)
                -- ($ (#2.north) + (-1pt, 0) $);
\end{tikzpicture}}
\newcommand{\XIN}[2]{{\tikz[baseline, remember picture]{\node[anchor=base] (#2) { \ensuremath{\IN{#1}} };}}}
\newcommand{\XOUT}[2]{{\tikz[baseline, remember picture]{\node[anchor=base] (#2) { \ensuremath{\OUT{#1}} };}}}

We beschouwen dezelfde rij $[6,4,2,1,3,9]$ als in \cref{example:sorteer:insertion}.
We splitsen de rij weeral in twee: een gesorteerd linkerdeel en een ongesorteerd rechterdeel.
Het linkerdeel is op dit moment nog leeg, en het rechterdeel omvat de volledige rij.
Selection Sort bestaat erin om de linkerdeelrij op te bouwen door telkens
het minimum van de rechterdeelrij eraan toe te voegen. We zoeken dus naar het minimum:
dit blijkt \inlinecode{1} te zijn.
We verplaatsen dit minimumelement naar voren door het te verwisselen
met het element dat momenteel vooraan in de rij staat:
\[ \newcommand{\XI}[1]{\XIN{#1}{elta#1}} \newcommand{\XO}[1]{\XOUT{#1}{elta#1}}
  [ \XI{6}, \XI{4}, \XI{2}, \XI{1}, \XI{3}, \XI{9} ]
\]
\SWAP{elta6}{elta1}
We herhalen dit proc\'ed\'e: we verwisselen het rode minimum telkens met het eerste rode element.
\[
  \newcommand{\XI}[2]{\XIN{#1}{elt#2#1}}\newcommand{\XO}[2]{\XOUT{#1}{elt#2#1}}
  \begin{array}{rc}
    & [ \XO1b, \XI4b, \XI2b, \XI6b, \XI3b, \XI9b  ] \\[2mm]
    \rightarrow &
    [ \XO1c, \XO2c, \XI4c, \XI6c, \XI3c, \XI9c ] \\[2mm]
    \rightarrow &
    [ \XO1d, \XO2d, \XO3d, \XI6d, \XI4d, \XI9d ] \\[2mm]
    \rightarrow &
    [ \XO1e, \XO2e, \XO3e, \XO4e, \XI6e, \XI9e ] \\[2mm]
    \rightarrow &
    [ \XO1f, \XO2f, \XO3f, \XO4f, \XO6f, \XO9f ]
  \end{array}
\]
\SWAP{eltb4}{eltb2}\SWAP{eltc4}{eltc3}\SWAP{eltd6}{eltd4}\SWAP{elte6}{elte6}
We zien dat het kan voorkomen dat we een element met zichzelf verwisselen,
namelijk indien het eerste rode element reeds het minimum is. Zoals
je kan zien kan dit geen kwaad en hoeven we in onze code geen rekening te
houden met dit randgeval.
\end{example}

\codefigure[numbers=left]{sorteren/selection.js}{Selection Sort}{code:sorting:selection-sort}

De code voor Selection Sort vind je in \cref{code:sorting:selection-sort}.
\begin{itemize}
  \item We maken gebruik van een hulpfunctie \inlinecode{zoekIndexMinimum}
        (\crefrange{line:sorteer:selection:zoekMin}{line:sorteer:selection:zoekMin-end}).
        Deze zoekt de index van het minimaal element uit \inlinecode{rij} dat zich bevindt
        tussen indices \inlinecode{i} en \inlinecode{imax} (zie ook \cref{example:sorteer:selection:min}
        op \cpageref{example:sorteer:selection:min}).
  \item De functie \inlinecode{zoekIndexMinimum} gaat de rij af
        van index \inlinecode{i} tot \inlinecode{imax}
        door middel van de \inlinecode{while}-lus op
        \crefrange{line:sorteer:selection:zoekMin:while}{line:sorteer:selection:zoekMin:while-end}.
        De lokale variabele \inlinecode{resultaat} houdt
        de index van het totnogtoe kleinste element bij.
        Telkens er een kleiner element wordt gevonden,
        wordt \inlinecode{resultaat} aangepast (\cref{line:sorteer:selection:zoekMin:set-resultaat}).
  \item De functie \inlinecode{selectionSort} gedefinieerd op lijnen
        \crefrange{line:sorteer:selection:sort}{line:sorteer:selection:sort-end}
        voert het eigenlijke sorteren uit.
        De variabele \inlinecode{i} stelt de denkbeeldige grens voor
        tussen de gesorteerde linkerdeelrij (gaande van \inlinecode{0} tot \inlinecode{i-1})
        en de ongesorteerde rechterdeelrij (\inlinecode{i} tot \inlinecode{rij.length-1}).
        Bij elke iteratie van de \inlinecode{for}-lus wordt de index van
        het kleinste element uit de rechterdeelrij gezocht
        met een oproep naar \inlinecode{zoekMinimumIndex}. Merk op hoe
        de grenzen van de rechterdeelrij \inlinecode{i} tot \inlinecode{rij.length-1} worden doorgegeven
        als argumenten. Op \cref{line:sorteer:selection:swap} wordt het kleinste element
        verwisseld met het eerste element uit de rechterdeelrij.
  \item Nota: hoewel \inlinecode{zoekIndexMinimum} de variabele \inlinecode{i}
        wijzigt (\cref{line:sorteer:selection:zoekMin:i-inc,line:sorteer:selection:zoekMin:i-inc2}),
        heeft dit geen invloed op de waarde van \inlinecode{i} binnen \inlinecode{selectionSort}.
        \inlinecode{zoekIndexMinimum} werkt immers met een \emph{kopie} van \inlinecode{i}.
\end{itemize}

\begin{example} \label{example:sorteer:selection:min}
We illustreren de werking van \inlinecode{zoekIndexMinimum}. Zoals eerder
uitgelegd, zal een oproep \inlinecode{zoekIndexMinimum(rij, i, imax)} als resultaat
opleveren de index van het kleine element waarvan de index ligt tussen \inlinecode{i}
en \inlinecode{imax}. De volgende tabel geeft een aantal voorbeelden die inwerken
op de volgende rij (voor de duidelijkheid staan de indices in subscript):
\[
  \newcommand{\IDX}[2]{#1_{\scriptscriptstyle(#2)}}
  [\IDX70, \IDX31, \IDX12, \IDX63, \IDX44, \IDX25, \IDX96]
\]
\begin{center}
  \newcommand{\NHL}[1]{{\color{red} #1}}
  \newcommand{\HL}[1]{{\color{green} #1}}
  \newcommand{\MIN}[1]{\framebox{#1}}
  \begin{tabular}{cccc}
    {\tt i} & {\tt imax} & resultaat & visueel voorgesteld \\ \hline
    0 & 6 & 2 & $[\HL{7,3,\MIN1,6,4,2,9}]$ \\
    0 & 3 & 2 & $[\NHL{\HL{7,3,\MIN1,6},4,2,9}]$ \\
    0 & 2 & 2 & $[\NHL{\HL{7,3,\MIN1},6,4,2,9}]$ \\
    0 & 1 & 1 & $[\NHL{\HL{7,\MIN3},1,6,4,2,9}]$ \\
    3 & 6 & 5 & $[\NHL{7,3,1,\HL{6,4,\MIN2,9}}]$ \\
    6 & 6 & 6 & $[\NHL{7,3,1,6,4,2,\HL{\MIN9}}]$ \\
  \end{tabular}
\end{center}
In de laatste kolom staat de deelrij \inlinecode{i}\dots\inlinecode{imax} in het groen aangegeven
en staat het minimumelement omkaderd.
\end{example}

\begin{exercise}
De implementatie van \inlinecode{zoekIndexMinimum}
in \cref{code:sorting:selection-sort} verwacht dat \inlinecode{i <= imax}.
Wat gebeurt er als men \inlinecode{zoekIndexMinimum} oproept met \inlinecode{i > imax}?
\begin{solution}
Het algoritme geeft in dit geval \inlinecode{i} terug, wat in feite foutief is.
De functie \inlinecode{selectionSort} roept echter \inlinecode{zoekIndexMinimum}
op met \inlinecode{i <= imax} waardoor alles correct werkt. In principe zou
men echter \inlinecode{zoekIndexMinimum} robuuster moeten maken.
\end{solution}
\end{exercise}

\begin{example}
\newcommand{\NHL}[1]{{\color{red} #1}}
\newcommand{\HL}[1]{{\color{green} #1}}
\newcommand{\MIN}[1]{\framebox{#1}}
We passen het Selection Sort algoritme toe op de rij $[7,3,1,6,4,2,9]$.
De volgende tabel bevat de waarden van \inlinecode{i}, \inlinecode{minimumIndex} en
\inlinecode{rij} juist na de oproep naar \inlinecode{verwissel};
met andere woorden, de tabel geeft weer wat we te zien krijgen
zouden we juist na \cref{line:sorteer:selection:swap} een \inlinecode{alert} toevoegen.
\begin{center}
  \begin{tabular}{c|c|c}
    {\tt \bfseries i} & {\tt \bfseries minimumIndex} & {\tt \bfseries rij} \\
    \hline
    voor de lus & & $[\NHL{7,3,1,6,4,2,9}]$ \\
    0 & 2 & $[\HL1,\NHL{3,7,6,4,2,9}]$ \\
    1 & 5 & $[\HL{1,2},\NHL{7,6,4,3,9}]$ \\
    2 & 5 & $[\HL{1,2,3},\NHL{6,4,7,9}]$ \\
    3 & 4 & $[\HL{1,2,3,4},\NHL{6,7,9}]$ \\
    4 & 4 & $[\HL{1,2,3,4,6},\NHL{7,9}]$ \\
    5 & 5 & $[\HL{1,2,3,4,6,7},\NHL{9}]$ \\
  \end{tabular}
\end{center}
\end{example}

\begin{exercise}
Op \cref{line:sorteer:selection:sort:for} van \cref{code:sorting:selection-sort} staat
als \inlinecode{for}-conditie \inlinecode{i < rij.length - 1}. Waarom staat de \inlinecode{-1} er?
\begin{solution}
Het geval \inlinecode{i == rij.length - 1} betekent dit dat we aan het laatste
element toe zijn. De rode rechterdeelrij bevat dus slechts \'e\'en enkel element.
Het minimum hiervan vinden is triviaal: het is dat element zelf.
We zouden dus \inlinecode{minimumIndex == i} bekomen. Vervolgens zouden we
op \cref{line:sorteer:selection:swap} die element met zichzelf verwisselen,
wat geen effect heeft. Met andere woorden, het laatste element krijgen we ``gratis'':
het staat automatisch reeds op de juiste plaats in de rij.
\end{solution}
\end{exercise}

%%% Local Variables: 
%%% mode: latex
%%% TeX-master: "../main"
%%% End: 
