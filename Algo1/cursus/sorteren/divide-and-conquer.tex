\section{Verdeel en Heers Algoritmen}
In de vorige paragraaf toonden we aan hoe Insertion Sort en Selection Sort
beide geconstrueerd zijn op een gemeenschappelijke basis, namelijk
dat we over twee rijen beschikken, een ongesorteerde \inlinecode{input} en een gesorteerde \inlinecode{output},
waarbij we stapsgewijs elementen verhuisden van \inlinecode{input} naar \inlinecode{output}.
Dit verhuizen bestond uit twee stappen: \emph{selectie} en \emph{plaatsing};
Insertion Sort heeft een gemakkelijke selectie maar een complexe plaatsing,
Selection Sort doet het omgekeerde.

Een gelijkaardige structuur is te vinden bij Merge Sort en Quicksort. Om dit duidelijk te maken
moeten we eerst uitleggen welke de basisstructuur is waarop beide gebaseerd zijn.

Merge Sort en Quicksort maken beide gebruik van een ``Verdeel en Heers'' methode.
Algemeen houdt dit in dat het op te lossen probleem in kleinere deelproblemen
wordt verdeeld, dewelke afzonderlijk verwerkt worden en waarvan de resultaten
in een finale stap samengevoegd worden.

In het geval van sorteeralgoritmen betekent dit dat de te sorteren rij
eerst wordt opgesplitst in twee deelrijen. Deze twee worden dan apart
gesorteerd, waarna de gesorteerde deelrijen
weer samengevoegd worden.
Het opsplitsen van rijen gaat uiteraard niet door tot in het oneindige:
uiteindelijk stuiten we op een deelrij van lengte 1, dewelke
triviaal gesorteerd is.
\Cref{fig:sorteer:divcon} toont een visuele voorstelling van dit proces.

\begin{figure}
{
  \newcommand{\IELT}{{\color{red}\square}}
  \newcommand{\OELT}{{\color{green}\square}}
  \newcommand{\X}[2]{{\tikz[baseline, remember picture]{\node[anchor=base] (divconvis#2) { #1 };}}}

  \begin{center}
    \begin{tabular}{c}
      \makebox[.8\linewidth]{\X{$\IELT\IELT\IELT\IELT\IELT\IELT\IELT\IELT$}{a}} \\[3mm]
      \makebox[.4\linewidth]{\X{$\IELT\IELT\IELT\IELT$}{b1}}%
      \makebox[.4\linewidth]{\X{$\IELT\IELT\IELT\IELT$}{b2}} \\[3mm]
      \makebox[.2\linewidth]{\X{$\IELT\IELT$}{c1}}%
      \makebox[.2\linewidth]{\X{$\IELT\IELT$}{c2}}%
      \makebox[.2\linewidth]{\X{$\IELT\IELT$}{c3}}%
      \makebox[.2\linewidth]{\X{$\IELT\IELT$}{c4}} \\[3mm]
      \makebox[.1\linewidth]{\X{$\OELT$}{d1}}%
      \makebox[.1\linewidth]{\X{$\OELT$}{d2}}%
      \makebox[.1\linewidth]{\X{$\OELT$}{d3}}%
      \makebox[.1\linewidth]{\X{$\OELT$}{d4}}%
      \makebox[.1\linewidth]{\X{$\OELT$}{d5}}%
      \makebox[.1\linewidth]{\X{$\OELT$}{d6}}%
      \makebox[.1\linewidth]{\X{$\OELT$}{d7}}%
      \makebox[.1\linewidth]{\X{$\OELT$}{d8}} \\[3mm]
      \makebox[.2\linewidth]{\X{$\OELT\OELT$}{e1}}%
      \makebox[.2\linewidth]{\X{$\OELT\OELT$}{e2}}%
      \makebox[.2\linewidth]{\X{$\OELT\OELT$}{e3}}%
      \makebox[.2\linewidth]{\X{$\OELT\OELT$}{e4}} \\[3mm]
      \makebox[.4\linewidth]{\X{$\OELT\OELT\OELT\OELT$}{f1}}%
      \makebox[.4\linewidth]{\X{$\OELT\OELT\OELT\OELT$}{f2}} \\[3mm]
      \makebox[.8\linewidth]{\X{$\OELT\OELT\OELT\OELT\OELT\OELT\OELT\OELT$}{g}}
    \end{tabular}

    \newcommand{\SPLIT}[3]{\draw[->] (divconvis#1.south) -- (divconvis#2.north east);
                           \draw[->] (divconvis#1.south) -- (divconvis#3.north west);}
    \newcommand{\MERGE}[3]{\draw[->] (divconvis#1.south east) -- (divconvis#3.north west);
                           \draw[->] (divconvis#2.south west) -- (divconvis#3.north east);}

    \begin{tikzpicture}[overlay, remember picture]
      \SPLIT{a}{b1}{b2}
      \SPLIT{b1}{c1}{c2} \SPLIT{b2}{c3}{c4}
      \SPLIT{c1}{d1}{d2} \SPLIT{c2}{d3}{d4} \SPLIT{c3}{d5}{d6} \SPLIT{c4}{d7}{d8}
      \MERGE{d1}{d2}{e1} \MERGE{d3}{d4}{e2} \MERGE{d5}{d6}{e3} \MERGE{d7}{d8}{e4}
      \MERGE{e1}{e2}{f1} \MERGE{e3}{e4}{f2}
      \MERGE{f1}{f2}{g}
    \end{tikzpicture}
  \end{center}
}
\caption{Verdeel en Heers}\label{fig:sorteer:divcon}
\end{figure}

Zoals we bij Insertion Sort en Selection Sort te maken hadden met een selectie en een plaatsing,
hebben we bij Merge Sort en QuickSort een \emph{splitsing} en een \emph{samenvoeging}.
De complexiteit moet ergens gestoken worden: Merge Sort kiest voor een simpele splitsing
in ruil voor een moeilijke samenvoeging, QuickSort doet het tegenovergestelde.

\subsection{Merge Sort} \label{sec:sorteren:merge-sort}
We geven eerst een voorbeeld van hoe Merge Sort conceptueel werkt. Dit model
zullen we in een tweede stap aanpassen om het effici\"ent te kunnen implementeren.

\begin{example} \label{example:sorteer:mergesort}
We willen de rij $[4,1,3,6,5,8,7,2]$ sorteren. We splitsen deze in twee.
Elke manier is goed, zolang we maar twee deelrijen van gelijke lengte\footnote{Indien de te splitsen
rij een oneven aantal elementen bevat, mag de ene deelrij een element meer bevatten dan de andere.
Met andere woorden, een rij van lengte 8 moet opgesplitst worden in twee rijen van 4,
een rij van lengte 9 in deelrijen van lengtes 4 en 5.} verkrijgen.
We doen dit op de meest voor de hand liggende wijze:
\[
  [4,1,3,6,5,8,7,2] \qquad\rightarrow\qquad [4,1,3,6] \quad [5,8,7,2]
\]
We sorteren beide deze deelrijen (de details hiervan slaan we over):
\[
  [1,3,4,6] \quad [2,5,7,8]
\]
Deze twee deelrijen moeten samengevoegd worden tot \'e\'en enkele gesorteerde rij.


We maken hiervoor een nieuwe lege rij aan, laten we deze \inlinecode{output} noemen.
Vervolgens vergelijken we herhaaldelijk de eerste elementen van beide deelrijen
(1 en 2 in dit voorbeeld) en we verhuizen het kleinste van de twee naar \inlinecode{output}.
Dit geeft ons, in stappen:
\[
  \newcommand{\SPACE}{\\[5mm]}
  \newcommand{\X}[2]{{\tikz[baseline, remember picture]{\node[anchor=base] (msmerging#2) {\ensuremath{#1}};}}}
  \begin{array}{c|ll@{\quad}||@{\quad}c|ll}
    {\bf stap} & \multicolumn{1}{c}{\bf deelrijen} & \textrm{\tt\bfseries output} &
    {\bf stap} & \multicolumn{1}{c}{\bf deelrijen} & \textrm{\tt\bfseries output} \\[2mm]
    1 &
    \begin{array}{l}
      [\X1a,\X3z,\X4z,\X6z] \\{}
      [\X2z,\X5z,\X7z,\X8z]
    \end{array}
    &
    [\X{}b]
    &
    5
    &
    \begin{array}{l}
      [\X6z] \\{}
      [\X5i,\X7z,\X8z]
    \end{array}
    &
    [\X1z, \X2z, \X3z, \X4z, \X{}j]
\SPACE
    2 &
    \begin{array}{l}
      [\X3z,\X4z,\X6z] \\{}
      [\X2c,\X5z,\X7z,\X8z]
    \end{array}
    &
    [\X1z, \X{}d]
    &
    6
    &
    \begin{array}{l}
      [\X6k] \\{}
      [\X7z,\X8z]
    \end{array}
    &
    [\X1z, \X2z, \X3z, \X4z, \X5z, \X{}l]
\SPACE
    3 &
    \begin{array}{l}
      [\X3e,\X4z,\X6z] \\{}
      [\X5z,\X7z,\X8z]
    \end{array}
    &
    [\X1z, \X2z, \X{}f]
    &
    7
    &
    \begin{array}{l}
      [] \\{}
      [\X7m,\X8z]
    \end{array}
    &
    [\X1z, \X2z, \X3z, \X4z, \X5z, \X6z, \X{}n]
\SPACE
    4 &
    \begin{array}{l}
      [\X4g,\X6z] \\{}
      [\X5z,\X7z,\X8z]
    \end{array}
    &
    [\X1z, \X2z, \X3z, \X{}h]
    &
    8
    &
    \begin{array}{l}
      [] \\{}
      [\X8o]
    \end{array}
    &
    [\X1z, \X2z, \X3z, \X4z, \X5z, \X6z, \X7z, \X{}p]
  \end{array}
\]
\newcommand{\MOVE}[2]{\draw[->] (msmerging#1.north) --
                                 ++(0,2pt) --
                                 ($ (msmerging#2.north) + (0,12pt) $) --
                                 (msmerging#2.north);}
\newcommand{\MOVEB}[2]{\draw[->] (msmerging#1.south) --
                                  ++(0,-2pt) -- 
                                  ($ (msmerging#2.south) + (0,-8pt) $) --
                                  (msmerging#2.south);}
\begin{tikzpicture}[overlay, remember picture]
  \MOVE{a}{b}
  \MOVEB{c}{d}
  \MOVE{e}{f}
  \MOVE{g}{h}
  \MOVEB{i}{j}
  \MOVE{k}{l}
  \MOVEB{m}{n}
  \MOVEB{o}{p}
\end{tikzpicture}
\end{example}

In dit voorbeeld negeerden we hoe we de deelrijen zelf sorteerden.
Dit klinkt vrij absurd: het ganse doel was om te illustreren hoe we een rij sorteren en
juist dat deel slaan we over.

In feite hebben we niets overgeslagen: het sorteren van de deelrijen
gebeurt op exact dezelfde manier als waarop we de volledige rij zelf aanpakken, namelijk
opsplitsen, sorteren en samenvoegen. Dit klinkt weliswaar
als een cirkelredenering: als het sorteren van een rij telkens weer
afhangt van het sorteren van twee andere rijen, hebben we de situatie dan niet
enkel exponentieel erger gemaakt?

Het is belangrijk op te merken dat de deelrijen telkens kleiner worden:
eerst hebben we in \cref{example:sorteer:mergesort} een rij van lengte 8, dewelke
we opsplitsen in 2 deelrijen van lengte 4, die vervolgens leiden tot 4 ``deeldeelrijen'' van lengte 2.
Uiteindelijk komen we uit op 8 rijtjes van elk een enkel element, dewelke automatisch
gesorteerd zijn en geen verdere behandeling vergen. Het principe
gebruikt in \cref{example:sorteer:mergesort} kan dus weldegelijk gebruikt worden
om te sorteren.

\Cref{code:sorting:merge-sort} toont een implementatie van Merge Sort. Merk op dat we
de functie \inlinecode{voegSamen} nergens gedefinieerd hebben; dit stellen we nog even uit.

\codefigure[numbers=left]{sorteren/mergesort.js}{Merge Sort}{code:sorting:merge-sort}

\begin{itemize}
  \item De functie \inlinecode{mergeSortRec}
        (\crefrange{line:sorteer:merge:mergeSortRec}{line:sorteer:merge:mergeSortRec-end} voert
        het eigenlijke werk uit. Vermits het algoritme recursief wordt
        opgeroepen op alsmaar kleiner wordende deelrijen, is het noodzakelijk
        om aan te geven over welke deelrij we het hebben.
        \inlinecode{begin} en \inlinecode{einde} stellen
        de grenzen voor. Bijvoorbeeld, beschikken we over \inlinecode{rij = [1,5,4,3,2,6]}
        en willen we de deelrij \inlinecode{[4,3,2]} sorteren, dan kunnen we hiervoor
        \inlinecode{mergeSortRec(rij, 2, 4)} gebruiken.
  \item Op \cref{line:sorteer:merge:rec:recleft,line:sorteer:merge:rec:recright}
        roept \inlinecode{mergeSortRec} zichzelf op. We moeten ervoor zorgen
        dat er een pad bestaat waarin dit \emph{niet} gebeurt, anders hebben we te maken met
        oneindige recursie. Voor Merge Sort willen we de
        triviale gevallen waarbij de lijst slechts \'e\'en element
        apart afhandelen (dus geen splits-sorteer-samenvoeging). Dit gebeurt
        door middel van de \inlinecode{if} op \cref{line:sorteer:merge:rec:if}.
        De conditie \inlinecode{begin < einde} komt overeen met het
        checken of een lijst meer dan \'e\'en element bevat. Zo ja
        mogen \crefrange{line:sorteer:merge:rec:midden}{line:sorteer:merge:rec:merge}
        uitgevoerd worden. Zo niet, dan moet er niets gebeuren vermits
        een rij bestaande uit een enkel element reeds gesorteerd is.
  \item \Crefrange{line:sorteer:merge:rec:midden}{line:sorteer:merge:rec:merge}
        stellen het splitsen-sorteren-samenvoegen voor.
        \begin{description}
          \item[Splitsen] Op \cref{line:sorteer:merge:rec:midden} wordt de index uitgerekend
            van het middelste element. \inlinecode{Math.floor} rondt af naar beneden:
            deze oproep is nodig indien de te sorteren rij een oneven aantal elementen bevat.
          \item[Sorteren] Vervolgens worden de deelrijen gesorteerd op
            \cref{line:sorteer:merge:rec:recleft,line:sorteer:merge:rec:recright}.
            Let goed op de indices: de linkerdeelrij gaat van \inlinecode{begin}
            tot \inlinecode{midden}, de rechterdeelrij van \inlinecode{midden+1} tot \inlinecode{einde}.
          \item[Samenvoegen] Uiteindelijk worden de twee deelrijen weer samengevoegd op
            \cref{line:sorteer:merge:rec:merge}.
        \end{description}
        Dit is een visualisering van de betekenissen van de verschillende indices:
        \[
          \newcommand{\HL}[1]{{\color{green}#1}}
          \begin{array}{cccccccccccc}
            && \rotatebox{-90}{\hskip2mm{\tt begin}$\rightarrow$}
            &&& \rotatebox{-90}{{\tt midden}$\rightarrow$}
            &&& \rotatebox{-90}{\hskip2mm{\tt einde}$\rightarrow$} \\
            \square & \square & \HL\square & \HL\square & \HL\square & \HL\square &
            \HL\square & \HL\square &\HL\square & \square & \square & \square \\
          \end{array}
        \]
        Hier is \inlinecode{begin==2}, \inlinecode{einde==8} en \inlinecode{midden==5}.
\end{itemize}

\begin{exercise}
Op \cref{line:sorteer:merge:rec:if} in \cref{code:sorting:merge-sort}
staat de conditie \inlinecode{begin < einde}. Wat gebeurt er als we dit vervangen
door \inlinecode{begin != einde}? Dit maakt enkel een verschil uit wanneer
\inlinecode{begin > einde}, maar dit komt toch nooit voor?
\begin{solution}
\inlinecode{mergeSortRec} zal zichzelf nooit oproepen met \inlinecode{begin > einde},
maar \inlinecode{mergeSort} zal echter wel \inlinecode{mergeSortRec} kunnen oproepen
met \inlinecode{begin==0} en \inlinecode{einde==-1}: dit komt voor indien
men de lege rij wil sorteren. Hoewel dit wel een randgeval is, is het belangrijk
dat dit ook correct wordt afgehandeld.
\end{solution}
\end{exercise}


\subsubsection{Samenvoegen van rijen}
Om Merge Sort volledig af te werken rest ons nog de functie \inlinecode{voegSamen} te defini\"eren.
Hoewel het conceptueel eenvoudig is (zie \cref{example:sorteer:mergesort} op
\cpageref{example:sorteer:mergesort}), is het algoritme vrij omslachtig zoals te zien is
in \cref{code:sorting:merge-sort:merge}. \Cref{example:sorteer:mergesort:voegsamen}
op \cpageref{example:sorteer:mergesort:voegsamen} geeft een gedetailleerde uitwerking.

\codefigure[numbers=left]{sorteren/mergesort-voegsamen.js}{Samenvoegen van gesorteerde deelrijen}{code:sorting:merge-sort:merge}

\begin{itemize}
  \item In eerste instantie is het belangrijk duidelijk te defini\"eren
        wat de betekenis van elk argument is.
        \begin{itemize}
          \item \inlinecode{rij} is een rij die twee gesorteerde deelrijen bevat, dewelke samengevoegd
                horen te worden.
          \item \inlinecode{begin} is de index van het eerste element van de linkerdeelrij.
          \item \inlinecode{midden} is de index van het laatste element van de linkerdeelrij.
          \item \inlinecode{midden+1} komt dan overeen met de index van het eerste element
                van de rechterdeelrij.
          \item \inlinecode{einde} is de index van het laatste element van de rechterdeelrij.
        \end{itemize}
        Gevisualiseerd geeft dit
        \[
          \settowidth{\MYLEN}{\inlinecode{midden+1} $\rightarrow\quad$}
          \newcommand{\BOX}[2]{\rotatebox{-90}{\parbox{\MYLEN}{\flushright #1\ensuremath{\color{#2}\square}}}}
          \newcommand{\IDX}[2]{\BOX{\inlinecode{#1} \ensuremath{\rightarrow\;}}{#2}}
          \BOX{}{black} \BOX{}{black} \BOX{}{black}
          \IDX{begin}{green}
          \BOX{}{green} \BOX{}{green}
          \IDX{midden}{green}
          \IDX{midden+1}{blue}
          \BOX{}{blue}
          \BOX{}{blue}
          \IDX{einde}{blue}
          \BOX{}{black}
          \BOX{}{black}
        \]
        Opdat \inlinecode{voegSamen} correct werkt, moeten de elementen
        van \inlinecode{begin}\dots\inlinecode{midden} gesorteerd zijn, alsook
        die van \inlinecode{midden+1}\dots\inlinecode{einde}.
  \item Zoals uitgelegd werd in \cref{example:sorteer:mergesort} (\cpageref{example:sorteer:mergesort})
        gebeurt het samenvoegen van de twee deelrijen door telkens het kleinste eerste element
        te verhuizen naar een \inlinecode{output} rij.
        De lokale variabele \inlinecode{links} wordt gebruikt om de
        linkerdeelrij af te gaan en moet dus de indices van \inlinecode{begin} tot \inlinecode{midden} afgaan.
        Analoog voor \inlinecode{rechts} gaat de rechterdeelrij af en
        varieert van \inlinecode{midden+1} tot \inlinecode{einde}.
        Op \cref{line:sorteer:mergesort:voegSamen:output} maken we de rij
        \inlinecode{output} aan, dewelke voldoende groot moet zijn om de elementen
        uit de twee deelrijen te bevatten: deze moet dus \inlinecode{einde-begin+1} lang zijn.
        De variabele \inlinecode{i} wordt gebruikt om aan te geven waar in
        \inlinecode{output} het volgende element moet geplaatst worden. Deze zal
        dus alle waarden van \inlinecode{0} tot \inlinecode{einde-begin} afgaan.
  \item In de eerste \inlinecode{while}-lus
        (\crefrange{line:sorteer:mergesort:voegSamen:while1}{line:sorteer:mergesort:voegSamen:while1-end})
        gaat het algoritme telkens op zoek welke van de twee deelrijen
        het volgende kleinste element bevat (\cref{line:sorteer:mergesort:voegSamen:while1:if}).
        Indien dit de linkerdeelrij is, wordt het element gekopieerd naar
        \inlinecode{output} (\cref{line:sorteer:mergesort:voegSamen:while1:lcopy}) en
        worden de indices \inlinecode{links} (\cref{line:sorteer:mergesort:voegSamen:while1:linc})
        en \inlinecode{i} (\cref{line:sorteer:mergesort:voegSamen:while1:inc}) ge\"incrementeerd.
        Het geval dat de rechterdeelrij het kleinste element bevat wordt analoog afgehandeld
        (\cref{line:sorteer:mergesort:voegSamen:while1:rcopy,line:sorteer:mergesort:voegSamen:while1:rinc,%
        line:sorteer:mergesort:voegSamen:while1:inc}).
  \item Opdat de eerste \inlinecode{while}-lus de eerste elementen van beide deelrijen
        kan vergelijken, mogen deze niet leeg zijn. De uitdrukking
        \inlinecode{links <= midden && rechts <= einde} op \cref{line:sorteer:mergesort:voegSamen:while1}
        drukt deze voorwaarde uit.
  \item Uiteindelijk zal \'e\'en van beide deelrijen uitgeput raken terwijl de andere nog minstens
        \'e\'en element bevat. Zo kan je in \cref{example:sorteer:mergesort} (\cpageref{example:sorteer:mergesort})
        zien dat in stap 6 het laatste element van de linkerdeelrij wordt overgezet
        naar \inlinecode{output} en er nog twee elementen overblijven in de rechterdeelrij (7 en 8).
        Zulke overblijvende elementen worden niet afgehandeld door de eerste \inlinecode{lus} maar
        moeten wel nog overgeplaatst worden naar \inlinecode{output}. Dit is de taak
        van de twee lussen op
        \crefrange{line:sorteer:mergesort:voegSamen:while2}{line:sorteer:mergesort:voegSamen:while3-end}.
  \item De lus op \crefrange{line:sorteer:mergesort:voegSamen:while2}{line:sorteer:mergesort:voegSamen:while2-end}
        dient om alle resterende elementen uit de linkerdeelrij naar \inlinecode{output} te kopi\"eren.
        De andere lus
        (\crefrange{line:sorteer:mergesort:voegSamen:while3}{line:sorteer:mergesort:voegSamen:while3-end})
        zorgt voor resterende elementen in de rechterdeelrij. Merk op dat slechts \'e\'en van deze
        beide lussen zal worden uitgevoerd! Stel bijvoorbeeld dat de linkerdeelrij
        als eerste uitgeput raakt: dit betekent dat \inlinecode{links > midden}, met als gevolg
        dat de eerste lus
        (\crefrange{line:sorteer:mergesort:voegSamen:while1}{line:sorteer:mergesort:voegSamen:while1-end})
        zal eindigen. We komen dan de tweede lus tegen
        (\crefrange{line:sorteer:mergesort:voegSamen:while2}{line:sorteer:mergesort:voegSamen:while2-end}).
        Deze heeft als conditie \inlinecode{links <= midden}, wat evalueert naar \inlinecode{false}.
        Deze lus wordt dus volledig overgeslagen. Dan komt de beurt aan de derde lus
        (\crefrange{line:sorteer:mergesort:voegSamen:while3}{line:sorteer:mergesort:voegSamen:while3-end}):
        deze zal wel minstens \'e\'en iteratie doorlopen.
  \item Als laatste stap overschrijven we de deelrijen in \inlinecode{rij}
        met de inhoud van \inlinecode{output}
        (\crefrange{line:sorteer:mergesort:voegSamen:for}{line:sorteer:mergesort:voegSamen:for-end}).
        Het is in principe mogelijk het samenvoegen te laten gebeuren
        zonder hulprij \inlinecode{output} en rechtstreeks \inlinecode{rij} zelf te manipuleren,
        maar dit is te complex voor deze cursus.
\end{itemize}

\begin{example} \label{example:sorteer:mergesort:voegsamen}
We beschouwen
\begin{center}
   \settowidth{\MYLEN}{\inlinecode{midden+1} $\rightarrow$}
   \newcommand{\BOX}[3]{\rotatebox{-90}{\parbox{\MYLEN}{\flushright #1\rotatebox{90}{\ensuremath{\color{#2}#3}}}}}
   \newcommand{\IDX}[2]{\BOX{\inlinecode{#1} \ensuremath{\rightarrow\;}}{#2}}

   \BOX{}{black}{7}
   \BOX{}{black}{0}
   \IDX{begin}{green}{2}
   \BOX{}{green}{4}
   \BOX{}{green}{8}
   \IDX{midden}{green}{9}
   \BOX{}{blue}{3}
   \IDX{einde}{blue}{5}
   \BOX{}{black}{1}
\end{center}
m.a.w.\ we onderzoeken hoe de oproep \inlinecode{voegSamen([7,0,2,4,8,9,3,5,1], 2, 5, 7)} afgehandeld wordt.
\newcommand{\LEFTARR}{{\color{green} \downarrow}}
\newcommand{\RIGHTARR}{{\color{blue} \bigtriangledown}}
\newcommand{\IARR}{\vee}
\begin{center}
  \newcommand{\LEFT}[1]{\ensuremath{\stackrel{\LEFTARR}}{{#1}}}
  \newcommand{\RIGHT}[1]{\ensuremath{\stackrel{\RIGHTARR}}{{#1}}}
  \newcommand{\II}[1]{\ensuremath{\stackrel{\IARR}}{{#1}}}
  \begin{tabular}{cccccc}
    {\bf lus} & {\tt\bfseries links} & {\tt\bfseries rechts} & {\tt\bfseries i} & {\tt rij} & {\tt output} \\
    \hline\hline
    (init) & {\tt 2} & {\tt 6} & {\tt 0} &
    $[7,0,{\color{green}\LEFT2,4,8,9},{\color{blue}\RIGHT3,5},1]$ &
    $[ \II{\times}, \times, \times, \times, \times, \times ]$
\\ \hline
    1 & {\tt 3} & {\tt 6} & {\tt 1} &
    $[7,0,{\color{green}2,\LEFT4,8,9},{\color{blue}\RIGHT3,5},1]$ &
    $[ 2, \II{\times}, \times, \times, \times, \times ]$
\\ \hline
    1 & {\tt 3} & {\tt 7} & {\tt 2} &
    $[7,0,{\color{green}2,\LEFT4,8,9},{\color{blue}3,\RIGHT5},1]$ &
    $[ 2, 3, \II{\times}, \times, \times, \times ]$
\\ \hline
    1 & {\tt 4} & {\tt 7} & {\tt 3} &
    $[7,0,{\color{green}2,4,\LEFT8,9},{\color{blue}3,\RIGHT5},1]$ &
    $[ 2, 3, 4, \II{\times}, \times, \times ]$
\\ \hline
    1 & {\tt 4} & {\tt 8} & {\tt 4} &
    $[7,0,{\color{green}2,4,\LEFT8,9},{\color{blue}3,5},\RIGHT1]$ &
    $[ 2, 3, 4, 5, \II{\times}, \times ]$
\\ \hline
    2 & {\tt 5} & {\tt 8} & {\tt 5} &
    $[7,0,{\color{green}2,4,8,\LEFT9},{\color{blue}3,5},\RIGHT1]$ &
    $[ 2, 3, 4, 5, 8, \II{\times} ]$
\\ \hline
    2 & {\tt 6} & {\tt 8} & {\tt 6} &
    $[7,0,{\color{green}2,4,8,9},{\color{blue}\LEFT3,5},\RIGHT1]$ &
    $[ 2, 3, 4, 5, 8, 9 ] \II{\phantom{\times}}$
  \end{tabular}
\end{center}
De eerste kolom geeft weer door welke \inlinecode{while}-lus de huidige stap werd afgehandeld.
De eerste rij komt overeen met de initi\"ele toestand.
De waarden van \inlinecode{links}, \inlinecode{rechts} en \inlinecode{i}
worden symbolisch voorgesteld door de pijlen $\LEFTARR$, $\RIGHTARR$ en $\IARR$, respectievelijk.
\end{example}


\subsection{Quicksort} \label{sec:sorteren:quicksort}
QuickSort is zoals Merge Sort een verdeel-sorteer-samenvoeging algoritme,
maar waarbij de complexiteit niet ligt in het samenvoegen, maar het opsplitsen.
We beginnen met een voorbeeld om een beeld te schetsen van de werking van het algoritme.

\begin{example} \label{example:sorteer:quicksort}
We sorteren dezelfde rij als in \cref{example:sorteer:mergesort},
namelijk $[4,1,3,6,5,8,7,2]$. In een eerste stap moeten we deze
rij splitsen in twee deelrijen van (ongeveer) gelijke lengte.
Dit doen we door een \emph{spil} te kiezen en vervolgens
de elementen kleiner of gelijk aan deze spil te scheiden
van de elementen groter dan deze spil: dit vormen de twee deelrijen.
\[
  \begin{array}{c}
    \square\square\square\square\dots\square\square\square\square \\
    \swarrow \quad \searrow \\
    \underbrace{\square\square\dots\square\square}_{\leq \mathrm{spil}} \quad \underbrace{\square\square\dots\square\square}_{> \mathrm{spil}}
  \end{array}
\]
De spilkeuze bepaalt hoe de rij opgesplitst wordt in twee deelrijen.
We tonen een aantal mogelijke spilkeuzes en de bijhorende splitsing:
\[
  \newcommand{\EXAMPLE}[3]{#1 & #2 & #3}
  \begin{array}{c|c|c}
    \mathbf{spil} & \mathbf{deelrij \leq spil} & \mathbf{deelrij > spil} \\
    \hline\hline
    \EXAMPLE{0}{[]}{[4,1,3,6,5,8,7,2]} \\ \hline
    \EXAMPLE{3}{[1,3,2]}{[4,6,5,8,7]} \\ \hline
    \EXAMPLE{4}{[1,3,2,4]}{[6,5,8,7]} \\ \hline
    \EXAMPLE{6}{[1,3,2,4,6,5]}{[8,7]} \\ \hline
    \EXAMPLE{10}{[4,1,3,6,5,8,7,2]}{[]}
  \end{array}
\]
We zien dat indien we 4 als spil kiezen, we twee deelrijen van gelijke lengte bekomen: dit is
ideaal. Algemeen kunnen we stellen dat we in principe telkens de mediaan\footnote{Het ``middelste'' element.}
van de rij moeten kiezen als spil. In de praktijk wordt dit echter niet gedaan.

Verschillende implementaties van Quicksort maken verschillende keuzes voor de spil: sommige
nemen het eerste element uit de rij als spil (4 in ons geval), anderen het laatste (2), nog anderen
kiezen een telkens een willekeurig element uit de rij, etc. Telkens hopen ze dat hun
spilkeuze leidt tot twee deelrijen van ongeveer gelijke lengte\footnote{Je kan je afvragen
waarom het zo belangrijk is dat de deelrijen gelijke lengtes hebben. Het is wiskundig aantoonbaar
dat dit de meest effici\"ente manier van sorteren is. Vergelijk het met een spelletje hoger lager
waar je een getal tussen 0 en 100 moet raden: de optimale speelwijze bestaat erin
om telkens het middelste element te raden: je gokt eerst 50, dan (indien lager) 25, dan (indien hoger)
37 of 38, etc. Iemand die gewoon 1, 2, 3, 4, 5, \dots gokt zal er veel langer over doen om het getal te vinden.}
, maar uiteraard is dit verre van altijd het geval. Dit is een pragmatische keuze:
het kost meer tijd om telkens de perfecte spil te zoeken dan om met suboptimale spils te werken.

In deze tekst zullen we als spil gewoon het eerste element uit de rij kiezen. In ons geval
is dit 4 en is dit toevallig de optimale spil. We bekomen als deelrijen $[1,3,2,4]$
en $[6,5,8,7]$.

Na de splitsing komt het sorteren van de deelrijen: dit geeft ons $[1,2,3,4]$ en
$[5,6,7,8]$. Als laatste stap moeten we de deze twee gesorteerde deelrijen
samenvoegen. Gezien we weten dat alle elementen van de eerste deelrij
kleiner zijn dan die uit de tweede, kunnen we gewoon de twee deelrijen
aan elkaar plakken:
\[
  [\underbrace{1,\;2,\;3,\;4}_{\textrm{linker}},\;\underbrace{5,\;6,\;7,\;8}_{\textrm{rechter}}]
\]
\end{example}

De aanpak ge\'illustreerd in \cref{example:sorteer:quicksort} is elegant:
hij voldoet perfect aan de splitsen-sorteren-samenvoegen structuur. Het kiezen
van een suboptimale spil (zoals het kiezen van het eerste element
van de rij, in plaats van de mediaan) heeft echter nefaste gevolgen, zoals
aangetoond wordt in volgend voorbeeld.

\begin{example}\label{example:sorteer:quicksort-oops}
We willen de rij $[9,4,3,8,6,2,1]$ sorteren. De spil is dan 9 en de twee
deelrijen worden $[9,4,3,8,6,2,1]$ en $[]$. Als we de eerste deelrij
dan willen sorteren, passen we hetzelfde algoritme toe: we kiezen weer
9 als spil, en dit levert weer dezelfde deelrijen op. Zo blijft dit doorgaan
tot in het oneindige.
\end{example}

\Cref{example:sorteer:quicksort-oops} toont aan dat indien
we de spil verkeerd kiezen (bv.\ de spil blijkt het grootste element uit de rij te zijn),
dit leidt tot oneindige recursie. Om dit te voorkomen,
moeten we ervoor zorgen dat beide deelrijen op zijn minst
\emph{korter} zijn dan de rij zelf.

We lossen dit eenvoudig op als volgt: we kiezen eerst de spil, we \emph{halen deze uit de rij},
we splitsen de rij zonder de spil, sorteren de bekomen deelrijen, en voegen alles samen.

\begin{example}
Het sorteren van $[4,1,3,6,5,8,7,2]$ gaat nu als volgt:
\begin{itemize}
  \item We kiezen 4 als spil.
  \item We splitsen de rij zonder spil op: we verkrijgen $[1,3,2]$ en $[6,5,8,7]$.
  \item We sorteren de deelrijen: $[1,2,3]$ en $[5,6,7,8]$.
  \item We voegen alles samen:
        \[
          [ \overbrace{1,\;2,\;3}^{\textrm{links}},\;
          \stackrel{\stackrel{\scriptstyle\textrm{spil}}{\downarrow}}{4},\;
          \overbrace{5,6,7,8}^{\textrm{rechts}} ]
        \]
\end{itemize}
\end{example}

\codefigure[numbers=left]{sorteren/quicksort.js}{Quicksort}{code:sorting:quicksort}

We kunnen nu overgaan tot de implementatie. We pakken het top-down aan
en geven eerst de code voor \inlinecode{quickSort} zelf, zie hiervoor
\cref{code:sorting:quicksort}.
\begin{itemize}
  \item Zoals bij Merge Sort (\cref{code:sorting:merge-sort}) defini\"eren we
        twee functies: \inlinecode{quickSortRec}
        (\crefrange{line:sorteer:quick:quickSortRec}{line:sorteer:quick:quickSortRec-end})
        die van een gegeven \inlinecode{rij} het stuk gaande van \inlinecode{begin}
        tot \inlinecode{einde} sorteert, en \inlinecode{quickSort}
        (\crefrange{line:sorteer:quick:quickSort}{line:sorteer:quick:quickSort-end})
        die \inlinecode{quickSortRec} oproept op de volledige rij.
  \item \Cref{line:sorteer:quick:quickSortRec:if} checkt of er iets gesorteerd moet worden:
        indien de te sorteren deelrij 0 of 1 element bevat, hoeft er niets gedaan te worden.
  \item Op \cref{line:sorteer:quick:quickSortRec:partition} roepen we de
        totnogtoe ongedefinieerde functie \inlinecode{partitioneer} op.
        We verwachten dat deze functie de gegeven deelrij \inlinecode{rij[begin...einde]} partitioneert.
        Als resultaat geeft het terug waar de spil gepositioneerd staat (zie
        \cref{example:sorteer:quicksort:partitioneer}).
  \item We weten dat de spil juist staat. We hoeven nu enkel nog de linkerdeelrij
        \inlinecode{rij[begin...spilIndex-1]} en de rechterdeelrij
        \inlinecode{rij[spilIndex+1...einde]} te sorteren; dit doen
        we met recursieve oproepen op
        \cref{line:sorteer:quick:quickSortRec:recl,line:sorteer:quick:quickSortRec:recr}.
  \item Voor de samenvoegingsstap hoeven we niets te doen doordat we het sorteren
        in-place uitvoeren.
\end{itemize}

\begin{example} \label{example:sorteer:quicksort:partitioneer}
\newcommand{\HERE}{\ensuremath{\downarrow}}
Algemeen verwachten we hetvolgende van \inlinecode{partitioneer}:
\begin{center}
  \begin{tikzpicture}
    \foreach \x in {0, 1, ..., 2} {
      \node at (\x, 0) {$\square$};
      \node at (\x, 2) {$\square$};
      \draw [->] (\x, 1.9) -- (\x, 0.1);
    };
    \foreach \x in {3, 4, ..., 8} {
      \node at (\x, 0) {$\color{green}\square$};
      \node at (\x, 2) {$\color{green}\square$};
    };
    \foreach \x in {9, 10, ..., 11} {
      \node at (\x, 0) {$\square$};
      \node at (\x, 2) {$\square$};
      \draw [->] (\x, 1.9) -- (\x, 0.1);
    };
    \node (begin) at (3, 2.75) { {\tt begin} };
    \draw [->] (begin) -- (3, 2.2);
    \node (end) at (8, 2.75) { {\tt end} };
    \draw [->] (end) -- (8, 2.2);
    \draw [->] (3, 1.9) -- (3, 1) -- (5, 1) -- (5, 0.2);
    \draw (3.8, 1.9) -- (8.2, 1.9);
    \draw (2.8, 0.2) -- (4.2, 0.2);
    \draw (5.8, 0.2) -- (8.2, 0.2);
    \draw [->] (5.5, 1.9) -- (5.5, 1.5) -- (3.5, 1.5) -- (3.5, 0.2);
    \draw [->] (6.5, 1.9) -- (6.5, 1.5) -- (7, 1.5) -- (7, 0.2);
    \node at (4.45, 1.65) {$\scriptstyle \leq \mathrm{spil}$};
    \node at (7, 1.65) {$\scriptstyle > \mathrm{spil}$};
  \end{tikzpicture}
\end{center}
Concrete voorbeelden zijn
\begin{center}
  \newcommand{\PIVOT}[1]{\stackrel{{\color{black}\HERE}}{#1}}
  \newcommand{\SUB}[1]{{\color{green}#1}}
  \begin{tabular}{ccccc}
    {\tt\bfseries rij} &
    {\tt\bfseries begin} &
    {\tt\bfseries einde} &
    {\tt\bfseries rij} na oproep &
    {\bf resultaat} \\
    \hline
    $[\SUB{3,1,5}]$ & 0 & 2 & $[\SUB{1,\PIVOT3,5}]$ & 1 \\{}
    $[\SUB{3,2,1}]$ & 0 & 2 & $[\SUB{1,2,\PIVOT3}]$ & 2 \\{}
    $[\SUB{4,9,2,3,8,7,5,6,1}]$ & 0 & 8 & $[\SUB{3,1,2,\PIVOT4,8,7,5,6,9}]$ & 3 \\{}
    $[4,9,\SUB{2,3,8,7,5},6,1]$ & 2 & 6 & $[4,9,\SUB{\PIVOT2,3,8,7,5},6,1]$ & 2 \\{}
    $[7,4,6,\SUB{2,1,5,8,9},3]$ & 3 & 7 & $[7,4,6,\SUB{1,\PIVOT2,5,8,9},3]$ & 4 \\{}
  \end{tabular}
\end{center}
Deze tabel geeft weer wat de verwachte effecten zijn van de oproep \inlinecode{partitioneer(rij, begin, einde)}.
De kolom ``\inlinecode{rij}'' bevat de array waarvan we een deelrij wensen te partitioneren.
\inlinecode{begin} en \inlinecode{einde} geven weer wat de grenzen zijn van deze deelrij.
De kolom ``\inlinecode{rij} na oproep'' toont aan hoe de array er na de oproep moet uitzien:
elementen kleiner dan of gelijk aan de spil moeten links van de spil staan, de rest rechts ervan.
Het resultaat van de oproep moet de index van de spil zijn: deze wordt ook visueel aangegeven
met een $\HERE$.
\end{example}

Nu we weten wat we van \inlinecode{partitioneer} verwachten, kunnen we een algoritme
ervoor ontwikkelen. Intu\"itief werkt het als volgt: we weten
dat links de elementen kleiner dan of gelijk aan de spil moeten geplaatst worden en
rechts die die groter zijn. We beginnen aan beide uiteinden van de rij en bewegen naar binnen toe.
Wanneer we links een te groot element vinden en rechts een te klein, verwisselen we deze van plaats.
Uiteindelijk bereiken we het midden en is de rij gepartitioneerd. We geven een voorbeeld.

\begin{example}
\newcommand{\GENERIC}[3]{\rotatebox{-90}{\parbox{\MYLEN}{\flushright #1\ensuremath{\color{#2}\rotatebox{90}{$#3$}}}}}
\newcommand{\OUT}[1]{\GENERIC{}{black}{#1}}
\newcommand{\IN}[1]{\GENERIC{}{green}{#1}}
\newcommand{\IDX}[2]{\GENERIC{\inlinecode{#1} $\rightarrow\;$}{green}{#2}}
\newcommand{\LEFT}{\ensuremath{\downharpoonleft}}
\newcommand{\RIGHT}{\ensuremath{\downharpoonright}}
\newcommand{\HL}[1]{{\color{green}#1}}
\settowidth{\MYLEN}{\inlinecode{begin} $\rightarrow\quad$}
\[
  \OUT{8}\OUT{2}\OUT{0}
  \IDX{begin}{5}\IN{3}\IN{2}\IN{6}\IN{1}\IN{7}\IN{0}\IN{2}\IN{9}\IN{1}\IN{8}\IDX{einde}{7}
  \OUT{3}\OUT{5}
\]
We kiezen het eerste element van de deelrij, dit is dus 5. We gaan nu de rest van de deelrij
(het stuk zonder de spil) partitioneren. We beginnen aan de uiteindes:
\[
  \begin{array}{ccccccccccccccccc}
      &   &   &   & \LEFT &   &   &   &   &   &   &   &   &   & \RIGHT &   &   \\
    8 & 2 & 0 & \HL5 & \HL3 & \HL2 & \HL6 & \HL1 & \HL7 & \HL0 & \HL2 & \HL9 & \HL1 & \HL8 & \HL7 & 3 & 5
  \end{array}
\]
De linkerpijl verschuiven we naar rechts zolang het elementen tegenkomt
die kleiner dan of gelijk zijn aan de spil. De rechterpijl verplaatsen we naar links
zolang de elementen groter zijn dan de spil.
\[
  \begin{array}{ccccccccccccccccc}
      &   &   &   &  &   & \LEFT &   &   &   &   &   & \RIGHT  &   & &   &   \\
    8 & 2 & 0 & \HL5 & \HL3 & \HL2 & \HL6 & \HL1 & \HL7 & \HL0 & \HL2 & \HL9 & \HL1 & \HL8 & \HL7 & 3 & 5
  \end{array}
\]
De 6 en de 1 staan niet op hun plaats. We verwisselen ze:
\[
  \begin{array}{ccccccccccccccccc}
      &   &   &   &  &   & \LEFT &   &   &   &   &   & \RIGHT  &   & &   &   \\
    8 & 2 & 0 & \HL5 & \HL3 & \HL2 & \HL1 & \HL1 & \HL7 & \HL0 & \HL2 & \HL9 & \HL6 & \HL8 & \HL7 & 3 & 5
  \end{array}
\]
We laten de pijlen weer naar elkaar toe gaan:
\[
  \begin{array}{ccccccccccccccccc}
      &   &   &   &  &   & &   & \LEFT &   & \RIGHT &   & &   & &   &   \\
    8 & 2 & 0 & \HL5 & \HL3 & \HL2 & \HL1 & \HL1 & \HL7 & \HL0 & \HL2 & \HL9 & \HL6 & \HL8 & \HL7 & 3 & 5
  \end{array}
\]
Weeral verwisselen we de elementen van plaats:
\[
  \begin{array}{ccccccccccccccccc}
      &   &   &   &  &   & &   & \LEFT &   & \RIGHT &   & &   & &   &   \\
    8 & 2 & 0 & \HL5 & \HL3 & \HL2 & \HL1 & \HL1 & \HL2 & \HL0 & \HL7 & \HL9 & \HL6 & \HL8 & \HL7 & 3 & 5
  \end{array}
\]
We bewegen de pijlen:
\[
  \begin{array}{ccccccccccccccccc}
      &   &   &   &  &   & &   & & \RIGHT & \LEFT &   & &   & &   &   \\
    8 & 2 & 0 & \HL5 & \HL3 & \HL2 & \HL1 & \HL1 & \HL2 & \HL0 & \HL7 & \HL9 & \HL6 & \HL8 & \HL7 & 3 & 5
  \end{array}
\]
We zien dat de pijlen elkaar voorbijgegaan zijn: dit betekent dat de rij
gepartitioneerd is. We willen nu nog de spil juist plaatsen.
De rechterpijl $\RIGHT$ (die nu wel links staat) staat boven het laatste
element van de linkerdeelrij. We kunnen de spil daarheen verhuizen:
\[
  \begin{array}{ccccccccccccccccc}
      &   &   &   &  &   & &   & & \RIGHT & \LEFT &   & &   & &   &   \\
    8 & 2 & 0 & \HL0 & \HL3 & \HL2 & \HL1 & \HL1 & \HL2 & \HL5 & \HL7 & \HL9 & \HL6 & \HL8 & \HL7 & 3 & 5
  \end{array}
\]
Merk op dat de buitenliggende elementen ($[8, 2, 0]$ links, $[3, 5]$ rechts) volledig genegeerd moeten worden.
\end{example}

\codefigure[numbers=left]{sorteren/quicksort-partition.js}{Partitioneren}{code:sorting:quicksort:partition}

De implementatie voor de functie \inlinecode{partitioneer} is te vinden in
\cref{code:sorting:quicksort:partition}.

{
\newcommand{\LEFT}{\ensuremath{\downharpoonleft}}
\newcommand{\RIGHT}{\ensuremath{\downharpoonright}}
\begin{itemize}
  \item Op \cref{line:sorteer:quicksort:partition:spil} kiezen we de spil, namelijk
        het eerste element van de deelrij.
  \item Op \cref{line:sorteer:quicksort:partition:links,line:sorteer:quicksort:partition:rechts}
        plaatsen we onze pijlen $\LEFT$ en $\RIGHT$ op hun initi\"ele posities.
  \item De \inlinecode{while}-lus op
        \crefrange{line:sorteer:quicksort:partition:while-1}{line:sorteer:quicksort:partition:while-1-end}
        verplaatst de linkerpijl~$\LEFT$ naar rechts tot een foutief geplaatst element
        gevonden wordt. De \inlinecode{while}-lus op
        \crefrange{line:sorteer:quicksort:partition:while-2}{line:sorteer:quicksort:partition:while-2-end}
        doet analoog voor de rechterpijl~$\RIGHT$.
  \item \Cref{line:sorteer:quicksort:partition:swap} wisselt de twee foutief geplaatste elementen
        van plaats.
  \item De spil wordt door \cref{line:sorteer:quicksort:partition:lastswap} op zijn plaats gezet.
  \item De positie van de spil wordt uiteindelijk als
        resultaat teruggegeven (\cref{line:sorteer:quicksort:partition:return}).
\end{itemize}
}


%%% Local Variables: 
%%% mode: latex
%%% TeX-master: "../main"
%%% End: 
