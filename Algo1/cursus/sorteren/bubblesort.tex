\section{Bubble Sort} \label{sec:sorteren:bubble-sort}
We beginnen in \cref{sec:sorteren:bubble-sort:werking} met een conceptuele
beschrijving van de werking van Bubble Sort. Daarna
implementeren we het algoritme (\cref{sec:sorteren:bubble-sort:implementatie}).
Deze implementatie is echter voor verbetering vatbaar: we zullen
het dan ook optimiseren, hetgeen we doen in twee stappen
(\cref{sec:sorteren:bubble-sort:opt1} en \cref{sec:sorteren:bubble-sort:opt2}).

\subsection{Werking} \label{sec:sorteren:bubble-sort:werking}
Een (vrij triviale) eigenschap van een gesorteerde rij is dat indien men
twee willekeurige opeenvolgende elementen beschouwt uit de rij, het eerste kleiner of gelijk is aan het tweede.
Bijvoorbeeld, de rij $[1, 5, 6, 9]$ is duidelijk gesorteerd. We kunnen dit nazien
door op te merken dat elk opeenvolgend paar elementen correct geordend is:
\[
  1 \leq 5 \qquad 5 \leq 6 \qquad 6 \leq 9
\]
Omgekeerd kunnen we dus ook stellen dat een ongeordende rij op zijn minst ergens
twee opeenvolgende elementen moet bevatten die niet juist geordend staan. Zo is
de rij $[1,4,6,5,8,9]$ niet helemaal juist gesorteerd.
\[
  1 \leq 4 \qquad 4 \leq 6 \qquad \underbrace{\color{red} 6 \not\leq 5}_{!!!} \qquad 5 \leq 8 \qquad 8 \leq 9
\]
Om een rij te sorteren kunnen we deze afspeuren
op zoek naar zulke verkeerd geordende koppels. Telkens we er zo \'e\'en vinden,
verwisselen we deze van plaats. Uiteindelijk zullen we
geen verkeerd geordend koppel meer vinden, wat betekent dat de rij gesorteerd is.

\begin{example}
We voeren dit eens handmatig uit: stel, we willen de rij $[5, 7, 2, 6]$ sorteren.
We merken op dat 7 en 2 verkeerd staan;
met andere woorden, we moeten deze verwisselen van plaats. De rij wordt dan $[5, 2, 7, 6]$.
Nu zien we dat 5 en 2 fout geordend zijn;
we wisselen deze om en we krijgen $[2, 5, 7, 6]$. We herhalen dit proces zolang we
``foute koppels'' vinden. In tabelvorm ziet
dit er zo uit:

{
\newcommand{\HL}[1]{\textbf{#1}}
\newcommand{\swap}{\ensuremath{\searrow\kern-10pt\swarrow}}
\begin{center}
  \begin{tabular}{rcccccccl}
    {[} & 5, && \HL7, && \HL2, && 6 & ] \\
    &&&& \swap \\
    {[} & \HL5, && \HL2, && 7, && 6 & ] \\
    && \swap \\
    {[} & 2, && 5, && \HL7, && \HL6 & ] \\
    &&&&&& \swap \\
    {[} & 2, && 5, && 6, && 7 & ] \\
  \end{tabular}
\end{center}
}
We hadden bij de tweede stap in plaats van 5 en 2 te verwisselen ook 7 en 6 kunnen van plaats doen wisselen.
Het maakt niet uit in welke volgorde we de elementen van plaats verwisselen: het valt wiskundig
te bewijzen\footnote{Hoeven jullie weliswaar niet te kunnen.} dat we sowieso uiteindelijk
zullen uitmonden in een gesorteerde rij.
\end{example}

Indien er dus meerdere verkeerd geordende koppels zijn in een rij,
kunnen we dus in principe willekeurig kiezen welke koppel we
als volgende onder handen nemen. Het Bubble Sort algoritme pakt het
als volgt aan: het begint helemaal links in de rij, vergelijkt de eerste twee elementen en
wisselt deze om indien nodig (d.i.\ dus als het eerste element groter is dan het tweede).
Daarna verschuift de aandacht van het algoritme
\'e\'en stap naar rechts: het 2de en 3de element uit de rij worden vergeleken en weeral
omgewisseld indien nodig. Zo gaat dit door
tot het voorlaatste en laatste element. Dit afgaan van de volledige rij van links
naar rechts wordt een \emph{pass} genoemd.

\begin{example}
We werken een Bubble Sort pass uit. We nemen als voorbeeld de rij $[4,1,3,7,6,2]$.
{
\newcommand{\HL}[1]{\textbf{#1}}
\newcommand{\swap}{\ensuremath{\searrow\kern-10pt\swarrow}}
\newcommand{\keep}{\ensuremath{\downarrow}}
\begin{center}
  \begin{tabular}{rcccccccccccl}
    {[} & \HL4, && \HL1, && 3, && 7, && 6, && 2 & ] \\
    && \swap && \phantom{\swap} && \phantom{\swap} && \phantom{\swap} && \phantom{\swap} \\
    {[} & 1, && \HL4, && \HL3, && 7, && 6, && 2 & ] \\
    &&&& \swap \\
    {[} & 1, && 3, && \HL4, && \HL7, && 6, && 2 & ] \\
    &&&&& \keep && \keep \\
    {[} & 1, && 3, && 4, && \HL7, && \HL6, && 2 & ] \\
    &&&&&&&& \swap \\
    {[} & 1, && 3, && 4, && 6, && \HL7, && \HL2 & ] \\
    &&&&&&&&&& \swap \\
    {[} & 1, && 3, && 4, && 6, && 2, && 7 & ] \\
  \end{tabular}
\end{center}
}
In de eerste stap wordt 4 met 1 vergeleken. Deze staan duidelijk in de verkeerde
volgorde, dus moeten ze omgewisseld worden. Hetzelfde geldt voor de tweede stap:
4 staat voor 3, dus moeten ze ook verwisseld worden. In de derde stap blijkt
dat 4 en 7 reeds juist staan: deze kunnen we op hun plaats laten staan.
Daarna volgen er nog twee stappen waarbij telkens blijkt dat de elementen
verkeerd geordend staan.
\end{example}

Aan de hand van dit voorbeeld kunnen we zien dat \'e\'en enkele pass niet zal voldoen
om de rij volledig te sorteren. De oplossing is eenvoudig: we blijven maar passes
uitvoeren zolang de rij niet gesorteerd is.

\begin{example} \label{example:sorteren:passes}
We voeren een tweede pass uit:
{
\newcommand{\HL}[1]{\textbf{#1}}
\newcommand{\swap}{\ensuremath{\searrow\kern-10pt\swarrow}}
\newcommand{\keep}{\ensuremath{\downarrow}}
\begin{center}
  \begin{tabular}{rcccccccccccl}
    {[} & \HL1, && \HL3, && 4, && 6, && 2, && 7 & ] \\
    & \keep & \phantom{\swap} & \keep & \phantom{\swap} && \phantom{\swap} && \phantom{\swap} && \phantom{\swap} \\
    {[} & 1, && \HL3, && \HL4, && 6, && 2, && 7 & ] \\
    &&& \keep && \keep \\
    {[} & 1, && 3, && \HL4, && \HL6, && 2, && 7 & ] \\
    &&&&& \keep && \keep \\
    {[} & 1, && 3, && 4, && \HL6, && \HL2, && 7 & ] \\
    &&&&&&&& \swap \\
    {[} & 1, && 3, && 4, && 2, && \HL6, && \HL7 & ] \\
    &&&&&&&&& \keep && \keep \\
    {[} & 1, && 3, && 4, && 2, && 6, && 7 & ]
  \end{tabular}
\end{center}}
{\flushleft Een derde pass transformeert de rij naar}
\begin{center}
  [ 1, 3, 4, 6, 2, 7 ] $\quad\rightarrow\quad \dots \quad\rightarrow\quad$ [ 1, 3, 2, 4, 6, 7 ]
\end{center}
Een vierde pass blijkt ook nog eens nodig te zijn:
\begin{center}
  [ 1, 3, 2, 4, 6, 7 ] $\quad\rightarrow\quad \dots \quad\rightarrow\quad$ [ 1, 2, 3, 4, 6, 7 ]
\end{center}
\end{example}
We kunnen ons afvragen hoeveel passes er nodig zijn om een rij gesorteerd te krijgen.
Indien de rij slechts \'e\'en element bevat, is het triviaal: zo een rij is reeds gesorteerd,
dus er zijn geen passes nodig.
In het geval van een rij bestaande uit twee elementen, onderscheiden we twee mogelijkheden:
ofwel staan ze reeds in de juiste volgorde en zijn er dus geen passes nodig,
ofwel staan ze verkeerd maar zal een enkele pass de rij gesorteerd krijgen.
Algemener is het aantoonbaar dat om een rij van $N$ aantal elementen te sorteren,
er maximaal $N - 1$ passes nodig zijn.
Met andere woorden, gegeven een rij van lengte $N$ kan men er gewoon $N-1$ passes op
uitvoeren om dan met zekerheid te eindigen met een gesorteerde rij.
We zijn nu klaar om dit algoritme te implementeren.

\subsection{Implementatie} \label{sec:sorteren:bubble-sort:implementatie}
Bubble Sort moet twee elementen uit een rij kunnen verwisselen.
De andere sorteeralgoritmen die we later bespreken zullen
deze operatie ook veelvuldig gebruiken, waardoor het interessant wordt hiervoor een aparte functie te
defini\"eren. Zodoende hoeven we de code voor het verwisselen van twee elementen niet telkens te dupliceren.
De code voor de verwisselfunctie is te zien in \cref{code:sorting:swap}.

\codefigure{sorteren/swap.js}{Een functie om twee elementen uit een rij van plaats
te verwisselen}{code:sorting:swap}

\begin{example}
We illustreren het gebruik van de verwisselfunctie.
\examplecode{sorteren/swap-vb.js}
\end{example}

\codefigure[numbers=left]{sorteren/bubblesort.js}{Bubble Sort}{code:sorting:bubble-sort}

Zoals besproken in de vorige paragraaf, kunnen we een rij van lengte $N$ sorteren door
$N - 1$ passes erop uit te voeren.
\Cref{code:sorting:bubble-sort} toont de volledige code voor het Bubble Sort algoritme. We lichten de code toe:

\begin{itemize}
  \item \Crefrange{line:sorteer:bubble-sort:pass-begin}{line:sorteer:bubble-sort:pass-end} defini\"eren
        de functie \inlinecode{bubbleSortPass} dewelke een enkele pass uitvoert op de gegeven rij.
        Het is een hulpfunctie die we later in de code (op~\cref{line:sorteer:bubble-sort:sort-passcall})
        zullen oproepen.
  \item \Crefrange{line:sorteer:bubble-sort:pass-forloop}{line:sorteer:bubble-sort:pass-forloop-end}
        vormen een lus waarbij de variabele \lstinline{i} de waarden \inlinecode{0} tot \inlinecode{rij.length-2}
        afgaat: dit komt overeen met het van links naar rechts afgaan van de rij.
        Merk op dat bij het afgaan van een rij de iteratievariabele (hier dus \inlinecode{i})
        \emph{doorgaans} \inlinecode{0} tot \inlinecode{rij.length-1} afgaat; dit is hier echter
        niet het geval: \inlinecode{i} stopt hier \'e\'en stap eerder omdat
        we telkens \inlinecode{rij[i]} en \inlinecode{rij[i+1]} willen vergelijken.
        Indien \inlinecode{i} zou gaan tot \inlinecode{rij.length-1}, zouden
        we in de laatste lusiteratie \inlinecode{rij[rij.length]} willen uitlezen, wat voorbij het
        einde van de rij is.
  \item \Cref{line:sorteer:bubble-sort:pass-if} checkt of de elementen
        met index \inlinecode{i} en \inlinecode{i+1} in de juiste volgorde staan.
        Zoniet, wordt \cref{line:sorteer:bubble-sort:pass-swap} uitgevoerd:
        deze roept de functie \inlinecode{verwissel} op om de elementen
        om te wisselen.
  \item \Crefrange{line:sorteer:bubble-sort:sort-begin}{line:sorteer:bubble-sort:sort-end}
        bevatten de definitie voor de functie \inlinecode{bubbleSort}. Als we een rij willen
        sorteren, kunnen we deze functie erop loslaten (zie later, \cref{example:sorteer-bubblesort}).
  \item \Crefrange{line:sorteer:bubble-sort:sort-for}{line:sorteer:bubble-sort:sort-for-end}
        stellen een lus voor die \inlinecode{rij.length-1} keer een pass uitvoert.
        Stel dat de rij lengte 5 heeft, dan willen we dat er 4 maal een pass wordt uitgevoerd.
        We beginnen bij \inlinecode{i = 0} en itereren zolang \inlinecode{i < rij.length - 1}.
        De variabele \inlinecode{i} zal dus de waarden \inlinecode{0}, \inlinecode{1},
        \inlinecode{2} en \inlinecode{3} aannemen, wat overeenkomt met 4 iteraties, exact wat we willen.
        Wanneer \inlinecode{i} gelijk wordt aan \inlinecode{4}, is de conditie
        \inlinecode{i < rij.length - 1} niet meer waar en wordt de lus afgebroken.
  \item Op~\cref{line:sorteer:bubble-sort:sort-passcall} roepen we onze hulpfunctie
        \inlinecode{bubbleSortPass} op, dewelke dus een enkele pass uitvoert op \inlinecode{rij}.
\end{itemize}

\begin{example} \label{example:sorteer-bubblesort}
Om een rij te sorteren kunnen we volgende code gebruiken:
\examplecode{sorteren/bubblesort-usage.js}
\end{example}

\begin{exercise}
Op \cref{line:sorteer:bubble-sort:pass-if} van \cref{code:sorting:bubble-sort}
gebruiken we de operator \inlinecode{>} om de elementen op posities \inlinecode{i}
en \inlinecode{i+1} te vergelijken. Wat gebeurt er indien we \inlinecode{<} gebruiken in de plaats?
\begin{solution}
De elementen zullen van groot naar klein gesorteerd worden.
\end{solution}
\end{exercise}

\subsection{Eerste Optimisatie} \label{sec:sorteren:bubble-sort:opt1}
De Bubble Sort implementatie uit de vorige paragraaf (\cref{code:sorting:bubble-sort})
kan wat effici\"enter gemaakt worden: nu worden voor een rij van $N$ elementen
$N-1$ passes uitgevoerd, maar soms is dit overbodig en is de rij reeds gesorteerd
na minder passes. Beschouw bijvoorbeeld het geval waarin we \inlinecode{bubbleSort}
oproepen met een reeds gesorteerde rij: idealiter willen we dat het algoritme
dit doorheeft en meteen zijn werk stopzet. In \cref{example:sorteren:passes} (\cpageref{example:sorteren:passes})
zagen we eveneens dat 4 passes bleken voldoende te zijn, terwijl de rij 6 elementen bevatte.

\codefigure{sorteren/bubblesort-opt1-naive.js}{Een na\"ieve manier om Bubble Sort
te optimiseren}{code:sorting:bubble:naive-opt}

We zouden het algoritme conceptueel kunnen neerschrijven zoals getoond in \cref{code:sorting:bubble:naive-opt}:
in plaats van $N-1$ passes uit te voeren voor een rij van lengte $N$, blijven
we maar passes uitvoeren zolang de rij niet gesorteerd is.
We moeten echter bij elke iteratie checken of de rij nu eindelijk gesorteerd is,
maar dit checken kost ook tijd. Hebben we door deze ``optimisatie'' het algoritme
echt wel sneller gemaakt?

Stel dat we een rij van lengte 10 hebben waarvoor blijkt dat er 9 passes nodig
zijn om het deze gesorteerd te krijgen. In de eerste versie van onze implementatie
(\cref{code:sorting:bubble-sort}) moesten we gewoon 9 keer \inlinecode{bubbleSortPass} oproepen.
In onze nieuwe versie (\cref{code:sorting:bubble:naive-opt}) hebben we eveneens
9 keer \inlinecode{bubbleSortPass} moeten oproepen, maar daarbij ook nog eens
10 keer \inlinecode{isGesorteerd}! De nieuwe versie werkt in dit geval dus trager.

Er zijn uiteraard ook gevallen waar onze nieuwe implementatie sneller werkt. Men kan stellen
dat deze sneller werkt voor quasi-gesorteerde rijen (waarvoor weinig passes nodig zijn)
maar trager voor ``moeilijk sorteerbare'' rijen. Het is betwistbaar of dit een nuttige optimisatie is.

We kunnen het gelukkig iets slimmer aanpakken: we kunnen \inlinecode{bubbleSortPass}
zelf gebruiken om te weten te komen of een rij gesorteerd is: zo slaan
we twee vliegen in \'e\'en klap en is een aparte oproep naar \inlinecode{isGesorteerd} overbodig geworden.

Wanneer \inlinecode{bubbleSortPass} een niet-gesorteerde rij ontvangt,
zal het bepaalde elementen van plaats verwisselen. Indien de rij in kwestie echter
w\'el reeds gesorteerd is, zal \inlinecode{bubbleSortPass} alle elementen
ongemoeid laten. Als we \inlinecode{bubbleSortPass} laten bijhouden of er elementen
verwisseld werden, kunnen we ook detecteren of de rij gesorteerd is. We implementeren
deze aanpak in \cref{code:sorting:bubble:opt}.

\codefigure[numbers=left]{sorteren/bubblesort-opt1.js}{Een geoptimiseerde Bubble Sort}{code:sorting:bubble:opt}

\begin{itemize}
  \item De functie \inlinecode{bubbleSortPass} breiden we uit met een lokale variabele
        \inlinecode{bewerkt} op \cref{line:sorteer:bubble-sort-opt:var-bewerkt}.
        Deze gebruiken we om bij te houden of er ergens elementen verwisseld werden.
  \item In het begin van een pass zijn er nog geen omwisselingen verricht:
        \inlinecode{bewerkt} moet dus ge\"initialiseerd worden op \inlinecode{false}
        op~\cref{line:sorteer:bubble-sort-opt:var-bewerkt}.
        Bij elke omwisseling overschrijven we (op~\cref{line:sorteer:bubble-sort-opt:bewerkt-true})
        deze waarde met \inlinecode{true}. 
  \item Op lijn \cref{line:sorteer:bubble-sort-opt:return} geven we de waarde
        van \inlinecode{bewerkt} terug. Zo kunnen we in de functie \inlinecode{bubbleSort}
        te weten komen of \inlinecode{bubbleSortPass} al dan niet de rij heeft aangepast.
  \item Ook de functie \inlinecode{bubbleSort} voorzien we van een lokale variabele:
        op \cref{line:sorteer:bubble-sort-opt:var-ongesorteerd} declareren
        we \inlinecode{ongesorteerd}, dewelke aangeeft of de rij gesorteerd is of niet.
  \item We initialiseren de variabele \inlinecode{ongesorteerd} op \inlinecode{true}:
        we weten op dat moment immers nog niet of \inlinecode{rij} gesorteerd is
        en het is veilig ervan uit te gaan dat dit nog niet het geval is.
  \item De lus op \crefrange{line:sorteer:bubble-sort-opt:while}{line:sorteer:bubble-sort-opt:while-end}
        zal blijven itereren zolang \inlinecode{ongesorteerd} \inlinecode{true} is.
        Op \cref{line:sorteer:bubble-sort-opt:ongesorteerd-set} roepen we \inlinecode{bubbleSortPass}
        op. Zoals eerder uitgelegd, geeft deze functie als resultaat terug of er elementen werden van
        plaats verwisseld. Indien het resultaat gelijk is aan \inlinecode{true},
        moeten we verder itereren; is het \inlinecode{false}, dan mogen we stoppen met sorteren.
        Dit bekomen we door het resultaat toe te kennen aan \inlinecode{ongesorteerd}.
\end{itemize}

\begin{exercise}
Op \cref{line:sorteer:bubble-sort-opt:cmp} van \cref{code:sorting:bubble:opt}
wordt \inlinecode{rij[i]} vergeleken met \inlinecode{rij[i+1]}. Dit gebeurt
met de \inlinecode{>} operator. Wat als men \inlinecode{>=} gebruikt?
\begin{solution}
Indien de rij twee gelijke elementen bevatten, zullen deze constant verwisseld
worden en eindigt het algoritme nooit.
\end{solution}
\end{exercise}

\subsection{Tweede Optimisatie} \label{sec:sorteren:bubble-sort:opt2}
Om deze tweede optimisatie te begrijpen, moeten we eerst wat inzicht
verwerven in de werking van Bubble Sort.

\begin{example} \label{example:sorteren:bubble:passes}
\newcommand{\HL}[1]{\textbf{#1}}
We beschouwen de rij $[ 9, 5, 2, 8, 1, 3, 7, 6 ]$ en voeren er Bubble Sort passes op uit:
\begin{center}
  \begin{tabular}{c|c}
    \textbf{pass} & \textbf{rij} \\
    \hline
    0 & $[ 9, 5, 2, 8, 1, 3, 7, 6 ]$ \\
    1 & $[ 5, 2, 8, 1, 3, 7, 6, \HL9 ]$ \\
    2 & $[ 2, 5, 1, 3, 7, 6, \HL8, \HL9 ]$ \\
    3 & $[ 2, 1, 3, \HL5, \HL6, \HL7, \HL8, \HL9 ]$ \\
    4 & $[ \HL1, \HL2, \HL3, \HL5, \HL6, \HL7, \HL8, \HL9 ]$ \\
  \end{tabular}
\end{center}
In het vet werd aangeduid welke deelrij (te beginnen van rechts) volledig gesorteerd is.
We zien dat dit bij elke pass groeit: na pass 1 is enkel de deelrij $[9]$ gesorteerd,
na pass 2 groeit deze tot $[8, 9]$, terwijl de 3de pass goed werk lijkt te leveren:
plots staan de elementen $[5, 6, 7, 8, 9]$ allemaal juist geordend. De 4de pass
werkt het af: de 3 overgebleven elementen staan nu ook correct.
\end{example}

Een pass gaat telkens de \emph{volledige} rij af, terwijl dit niet nodig blijkt te zijn:
de 2de pass uit \cref{example:sorteren:bubble:passes} hoeft slechts de 7 eerste elementen te bekijken,
de 3de pass enkel de 6 eerste, de 4de pass kan zich beperken tot de 3 eerste, etc.

Hoe kunnen we nu te weten komen tot waar een pass de rij moet afgaan?
In de vorige paragraaf zagen we hoe \inlinecode{bubbleSortPass} bijhield
of er omwisselingen werden uitgevoerd of niet. Maar wat als we ook de index
onthouden van het laatst verplaatste element? Indien een pass een rij doorloopt
en het laatst verwisselde koppel elementen indices \inlinecode{i} en \inlinecode{i+1}
hebben, dan kunnen we daaruit afleiden dat alle elementen vanaf \inlinecode{i+2} juist
gesorteerd staan. We maken gebruik van dit principe in \cref{code:sorting:bubble:opt2}.

\codefigure[numbers=left]{sorteren/bubblesort-opt2.js}{Finale versie van Bubble Sort}{code:sorting:bubble:opt2}

\begin{itemize}
  \item We moeten twee wijzigingen aanbrengen aan de functie \inlinecode{bubbleSortPass}. Ten eerste
        moeten we de index bijhouden van het laatste verplaatste element. Dit gebeurt
        op \cref{line:sorteer:bubble-sort-opt2:laatstBewerkt-set}. De initi\"ele waarde
        voor de lokale variabele \inlinecode{laatstBewerkt} kiezen we $-1$:
        deze ongeldige index gebruiken we om symbolisch aan te geven dat
        er (nog) geen laatste gewijzigde element is.
  \item Ten tweede moet \inlinecode{bubbleSortPass} verteld worden tot waar het de rij moet afgaan.
        Dit wordt voorgesteld door het argument \inlinecode{tot} dat ingevoerd wordt
        op \cref{line:sorteer:bubble-sort-opt2:bubbleSortPass}. \inlinecode{max} stelt de index
        voor van het laatste element dat bekeken moet worden door de huidige pass.
        Merk ook op dat de iteratievariabele \inlinecode{i} nu varieert van
        \inlinecode{0} tot \inlinecode{max-1} in plaats van tot \inlinecode{rij.length-2}\footnote{Indien
        je verward wordt door de \inlinecode{-1} enerzijds en de \inlinecode{-2} anderzijds:
        \inlinecode{i} moet gaan van \inlinecode{0} tot de \emph{index van het voorlaatste element}.
        Vermits \inlinecode{max} de index is van het laatste element, is de bovengrens \inlinecode{max-1}.
        In de vorige implementatie was de bovengrens \inlinecode{rij.length-2}: indien de rij $N$ elementen
        bevat, dan is de index van het laatste element $N-1$ en de index van het voorlaatste element dus
        $N-2$, vandaar \inlinecode{rij.length-2}.}.
  \item Ook de functie \inlinecode{bubbleSort} moet wat aangepast worden.
        De lokale variabele \inlinecode{ongesorteerdTot} wordt gebruikt
        om de index van het laatste ongesorteerde element te bevatten.
        Vermits in het begin de rij als volledig ongesorteerd beschouwd wordt,
        moet deze variabele ge\"initialiseerd worden op \inlinecode{rij.length-1}.
  \item Op \cref{line:sorteer:bubble-sort-opt2:body} wordt \inlinecode{bubbleSortPass} opgeroepen.
        Merk op dat we nu de index van het laatste ongesorteerde elementen mee moeten geven
        (het tweede argument). \inlinecode{bubbleSortPass} geeft de nieuwe
        waarde voor \inlinecode{ongesorteerdTot} terug, dit resultaat moeten we dus bewaren.
  \item De lusconditie op \cref{line:sorteer:bubble-sort-opt2:while} checkt
        of de rij volledig gesorteerd is. We weten dat \inlinecode{bubbleSortPass}
        als resultaat \inlinecode{-1} oplevert indien de rij gesorteerd is,
        wat ervoor zal zorgen dat het itereren zal eindigen.
\end{itemize}

\begin{example}
\newcommand{\HL}[1]{\textbf{#1}}
We herbekijken dezelfde reeks als uit \cref{example:sorteren:bubble:passes}
en rekenen handmatig uit hoe de waarden van de variabelen
\inlinecode{rij} en \inlinecode{gesorteerdTot} evolueren.
We tonen de waarden van beide variabelen nadat een pass werd uitgevoerd;
met andere woorden, het is alsof we een \inlinecode{alert} hebben toegevoegd
juist na \cref{line:sorteer:bubble-sort-opt2:body}.
\begin{center}
  \begin{tabular}{c|c|c}
    \textbf{pass} & {\tt rij} & {\tt gesorteerdTot} \\
    \hline
1 & {\tt [ 5, 2, 8, 1, 3, 7, 6, 9 ]} & {\tt 7} \\
2 & {\tt [ 2, 5, 1, 3, 7, 6, 8, 9 ]} & {\tt 6} \\
3 & {\tt [ 2, 1, 3, 5, 6, 7, 8, 9 ]} & {\tt 5} \\
4 & {\tt [ 1, 2, 3, 5, 6, 7, 8, 9 ]} & {\tt 1} \\
5 & {\tt [ 1, 2, 3, 5, 6, 7, 8, 9 ]} & {\tt -1} \\
  \end{tabular}
\end{center}
\end{example}

\begin{exercise}
\Cref{line:sorteer:bubble-sort-opt2:while} in \cref{code:sorting:bubble:opt2} toont als
while-conditie \inlinecode{i >= 0}. Kunnen we dit veranderen naar \inlinecode{i > 0}? Wat zijn hiervan
de gevolgen?
\begin{solution}
\inlinecode{i > 0} geeft exact hetzelfde als \inlinecode{i >= 0} vermits
\inlinecode{bubbleSortPass} nooit \inlinecode{0}
kan teruggeven: ofwel worden er geen elementen aangepast en geeft de functie \inlinecode{-1}
terug, ofwel wordt \inlinecode{i+1} teruggegeven waarbij \inlinecode{i} minimaal gelijk is aan \inlinecode{0}.
Met andere woorden, indien het element met index \inlinecode{0} wordt verplaatst, \emph{moet} ook
het element met index \inlinecode{1} zijn betrokken geweest. Vermits de index van het laatste verplaatste element
(d.i.\ dat met de hoogste index) moet teruggegeven worden, betekent dit dat \inlinecode{1} het resultaat moet zijn
(of een hogere index indien er nog andere elementen verwisseld werden).
\end{solution}
\end{exercise}

%%% Local Variables: 
%%% mode: latex
%%% TeX-master: "../main"
%%% End: 
