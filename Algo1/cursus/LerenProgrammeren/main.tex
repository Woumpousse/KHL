\chapter{Leren programmeren}
Een belangrijke bekwaamheid van een informaticus is het programmeren van software.
Dit is vanzelfsprekend voor informatici die als programmeurs tewerkgesteld worden,
maar zelfs informatici die bijvoorbeeld netwerkbeheer of systeemmanagement doen zullen
regelmatig gebruik moeten maken van deze competentie. Zo is het natuurlijk veel
eenvoudiger om een kort scriptje te schrijven dat automatisch nieuwe gebruikers
cre\"eert, backups maakt, systemen installeert, e.d., dan al dit werk manueel te doen.

Het ontwikkelen van een programma kan je opdelen in verschillende fases. Zo moet
je bijvoorbeeld eerst weten wat je moet doen alvorens je code kan beginnen te schrijven.
Dit klinkt misschien eenvoudig, maar de ervaring leert ons dat mensen anders denken en
dat dezelfde beschrijving van een probleem door verschillende mensen anders ge\"interpreteerd
kan worden. Eenmaal je weet wat er verwacht wordt, moet je natuurlijk ook nog
weten hoe je dit moet programmeren.

Het doel van dit vak is om de studenten aan te leren hoe je vertrekt van een probleemstelling
en komt tot een werkend programma. Het vak is oefening-geori\"enteerd, wat wil zeggen dat het
grootste deel van de contacturen zal gestoken worden in het maken van oefeningen. Dit is ook de
enige manier dat je kan leren programmeren: alle ervaring die je opdoet door oefeningen te maken
zal je later kunnen hergebruiken.

Voor deze cursus gebruiken we de programmeertaal JavaScript. Uiteindelijk is de keuze van de
programmeertaal niet echt belangrijk, aangezien de concepten die in dit vak aangeleerd worden
toepasbaar zijn in eender welke andere programmeertaal.


%%% Local Variables: 
%%% mode: latex
%%% TeX-master: "../main"
%%% End: 
