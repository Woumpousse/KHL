\documentclass[a4paper]{article}
\usepackage{a4wide}
\usepackage{graphicx}
\usepackage{listings}
\usepackage{xcolor}
\usepackage{pxfonts}

\newcommand{\newexercise}[1]{\clearpage\begin{center}\Huge\bf #1\end{center}}
\newcommand{\exercisemap}[1]{\begin{center}\includegraphics[height=8cm]{#1}\end{center}}
\newcommand{\exercisemaps}[2]{\begin{center}\includegraphics[height=5cm]{#1} \hspace{4mm} \includegraphics[height=5cm]{#2}\end{center}}

\lstdefinelanguage{JavaScript}{
  keywords={typeof, new, true, false, catch, function, return,
            null, catch, switch, var, if, in, for, while, do, else, case, break},
  keywordstyle=\color{blue}\bfseries,
  ndkeywords={class, export, boolean, throw, implements, import, this},
  ndkeywordstyle=\color{darkgray}\bfseries,
  identifierstyle=\color{black},
  sensitive=false,
  comment=[l]{//},
  morecomment=[s]{/*}{*/},
  basicstyle=\tt,
  commentstyle=\color{purple}\ttfamily,
  stringstyle=\color{red}\ttfamily,
  morestring=[b]',
  morestring=[b]",
  escapeinside=\`\`,
  frame=lines
}

\lstset{language=JavaScript}

\newcommand{\code}[1]{
  \begin{center}
    \begin{minipage}{.8\linewidth}
      \lstinputlisting{#1}
    \end{minipage}
  \end{center}
}

\pagestyle{empty}

\begin{document}

\newexercise{Inleiding}
\begin{itemize}
  \item Plaats de gegeven bestanden ergens op je home-schijf, in een directory naar keuze.
  \item Start de browser Chrome op\footnote{Om het even welke browser
        die het canvas element ondersteunt is goed. Op de pc's van de KHL
        staat momenteel IE9 als standaardbrowser geconfigureerd, dewelke
        Canvas spijtig genoeg niet ondersteunt.}.
  \item Open \verb'assignment.html' in de browser. Je moet nu
        \exercisemap{ex1} te zien krijgen. Als dit niet het geval is,
        meld dit meteen.
  \item Voor elk van deze oefeningen zal je het bestand \verb'instructions.js' moeten aanpassen.
  \item Er is een functie \verb'drive' gedefinieerd met een lege body (= het stuk tussen accolades).
        \code{drive.js}
        De functie heeft een enkele parameter genaamd \verb'car': deze
        stelt de auto voor die je zal moeten besturen en zonder ongelukken
        naar de eindbestemming (het groene vakje) brengen.
\end{itemize}


\newexercise{Oefening 1}
\exercisemap{ex1}

\begin{itemize}
  \item De auto is \'e\'en vakje verwijderd van zijn eindbestemming.
  \item Je kan de auto vooruit laten gaan m.b.v.\ de volgende code:
        \code{forward.js}
  \item Om je instructies uit te voeren, ga naar de browser, refresh (snelkoppeling F5) en druk op de ``Go!'' knop.
  \item Je moet nu het wagentje zien vooruitrijden. Hierna komt
        een boodschap die je feliciteert met het uitmuntende werk dat je geleverd
        hebt. Bij latere (moeilijkere) oefeningen zal dit minder neerbuigend overkomen :-).
\end{itemize}


\newexercise{Oefening 2}
\exercisemap{ex2}

\begin{itemize}
  \item Selecteer de tweede oefening.
  \item Deze keer is de bestemming twee vakjes ver.
  \item E\'en van de basiselementen van algoritmes is ``sequentie'':
        dit houdt simpelweg in dat instructies na elkaar
        kunnen uitgevoerd worden.
  \item Om een sequentie van opdrachten neer te schrijven
        moet je de opdrachten gewoon na elkaar in te typen.
\end{itemize}


\newexercise{Oefening 3}
\exercisemap{ex3}

\begin{itemize}
  \item Het wordt alsmaar moeilijker: de bestemming is nu \emph{drie} vakjes weg.
  \item Je zal wel een vermoeden hebben hoe dit probleem op te lossen.
\end{itemize}



\newexercise{Oefening 4}
\exercisemap{ex4}

\begin{itemize}
  \item Surprise, surprise\dots De eindbestemming is vier vakjes weg.
  \item We willen voorgoed van zulke idiote opdrachten afraken.
  \item We moeten een manier vinden om automatisch een aantal keer
        dezelfde opdracht te herhalen.
  \item Hiervoor bieden talen \emph{lussen}.
  \item JavaScript biedt een aantal soorten lussen. We focussen hier op de \verb'for'-lus.
  \item We kunnen de \verb'for'-lus als volgt gebruiken:
        \code{for.js}
  \item In de les zullen we de \verb'for'-lus in detail bespreken.
  \item Een korte uitleg: er wordt een lokale variabele \verb'i' aangemaakt
        dewelke ge\"initialiseerd wordt op op \verb'0'.
        Zolang \verb'i' niet gelijk is aan \verb'4' wordt
        de body herhaald en \verb'i' ge\"incrementeerd (\verb'i = i + 1').
        In bovenstaande lus betekent dit dus dat de body
        van de lus 4x herhaald wordt (voor \verb'i = 0', \verb'i = 1',
        \verb'i = 2' en \verb'i = 3'; wanneer \verb'i' gelijk wordt aan \verb'4',
        wordt de lus stopgezet).
\end{itemize}


\newexercise{Oefening 5}
\exercisemap{ex5}

\begin{itemize}
  \item Dankzij onze \verb'for' lus kunnen we gemakkelijk willekeurig ver rijden.
\end{itemize}



\newexercise{Oefening 6}
\exercisemap{ex6}

\begin{itemize}
  \item Nu zijn ze ons gewoon aan het tergen. Onze machtige lus is hier niet tegen opgewassen.
  \item We zijn verplicht gebruik te maken van een nieuwe feature van onze wagen: draaien.
  \item Gebruik \verb'turnRight(car)' om de auto ter plekke naar rechts te draaien.
\end{itemize}


\newexercise{Oefening 7}
\exercisemap{ex7}

\begin{itemize}
  \item Het ziet ernaar uit dat we twee lussen zullen nodig hebben\dots
\end{itemize}


\newexercise{Oefening 8}
\exercisemap{ex8}

\begin{itemize}
  \item Ha, dit is geen uitdaging meer.
\end{itemize}


\newexercise{Oefening 9}
\exercisemap{ex9}

\begin{itemize}
  \item Altijd weer dat tellen\dots
  \item Hier moeten we iets op vinden.
  \item Blijkt dat onze auto een sensor heeft!
  \item De sensor kan gebruikt worden om te zien of er een muur juist voor de auto staat.
  \item Gebruik maken van de sensor gaat met \verb'sensor(car)'.
        De functie geeft \verb'true' terug als er een obstakel is, \verb'false' in het andere geval.
  \item We kunnen onze \verb'for'-lus aanpassen om iets te herhalen zolang er geen muur in de weg is:
        \code{for-sensor.js}
  \item Het uitroepteken \verb'!' is de logische negatie: we willen immers
        iets herhalen zolang er \emph{niet} iets in de weg is.
\end{itemize}


\newexercise{oefening 10}
\exercisemap{ex10}

\begin{itemize}
  \item Aha! We hoeven onze code niet eens wijzigen! Ons programma werkt
        ongeacht hoe lang de doorgangen zijn.
  \item Het is een goed idee om af en toe de code op te kuisen.
        Opkuisen doen we best als de code werkt: indien
        het opkuisen een bug invoert, dan merken we dit meteen.
  \item De \verb'for'-lus ziet er enorm lelijk uit.
  \item Javascript biedt een eenvoudigere \verb'while'-lus aan.
        We kunnen hetzelfde schrijven als voorheen:
        \code{while.js}
  \item De \verb'while'-lus is de ``basislus''. Elke andere lus
        kan vertaald worden naar een \verb'while'.
  \item De \verb'for'-lus is een gespecialiseerde versie van de
        \verb'while'-lus. Indien een \verb'while'-lus voldoet aan een bepaald
        patroon, kan het leesbaarder zijn een \verb'for'-lus te gebruiken in de plaats.
        Meer details hierover tijdens de les.
\end{itemize}


\newexercise{Oefening 11}
\exercisemap{ex11}

\begin{itemize}
  \item Och, dit is gewoon smerig\dots
  \item E\'en enkele \verb'while'-lus kan ons tot aan het einde van een straat brengen.
  \item Hier zijn acht straten.
  \item We willen vermijden dat we acht keer de \verb'while'-lus moeten herhalen; daar zijn we te lui voor.
  \item We hebben eerder een \verb'for'-lus gebruikt om iets een aantal keer te herhalen.
        We kunnen deze nu gebruiken om de \verb'while'-lus acht maal te herhalen.
  \item We hebben hier dus te maken met \emph{geneste lussen}: een lus in een lus.
\end{itemize}


\newexercise{Oefening 12}
\exercisemap{ex12}

\begin{itemize}
  \item De sensor-optie voor de wagen was vrij duur\dots
  \item We hebben daarom enkele toegevingen moeten maken\dots
  \item E\'en ervan is dat we niet naar links kunnen draaien.
  \item Hoe lossen we dit toch op\dots
\end{itemize}


\newexercise{Oefening 13}
\exercisemap{ex13}

\begin{itemize}
  \item Ok, dit lijkt veel op een vorige oefening.
\end{itemize}


\newexercise{Oefening 14}
\exercisemap{ex14}

\begin{itemize}
  \item We willen onze code wat opkuisen.
  \item We weten dat we naar links kunnen draaien door drie keer naar rechts te draaien.
  \item We weten ook hoe we door een straat heen moeten rijden tot we niet meer kunnen.
  \item Stel dat we op een gigantisch plan rondrijden en het aantal nodige
        instructies zeer groot wordt. Het zou dan leesbaarder zijn om rechtstreeks
        \verb'turnLeft' te kunnen gebruiken i.p.v.\ drie maal \verb'turnRight', etc.
  \item JavaScript (en zo goed als alle programmeertalen) laten dit toe:
        algemeen heet dit \emph{abstractie}: we verbergen ``complexe'' operaties
        achter een eenvoudige facade.
  \item We defini\"eren zelf een functie als volgt:
        \code{turnleft.js}
  \item Definieer nu ook een functie \verb'straightAhead' die een straat tot op het einde doorrijdt.
  \item Pas je instructies aan door gebruik te maken van deze nieuwe functies.
\end{itemize}


\newexercise{Oefening 15}
\exercisemap{ex15}
\begin{itemize}
  \item We willen voortaan zo abstract mogelijk werken.
  \item Maak een functie \verb'backward' die de auto \'e\'en vakje naar achter rijdt.
  \item Let er ook op dat de auto zijn oorspronkelijke richting herstelt. Met andere
        woorden, indien de auto naar het noorden gericht staat, moet na een
        oproep naar \verb'backward' de auto weer naar het noorden kijken.
  \item Als je het programma uitvoert, zal je zien dat je de tijd niet krijgt
        om je terug goed te zetten; dit is ok. De simulator onderbreekt je
        programma automatisch van zodra je de eindbestemming bereikt hebt.
\end{itemize}


\newexercise{Oefening 16}
\exercisemap{ex16}
\begin{itemize}
  \item Maak een functie \verb'turn' die de auto 180 graden draait.
  \item De functie \verb'backward' moet als volgt geschreven kunnen worden:
        \code{backward.js}
  \item Voeg de functie \verb'backSensor' toe die checkt of er iets \emph{achter} de auto staat:
        \code{backsensor.js}
  \item Deze functie moet het resultaat van \verb'sensor(car)' bewaren in een variable,
        hier \verb'result' genaamd (om het even welke naam had gekund).
  \item \verb'return' wordt gebruikt om een resultaat terug te geven aan de oproeper van de functie.
  \item Schrijf een functie \verb'fullBackward' die de auto achteruit laat rijden
        (door herhaaldelijk \verb'backward' op te roepen) tot er een muur staat achter de wagen.
\end{itemize}


\newexercise{Oefening 17}
\exercisemap{ex17}
\begin{itemize}
  \item Hoewel het programma van de vorige oefening werkt voor deze kaart,
        is het niet echt effici\"ent. Waarom draait de auto zo veel ter plekke?
  \item Dit is een voorbeeld waar abstractie kan leiden tot oneffici\"ent gedrag.
        Het voordeel van functies is dat je eerst een eenvoudige, traag werkende implementatie kan
        maken, om deze --- indien nodig --- te herschrijven.
  \item Stel dat de auto plots uitgerust wordt met de mogelijkheid om rechtstreeks
        maar links te draaien, dan hoeven we enkel de functie \verb'turnLeft' aan te passen
        om hier gebruik van te maken. Dankzij abstractie kunnen we dan
        met \'e\'en enkele wijziging het programma laten genieten van deze nieuwe feature van onze auto.
  \item Idem dito voor indien onze wagen een tweede sensor achteraan krijgt ge\"installeerd:
        weeral zouden we met minimale moeite (\verb'backSensor') ons programma hiervan kunnen laten profiteren.
  \item In de praktijk is het dus zeer belangrijk om de juiste abstracties te ontwikkelen.
  \item Optimiseer de functie \verb'fullBackward' zodat de auto
        slechts twee maal 180 graden hoeft te draaien.
\end{itemize}


\newexercise{Oefening 18+19}
\exercisemaps{ex18}{ex19}
\begin{itemize}
  \item We willen ons programma wat slimmer maken en het wat autonomer laten werken.
  \item We willen dat \'e\'enzelfde programma op beide kaarten de eindbestemming haalt.
  \item Op de linkerkaart zal men dus naar links moeten draaien, op de rechterkaart naar rechts.
  \item Het enige dat je ter beschikking hebt is de autosensor.
  \item Baseer je op \verb'backSensor' uit een vorige oefening om nu
        \verb'leftSensor' te schrijven.
  \item Implementeer nu hetvolgende algoritme:
        \begin{itemize}
          \item Rij eerst vooruit tot het einde van de doorgang
          \item Kijk dan of er naar links een doorgang is. Zo ja, draai links en rij vooruit.
          \item Zo niet, draai naar rechts en rij vooruit.
        \end{itemize}
  \item Je hebt hier \emph{voorwaardelijke uitvoering} nodig: je wilt
        eerst checken of er aan een bepaalde voorwaarde voldaan wordt eer
        je bepaalde instructies uitvoert. Hiervoor gebruik je het \verb'if'-statement:
        \code{if.js}
\end{itemize}


\newexercise{Oefening 20}
\exercisemap{ex20}
\begin{itemize}
  \item Je kan de \verb'if'-statement uit de vorige oefening plaatsen in een lus
        om snel tot bij de eindbestemming van deze kaart te geraken.
  \item De gouden regel bij programmeren: laat zoveel mogelijk door de computer zelf doen.
\end{itemize}


\newexercise{Oefening 21}
\exercisemap{ex21}
\begin{itemize}
  \item Deze kaart zou je programma moeten kunnen oplossen zonder enige wijziging aan te brengen.
\end{itemize}


\newexercise{Oefening 22}
\exercisemap{ex22}
\begin{itemize}
  \item Je subliem algoritme dat zo goed werkte blijkt hier zijn weg te verliezen.
  \item We moeten de auto de derde straat links laten nemen.
  \item Om dit te doen zonder zelf vakjes te tellen (herinner je: we willen
        zo weinig mogelijk werk zelf verrichten), maken we gebruik van de sensor:
        \begin{itemize}
          \item Rij een vakje vooruit.
          \item Zolang vrij niet links is, rij vooruit.
          \item Rij voorbij deze straat.
          \item Zolang vrij niet links is, rij vooruit.
          \item Rij voorbij deze straat.
          \item Zolang vrij niet links is, rij vooruit.
        \end{itemize}
  \item Dit algoritme zorgt er dus voor dat de auto
        stilstaat aan de 3de straat naar links.
        Deze straat inrijden houdt dan in dat er links gedraaid en vooruit gereden moet worden.
  \item Implementeer dit in een functie \verb'thirdLeft'.
\end{itemize}

\newexercise{Oefening 23}
\exercisemap{ex23}
\begin{itemize}
  \item Implementeer een functie \verb'fourthLeft'.
\end{itemize}


\newexercise{Oefening 24}
\exercisemap{ex24}
\begin{itemize}
  \item Oh nee, 't is weer van dat\dots
  \item Implementeer een functie \verb'secondLeft'.
  \item Het geoefend oog zal opmerken dat deze functies (\verb'secondLeft', \verb'thirdLeft', \verb'fourthLeft')
        allen verdacht veel op elkaar lijken. Dit moeten we kunnen uitbuiten om minder werk te hoeven verrichten!
        Hoe kunnen we gebruik maken van deze gelijkenissen?
  \item We willen een enkele functie \verb'nthLeft' die als \emph{parameter} meekrijgt
        de hoeveelste straat links we moeten hebben. \verb'secondLeft(car)' wordt
        dan \verb'nthLeft(car, 2)', \verb'thirdLeft(car)' wordt \verb'nthLeft(car, 3)', enz.
  \item De implementatie van deze functie ziet er zo uit:
        \code{nthleft.js}
  \item De \verb'for'-lus herhaalt de body voor \verb'i' gaande van \verb'0' en \verb'n', m.a.w.\
        de body wordt \verb'n' keer herhaald. De waarde van \verb'n' kiezen we zelf wanneer
        we de functie \verb'nthLeft' oproepen.
  \item Herschrijf de oplossingen van de drie laatste oefeningen met \verb'nthLeft'.
\end{itemize}


\newexercise{Oefening 25}
\exercisemap{ex25}
\begin{itemize}
  \item Schrijf (op basis van \verb'nthLeft') de functie \verb'nthRight'.
  \item Gebruik deze om deze oefening op te lossen.
  \item Merk op dat \verb'nthLeft' en \verb'nthRight' weeral veel op elkaar lijken.
        Hoe zou je deze twee functies kunnen samenvoegen tot een enkele?
\end{itemize}


\newexercise{Oefening 26}
\begin{itemize}
  \item Uitdaging: schrijf een programma dat werkt voor willekeurige kaarten.
  \item Dit is een moeilijke oefening: zelfs op het examen krijg je niet zo'n moeilijke opgaves.
\end{itemize}


\end{document}

\end{documentclass}