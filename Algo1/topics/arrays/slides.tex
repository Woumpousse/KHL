\documentclass{../khlslides}

\title[Arrays]{Arrays}
\author{Fr\'ed\'eric Vogels}



\newcommand{\drawarray}[1]{
  \foreach[count=\i] \val in {#1} {
    \pgfmathparse{int(\i-1)}\let\idx\pgfmathresult
    \pgfmathparse{(\i-1)*1.2}\let\x\pgfmathresult
    \node[anchor=south west,cell] (cell \i) at (\x, 0) {\val};
    \node[index] at (cell \i.north west) {\tiny\idx};
  }
}

\newcommand{\drawheterogeneousarray}[1]{
  \foreach[count=\i] \val/\cell in {#1} {
    \pgfmathparse{int(\i-1)}\let\idx\pgfmathresult
    \pgfmathparse{(\i-1)*1.2}\let\x\pgfmathresult
    \node[anchor=south west,\cell] (cell \i) at (\x, 0) {\val};
    \node[index] at (cell \i.north west) {\tiny\idx};
  }
}

\pgfkeys{/tikz/.cd,
         cell/.style={drop shadow,fill=blue!50,minimum size=.8cm},
         cell/.style={drop shadow,fill=blue!50,minimum size=.8cm},
         highlighted cell/.style={drop shadow,fill=green!50,minimum size=.8cm},
         index/.style={fill=white,draw},
         box/.style={fill=red!10,drop shadow}}


\begin{document}

\maketitle

{
\newcommand{\CODE}[1]{
  \onslide<#1>
  \begin{minipage}[c][2cm]{\linewidth}
    \code{array-basics-#1.js}
  \end{minipage}
}
\newcommand{\ARRAY}[2]{
  \only<#1>{\drawarray{#2}}
}

\begin{frame}
  \frametitle{Arrays}
  \begin{center}
    \begin{tikzpicture}
      \coordinate (lower left) at (-.5,-.5);
      \coordinate (upper right) at (6,1.5);
      \coordinate (upper left) at (lower left |- upper right);
      \coordinate (lower right) at (lower left -| upper right);

      \draw[use as bounding box,box] (lower left) rectangle (upper right);
      \node[anchor=south] at ($ (upper left) !.5! (upper right) $) {\sc \Large \color{gray} before};
      \ARRAY{2}{,,,,}
      \ARRAY{3}{5,,,,}
      \ARRAY{4}{5,4,,,}
      \ARRAY{5}{5,4,5,,}
      \ARRAY{5}{5,4,5,4,}
    \end{tikzpicture}
  \end{center}

  \begin{overprint}
    \CODE{1}
    \CODE{2}
    \CODE{3}
    \CODE{4}
    \CODE{5}
    \CODE{6}
  \end{overprint}

  \begin{center}
    \begin{tikzpicture}
      \coordinate (lower left) at (-.5,-.5);
      \coordinate (upper right) at (6,1.5);
      \coordinate (upper left) at (lower left |- upper right);
      \coordinate (lower right) at (lower left -| upper right);

      \draw[use as bounding box,box] (lower left) rectangle (upper right);
      \node[anchor=north] at ($ (lower left) !.5! (lower right) $) {\sc \Large \color{gray} after};
      \ARRAY{1}{,,,,}
      \ARRAY{2}{5,,,,}
      \ARRAY{3}{5,4,,,}
      \ARRAY{4}{5,4,5,,}
      \ARRAY{5}{5,4,5,4,}
      \ARRAY{6}{5,4,5,4,18}
    \end{tikzpicture}
  \end{center}
\end{frame}
}



{
\pgfkeys{/tikz/.cd,
         cell/.style={drop shadow,fill=blue!50,minimum size=.8cm},
         index/.style={fill=white,draw},
         box/.style={fill=red!10,drop shadow}}
\newcommand{\CODE}[1]{
  \onslide<#1>
  \begin{minipage}[c][2cm]{\linewidth}
    \code{array-advanced-#1.js}
  \end{minipage}
}
\newcommand{\ARRAY}[2]{
  \only<#1>{\drawarray{#2}}
}

\begin{frame}
  \frametitle{Array Modification}
  \begin{center}
    \begin{tikzpicture}
      \coordinate (lower left) at (-.5,-.5);
      \coordinate (upper right) at (6,1.5);
      \coordinate (upper left) at (lower left |- upper right);
      \coordinate (lower right) at (lower left -| upper right);

      \draw[use as bounding box,box] (lower left) rectangle (upper right);
      \node[anchor=south] at ($ (upper left) !.5! (upper right) $) {\sc \Large \color{gray} before};
      \ARRAY{2}{1,2,3}
      \ARRAY{3}{1,2,3,4}
      \ARRAY{4}{1,2,3}
      \ARRAY{5}{0,1,2,3}
    \end{tikzpicture}
  \end{center}

  \begin{overprint}
    \CODE{1}
    \CODE{2}
    \CODE{3}
    \CODE{4}
    \CODE{5}
  \end{overprint}

  \begin{center}
    \begin{tikzpicture}
      \coordinate (lower left) at (-.5,-.5);
      \coordinate (upper right) at (6,1.5);
      \coordinate (upper left) at (lower left |- upper right);
      \coordinate (lower right) at (lower left -| upper right);

      \draw[use as bounding box,box] (lower left) rectangle (upper right);
      \node[anchor=north] at ($ (lower left) !.5! (lower right) $) {\sc \Large \color{gray} after};
      \ARRAY{1}{1,2,3}
      \ARRAY{2}{1,2,3,4}
      \ARRAY{3}{1,2,3}
      \ARRAY{4}{0,1,2,3}
      \ARRAY{5}{1,2,3}
    \end{tikzpicture}
  \end{center}
\end{frame}
}

\begin{frame}
  \frametitle{Array Modification}
  \begin{center}
    \begin{tikzpicture}
      \drawarray{1,2,3,4,5}
      \coordinate (position extra left) at ($ (cell 1.south west) + (-1.2,0) $);
      \coordinate (position extra right) at ($ (cell 5.south west) + (1.2,0) $);
      \node[minimum size=.8cm,draw,dashed,anchor=south west] (extra left) at (position extra left) {};
      \node[minimum size=.8cm,draw,dashed,anchor=south west] (extra right) at (position extra right) {};
      \draw[->] (extra left.north) +(1,1) .. controls +(left:.5cm) and +(up:.5cm) .. +(0,0) node[midway,sloped,above] {\tiny unshift};
      \draw[->] (extra left.south) .. controls +(down:.5cm) and +(left:.5cm) .. +(1,-1) node[midway,sloped,below] {\tiny shift};
      \draw[->] (extra right.north) +(-1,1) .. controls +(right:.5cm) and +(up:.5cm) .. +(0,0) node[midway,sloped,above] {\tiny push};
      \draw[->] (extra right.south) .. controls +(down:.5cm) and +(right:.5cm) .. +(-1,-1) node[midway,sloped,below] {\tiny pop};
    \end{tikzpicture}
  \end{center}
\end{frame}

{
\pgfkeys{/tikz/.cd,
         cell/.style={drop shadow,fill=blue!50,minimum size=.8cm},
         index/.style={fill=white,draw},
         box/.style={fill=red!10,drop shadow}}
\newcommand{\CODE}[1]{
  \onslide<#1>
  \begin{minipage}[c][1cm]{\linewidth}
    \code{array-slice-#1.js}
  \end{minipage}
}
\newcommand{\ARRAY}[2]{
  \only<#1>{\drawarray{#2}}
}

\begin{frame}
  \frametitle{Arrays Slicing}
  \begin{center}
    \begin{tikzpicture}
      \coordinate (lower left) at (-.5,-.5);
      \coordinate (upper right) at (6,1.5);
      \coordinate (upper left) at (lower left |- upper right);
      \coordinate (lower right) at (lower left -| upper right);

      \draw[use as bounding box,box] (lower left) rectangle (upper right);
      \node[anchor=south] at ($ (upper left) !.5! (upper right) $) {\sc \Large \color{gray} before};
      \ARRAY{1-}{1,2,3,4,5}
      \node (xs) at ($ (cell 1.west) + (-1,0) $) {xs};
      \draw[->] (xs) to (cell 1.west);
    \end{tikzpicture}
  \end{center}

  \begin{overprint}
    \CODE{1}
    \CODE{2}
    \CODE{3}
  \end{overprint}

  \begin{center}
    \begin{tikzpicture}
      \coordinate (lower left) at (-.5,-1.5);
      \coordinate (upper right) at (6,1.5);
      \coordinate (upper left) at (lower left |- upper right);
      \coordinate (lower right) at (lower left -| upper right);

      \draw[use as bounding box,box] (lower left) rectangle (upper right);
      \node[anchor=north] at ($ (lower left) !.5! (lower right) $) {\sc \Large \color{gray} after};
      \begin{scope}[yshift=.25cm]
        \ARRAY{1-}{1,2,3,4,5}
        \node (xs) at ($ (cell 1.west) + (-1,0) $) {xs};
        \draw[->] (xs) to (cell 1.west);
      \end{scope}
      \begin{scope}[yshift=-1.25cm]
        \ARRAY{1}{2,3,4}
        \ARRAY{2}{0,1}
        \ARRAY{3}{3,4,5}
        \node (ys) at ($ (cell 1.west) + (-1,0) $) {ys};
        \draw[->] (ys) to (cell 1.west);
      \end{scope}
    \end{tikzpicture}
  \end{center}
\end{frame}
}


\begin{frame}
  \frametitle{Array Concatenation}
  \begin{center}
    \begin{tikzpicture}[scale=.9,transform shape]
      \coordinate (lower left) at (-.5,-1);
      \coordinate (upper right) at (6,1.5);
      \coordinate (upper left) at (lower left |- upper right);
      \coordinate (lower right) at (lower left -| upper right);

      \draw[use as bounding box,box] (lower left) rectangle (upper right);
      \node[anchor=south] at ($ (upper left) !.5! (upper right) $) {\sc \Large \color{gray} before};

      \begin{scope}[yshift=.25cm]
        \drawarray{1,2,3}
        \node (xs) at ($ (cell 1.west) + (-1,0) $) {xs};
        \draw[->] (xs) to (cell 1.west);
      \end{scope}
      \begin{scope}[yshift=-.75cm]
        \drawarray{4,5}
        \node (ys) at ($ (cell 1.west) + (-1,0) $) {ys};
        \draw[->] (ys) to (cell 1.west);
      \end{scope}
    \end{tikzpicture}
  \end{center}

  \begin{overprint}
    \code{concat.js}
  \end{overprint}

  \begin{center}
    \begin{tikzpicture}[scale=.9,transform shape]
      \coordinate (lower left) at (-.5,-2.5);
      \coordinate (upper right) at (6,1.5);
      \coordinate (upper left) at (lower left |- upper right);
      \coordinate (lower right) at (lower left -| upper right);

      \draw[use as bounding box,box] (lower left) rectangle (upper right);
      \node[anchor=north] at ($ (lower left) !.5! (lower right) $) {\sc \Large \color{gray} after};
      \begin{scope}[yshift=.25cm]
        \drawarray{1,2,3}
        \node (xs) at ($ (cell 1.west) + (-1,0) $) {xs};
        \draw[->] (xs) to (cell 1.west);
      \end{scope}
      \begin{scope}[yshift=-1cm]
        \drawarray{4,5}
        \node (ys) at ($ (cell 1.west) + (-1,0) $) {ys};
        \draw[->] (ys) to (cell 1.west);
      \end{scope}
      \begin{scope}[yshift=-2.25cm]
        \drawarray{1,2,3,4,5}
        \node (zs) at ($ (cell 1.west) + (-1,0) $) {zs};
        \draw[->] (zs) to (cell 1.west);
      \end{scope}
    \end{tikzpicture}
  \end{center}
\end{frame}



\end{document}



%%% Local Variables: 
%%% mode: latex
%%% TeX-master: t
%%% End: 
