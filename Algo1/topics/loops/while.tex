\begin{frame}
  \frametitle{Lussen in JavaScript}
  \begin{columns}
    \column{.5\linewidth}
    \begin{center}
      \begin{tikzpicture}
        \node[flowchart/node] (while) {evalueren \PLACEHOLDER{conditie}};
        \node (loop entry) at ($ (while.north) + (0,0.5) $) {};
        \node[flowchart/node] (body) at ($ (while.south) + (2,-2) $) {\parbox{1.5cm}{\centering uitvoeren \PLACEHOLDER{body}}};
        \node (split) at ($ (while.south) + (0,-0.5) $) {};

        \draw[flowchart/arrow] ($ (while.north) + (0,1) $) -- (while.north);
        \draw[flowchart/arrow] (while.south) -- (split.center) -- ++(-2,-0.5) -- ++(0,-2.5);
        \draw[flowchart/arrow] (split.center) -- ++(2,-0.5) -- (body.north);
        \draw[flowchart/arrowline] (body.south) -- ++(0,-0.5) -- ++(1.5,0) |- (loop entry.center);

        \path ($ (while.south) + (0,-0.4) $) -- ++(-2,-0.5) node [midway,above,sloped] {false};
        \path ($ (while.south) + (0,-0.4) $) -- ++(2,-0.5) node [midway,above,sloped] {true};
      \end{tikzpicture}
    \end{center}

    \column{.5\linewidth}
    \code{while.js}
  \end{columns}
\end{frame}

\begin{frame}
  \frametitle{Oefening}
  \code[width=.4\linewidth]{exwhile.js}

  \begin{center}
    \begin{tikzpicture}
      \coordinate (sp x init) at (0,0);
      \coordinate (sp i init) at ($ (sp x init) + (1.5,0) $);
      \coordinate (sp i check) at ($ (sp i init) + (1.5,0) $);
      \coordinate (sp x mul 2) at ($ (sp i check) + (1.5,-1) $);
      \coordinate (sp i dec) at ($ (sp x mul 2) + (1.5,0) $);
      \coordinate (sp exit) at ($ (sp i check) + (1.5,0) $);

      \visible<2->{
        \draw[sequence link] (sp x init) -- (sp i init);
        \draw[sequence link] (sp i init) -- (sp i check);
        \draw[sequence link] (sp i check) -- (sp x mul 2) node[midway,black,font=\tiny,sloped,yshift=1mm] {\tt true};
        \draw[sequence link] (sp x mul 2) -- (sp i dec);
        \draw[sequence link,-latex] (sp i dec) -- ++(0,-.5) -| (sp i check);
        \draw[sequence link] (sp i check) -- (sp exit) node[midway,black,font=\tiny,sloped,yshift=1mm] {\tt false};

        \draw[sequence point] (sp x init) circle (.05) node[above,font=\tiny,black] {\tt var x=1};
        \draw[sequence point] (sp i init) circle (.05) node[above,font=\tiny,black] {\tt var i = 5};
        \draw[sequence point] (sp i check) circle (.05) node[above,font=\tiny,black] {\tt i > 0};
        \draw[sequence point] (sp x mul 2) circle (.05) node[above right,font=\tiny,black] {\tt x *= 2};
        \draw[sequence point] (sp i dec) circle (.05) node[above,font=\tiny,black] {\tt i--};
        \draw[sequence point] (sp exit) circle (.05);

        \node[anchor=north west,font=\small] at ($ (sp x init) + (0,-1) $) {
          \begin{tabular}{c@{\hspace{1mm}}c@{\hspace{1mm}}l}
            x & = & \tt
            \only<2-4>{1}%
            \only<5-7>{2}%
            \only<8-10>{4}%
            \only<11-13>{8}%
            \only<14-16>{16}%
            \only<17->{32}%
            \\
            \only<3->{i} & \only<3->{=} & \tt
            \only<3-5>{5}%
            \only<6-8>{4}%
            \only<9-11>{3}%
            \only<12-14>{2}%
            \only<15-17>{1}%
            \only<18->{0}%
            \\
          \end{tabular}
        };

        \only<2>{
          \draw[active] (sp x init) circle (.05);
        }

        \only<3>{
          \draw[active] (sp i init) circle (.05);
        }

        \only<4,7,10,13,16,19>{
          \draw[active] (sp i check) circle (.05);
        }

        \only<5,8,11,14,17>{
          \draw[active] (sp x mul 2) circle (.05);
        }

        \only<6,9,12,15,18>{
          \draw[active] (sp i dec) circle (.05);
        }

        \only<20>{
          \draw[active] (sp exit) circle (.05);
        }
      }
    \end{tikzpicture}
  \end{center}
\end{frame}


%%% Local Variables: 
%%% mode: latex
%%% TeX-master: "loops"
%%% End: 
