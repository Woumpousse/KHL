\begin{frame}
  \frametitle{BMI Berekening: Inputs en Outputs}
  \begin{center}
    \begin{tikzpicture}
      \path[use as bounding box] (-1,-1) rectangle (1,1);
      \node[minimum width=4cm,minimum height=1cm,fill=green!25,drop shadow] (bmi) {\tt berekenBMI()};
      \draw[-latex] ($ (bmi.west) ! .5 ! (bmi.north west) $) +(-2,0) -- +(0,0)
                    node[midway,above,font=\small] {lengte};
      \draw[-latex] ($ (bmi.west) ! .5 ! (bmi.south west) $) +(-2,0) -- +(0,0)
                    node[midway,below,font=\small] {gewicht};
      \draw[-latex] (bmi.east) -- +(2,0)
                    node[midway,below,font=\small] {bmi};
    \end{tikzpicture}
    \vskip5mm
    \visible<2>{We focussen op de inputs}
  \end{center}
\end{frame}

\begin{frame}
  \frametitle{Parameters}
  \begin{itemize}
    \item Functies kunnen \emph{parameters} hebben
    \item Vermelden expliciet welke inputs een functie nodig heeft
    \item BMI parameters: gewicht en lengte
  \end{itemize}
  \code[font size=\small]{bmi-params.js}
\end{frame}

{
  \newcommand{\CODE}[9]{
    \def\paramx{#1}
    \def\paramy{#2}
    \def\paramz{#3}
    \def\localx{#4}
    \def\localy{#5}
    \def\localz{#6}
    \def\argx{#7}
    \def\argy{#8}
    \def\argz{#9}
    \code{params.js}
  }
  \newcommand{\values}[4]{
    \node[anchor=north east,draw] (bar val) at ($ (bar.base) + (0,-.2) $) {\tt\tiny #1};
    \node[anchor=north east,draw] (x val) at ($ (x.base) + (0,-.2) $) {\tt\tiny #2};
    \node[anchor=north east,draw] (y val) at ($ (y.base) + (0,-.2) $) {\tt\tiny #3};
    \node[anchor=north east,draw] (z val) at ($ (z.base) + (0,-.2) $) {\tt\tiny #4};
    \draw (bar val) -- (bar);
    \draw (x val) -- (x);
    \draw (y val) -- (y);
    \draw (z val) -- (z);
  }
\begin{frame}
  \frametitle{Parameters}
  \begin{overprint}
    \onslide<1-3>
    \only<1-3>{{\CODE xyzabcabc}}

    \onslide<4-5>
    \only<4-5>{{\CODE xyzabcbac}}

    \onslide<6>
    \only<6>{{\CODE uvwabcbac}}

    \onslide<7>
    \only<7>{{\CODE rstabcbac}}

    \onslide<8>
    \only<8>{{\CODE abcabcbac}}
  \end{overprint}  
  \begin{tikzpicture}[overlay,remember picture]
    \only<2>{
      \draw[-latex] (argx) -- ++(0,.6) -| (paramx) node[pos=.75,yshift=1mm,sloped,font=\tiny] {kopie};
      \draw[-latex] (argy) -- ++(0,.5) -| (paramy) node[pos=.75,yshift=1mm,sloped,font=\tiny] {kopie};
      \draw[-latex] (argz) -- ++(0,.4) -| (paramz) node[pos=.75,yshift=1mm,sloped,font=\tiny] {kopie};
    }
    \only<3>{
      \values 7123
    }
    \only<5-8>{
      \values 5213
    }
  \end{tikzpicture}
  \begin{overprint}
    \onslide<2>
    \begin{center}
      De \emph{argumenten} {\tt a}, {\tt b} en {\tt c} worden gekopieerd en doorgegeven aan de functie
    \end{center}

    \onslide<3>
    \begin{center}
      De \emph{parameters} {\tt x}, {\tt y} en {\tt z} bevatten respectievelijk
      de waarden {\tt 1},~{\tt 2}~en~{\tt 3}
    \end{center}

    \onslide<4>
    \begin{center}
      We veranderen de volgorde van de argumenten
    \end{center}

    \onslide<5>
    \begin{center}
      Nu hebben de parameters andere waarden
    \end{center}

    \onslide<6-7>
    \begin{center}
      De namen van de parameters maken niets uit
    \end{center}

    \onslide<8>
    \begin{center}
      Parameters horen bij de functie, het zijn \emph{lokale variabelen}.
      Ze mogen dezelfde naam hebben als andere reeds bestaande variabelen.
    \end{center}
  \end{overprint}
\end{frame}
}

\begin{frame}
  \frametitle{Oefening}
  \code[width=.6\linewidth]{exparams.js}
  \begin{center}
     Wat is de waarde van {\tt result}? \visible<2>{10}
  \end{center}
\end{frame}

\begin{frame}
  \frametitle{Oefening}
  \code[width=.6\linewidth]{exparams2.js}
  \begin{center}
    Wat is het effect van de {\tt foo}-oproep? \\
    \visible<2>{
      Een functie krijgt \emph{kopies} mee van de argumenten: wijzigingen aan parameters hebben dus geen effect buiten de functie
    }
  \end{center}
\end{frame}

%%% Local Variables: 
%%% mode: latex
%%% TeX-master: "functions"
%%% End: 
