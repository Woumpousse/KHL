\begin{frame}
  \frametitle{Link met Java}
  \begin{center}
    \begin{tikzpicture}[remember picture,header/.style={minimum width=4cm,fill=blue!50}]
      \path[use as bounding box] (-5,0) rectangle (5,5);

      \node[header] at (-3, 5) {\sc\Large javascript};
      \node[header] at (3, 5) {\sc\Large java};

      \only<1-2>{
        \node at (-3,4) {Array met waarden};
        \node at (3,4) {Velden};

        \begin{scope}[xshift=-3.5cm,yshift=2cm]
          \drawvarray{h,m,s}
        \end{scope}

        \path let \p1=(cell 1.north) in
              node[anchor=north] at (3,\y1) {
                \inlinecode[width=4cm,language=myjava,font size=\small]{fields.java}
              };
      }

      \only<2>{
        \draw[-latex] (cell 1) -- (field h.west);
        \draw[-latex] (cell 2) -- (field m.west);
        \draw[-latex] (cell 3) -- (field s.west);
      }

      \only<3-4>{
        \node at (-3,4) {\tt maakTijdstip};
        \node at (3,4) {Constructor};

        \node[anchor=north] at (-3,3.5) {
          \inlinecode[language=javascript,font size=\tiny,width=4cm]{makeTime.js}
        };

        \node[anchor=north] at (3,3.5) {
          \inlinecode[language=java,font size=\tiny,width=4cm]{constructor.java}
        };
      }

      \only<4>{
        \draw[-latex] (init h) |- (set hour);
        \draw[-latex] (init m) |- (set minutes);
        \draw[-latex] (init s) |- (set seconds);
      }

      \only<5-6>{
        \node at (-3,4) {Basisoperaties};
        \node at (3,4) {Getters/setters};

        \node[anchor=north] at (-3,3.5) {
          \inlinecode[language=javascript,font size=\tiny,width=4cm]{getters.js}
        };

        \node[anchor=north] at (3,3.5) {
          \inlinecode[language=java,font size=\tiny,width=4cm]{getters.java}
        };
      }

      \only<6>{
        \draw[-latex] (get h) -- (get h 2);
        \draw[-latex] (get m) -- (get m 2);
        \draw[-latex] (get s) -- (get s 2);
      }

      \only<7-8>{
        \node at (-3,4) {Ontvangend object};
        \node at (3,4) {{\tt this}-variabele};

        \node[anchor=north] at (-3,3.5) {
          \inlinecode[language=javascript,font size=\tiny,width=4cm]{getUur.js}
        };

        \node[anchor=north] at (3,3.5) {
          \inlinecode[language=java,font size=\tiny,width=4cm]{getUur.java}
        };
      }

      \only<8>{
        \draw[-latex] (arg t) -- (this);
      }
    \end{tikzpicture}
  \end{center}
\end{frame}

\begin{frame}
  \frametitle{Link met Java}
  \structure{Access modifiers}
  \begin{itemize}
    \item Java biedt {\tt public} en {\tt private}
    \item Java dwingt af externe code niet aan de interne voorstelling komt
    \item JavaScript kan dit ook wel, zij het anders en complexer      
  \end{itemize}

  \vskip5mm

  \structure{Types}
  \begin{itemize}
    \item Een Java-klasse vormt een ``etiket'': het type
    \item Twee klassen met identieke velden kunnen dus onderscheiden worden dankzij het type
  \end{itemize}
\end{frame}


%%% Local Variables: 
%%% mode: latex
%%% TeX-master: "adt"
%%% End: 
