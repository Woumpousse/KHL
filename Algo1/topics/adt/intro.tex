\begin{frame}
  \frametitle{Tijdstip}
  \structure{Probleemstelling} \\
  Een \emph{tijdstip} wordt gekenmerkt door
  \begin{itemize}
    \item Uur: een getal tussen 0 en 23
    \item Minuten: een getal tussen 0 en 59
    \item Seconden: een getal tussen 0 en 59
  \end{itemize}
  \vskip4mm
  \structure{Vraag} \\
  Hoe stellen we een tijdstip voor in JavaScript?
\end{frame}

\begin{frame}
  \frametitle{Tijdstip}
  \begin{center}
    \begin{tikzpicture}
      \drawarray{14,33,52}
      \node (tijdstip) at (-2,1) {tijdstip};
      \draw[->] (tijdstip) to [bend right=30] (cell 1);
    \end{tikzpicture}
  \end{center}
  \vskip5mm
  \code[width=.55\linewidth]{time.js}
\end{frame}


\begin{frame}
  \frametitle{Datum}
  \structure{Probleemstelling} \\
  Een \emph{datum} wordt gekenmerkt door
  \begin{itemize}
    \item Dag: een getal tussen 1 en 31
    \item Maand: een getal tussen 1 en 12
    \item Jaar: een getal
  \end{itemize}
  \vskip4mm
  \structure{Vraag} \\
  Hoe stellen we een datum voor in JavaScript?
\end{frame}

\begin{frame}
  \frametitle{Datum}
  \begin{center}
    \begin{tikzpicture}
      \drawarray{18,12,1980}
      \node (datum) at (-2,1) {datum};
      \draw[->] (datum) to [bend right=30] (cell 1);
    \end{tikzpicture}
  \end{center}
  \vskip5mm
  \code[width=.6\linewidth]{date.js}
\end{frame}


\begin{frame}
  \frametitle{Tijdsinterval}
  \structure{Probleemstelling} \\
  Een \emph{tijdsinterval} wordt gekenmerkt door
  \begin{itemize}
    \item Dag: een datum
    \item Beginuur: een tijdstip
    \item Einduur: een tijdstip
  \end{itemize}
  \vskip4mm
  \structure{Vraag} \\
  Hoe stellen we een datum tijdsinterval voor in JavaScript?
\end{frame}

\begin{frame}
  \frametitle{Tijdsinterval}
  \begin{center}
    \begin{tikzpicture}
      \begin{scope}
        \drawarray[cell name=date]{26,9,2014}
      \end{scope}

      \begin{scope}[yshift=-2cm]
        \drawarray[cell name=start]{9,0,0}
      \end{scope}

      \begin{scope}[yshift=-4cm]
        \drawarray[cell name=end]{10,30,0}
      \end{scope}

      \begin{scope}[xshift=-3cm,yshift=-1cm]
        \drawvarray[cell name=interval]{,,}
      \end{scope}

      \draw[->] (interval 1.center) -- (date 1);
      \draw[->] (interval 2.center) -- (start 1);
      \draw[->] (interval 3.center) -- (end 1);

      \node (var) at (-5,0) {interval};
      \draw[->] (var) to [bend right=30] (interval 1);
    \end{tikzpicture}
  \end{center}
  \vskip5mm
  \code[width=.65\linewidth]{interval.js}
\end{frame}

\begin{frame}
  \frametitle{Les}
  \structure{Probleemstelling} \\
  Een \emph{les} wordt gekenmerkt door
  \begin{itemize}
    \item Een vak: een string
    \item Een lokaal: een string
    \item Een tijdsinterval
  \end{itemize}
  \vskip4mm
  \structure{Vraag} \\
  Hoe stellen we een les voor in JavaScript?
\end{frame}


\begin{frame}
  \frametitle{Les}
  \code[width=.8\linewidth]{lesson.js}
\end{frame}


\begin{frame}
  \frametitle{Student}
  \structure{Probleemstelling} \\
  Een \emph{student} wordt gekenmerkt door
  \begin{itemize}
    \item Een naam: een string
    \item Een r-nummer: een string
    \item Een uurrooster: een lijst lessen
  \end{itemize}
  \vskip4mm
  \structure{Vraag} \\
  Hoe stellen we een student voor in JavaScript?
\end{frame}

\begin{frame}
  \frametitle{Student}
  \code[width=.8\linewidth]{student.js}
\end{frame}


\begin{frame}
  \frametitle{Enzovoort}
  Een \emph{reeks} wordt gekenmerkt door
  \begin{itemize}
    \item Een naam: een string
    \item Een lijst studenten
  \end{itemize}

  \vskip5mm

  Een \emph{studierichting} wordt gekenmerkt door
  \begin{itemize}
    \item Een naam: een string
    \item Een lijst reeksen
  \end{itemize}

  \vskip5mm

  Een \emph{hogeschool} wordt gekenmerkt door
  \begin{itemize}
    \item Een naam: een string
    \item Een lijst studierichtingen
  \end{itemize}

  \vskip5mm

  Een \emph{associatie} wordt gekenmerkt door \dots
\end{frame}


\begin{frame}
  \frametitle{Zoek op!}
  Hoe lang duurt de eerste les van de 14de student van reeks 5
  in de 8ste studierichting van deze hogeschool?

  \vskip10mm

  \visible<2>{
    \code[width=.9\linewidth]{select.js}
  }
\end{frame}


\begin{frame}
  \frametitle{Problemen}
  \begin{center} 
    {\Huge Probleem \#1} \\[4mm]
    {\Large Verwarrende indices}
  \end{center}
  \begin{itemize}
    \item Onleesbaar: geen duidelijke informatie
    \item Zeer gemakkelijk om fouten te maken
    \item Werken met namen zou gemakkelijker zijn
  \end{itemize}
\end{frame}


\begin{frame}
  \frametitle{Dubbelzinnige Datastructuren}
  Beschouw de array
  \begin{center}
    \begin{tikzpicture}
      \drawarray{12,3,52}
    \end{tikzpicture}
  \end{center}
  Wat is dit?
  \begin{itemize}
    \item Is het de datum 12 maart 52?
    \item Is het het tijdstip 12:3:52?
    \item Zijn het 3D-co\"ordinaten $(12,3,52)$?
    \item Is het gewoon een lijst getallen?
    \item Of nog iets anders?
  \end{itemize}
\end{frame}


\begin{frame}
  \frametitle{Problemen}
  \begin{center} 
    {\Huge Probleem \#2} \\[4mm]
    {\Large Dubbelzinnige data}
  \end{center}
  \begin{itemize}
    \item Dezelfde data kan meerdere dingen voorstellen
    \item Computer maakt het verschil niet
    \item Zou het vectorieel product van datums uitrekenen
    \item We willen \emph{metadata}
          \begin{itemize}
            \item Een etiket op data
            \item ``Dit is een datum'' of ``Dit is een tijdstip''
          \end{itemize}
  \end{itemize}
\end{frame}



\begin{frame}
  \frametitle{Fragiliteit}
  \structure{Probleem} \\
  We willen van elke hogeschool ook het adres bijhouden.

  \vskip4mm
  \begin{center}
    \begin{minipage}{.7\linewidth}
      Een \emph{hogeschool} wordt gekenmerkt door
      \begin{itemize}
        \item Een naam: een string
        \item \alert{Een adres}
        \item Een lijst studierichtingen
      \end{itemize}
    \end{minipage}
  \end{center}

  \vskip4mm
  \structure{Gevolgen}
  \begin{itemize}
    \item Code moet overal aangepast worden
    \item Kan niet automatisch gebeuren
  \end{itemize}
\end{frame}



\begin{frame}
  \frametitle{Problemen}
  \begin{center} 
    {\Huge Probleem \#3} \\[4mm]
    {\Large Fragiel programma}
  \end{center}
  \begin{itemize}
    \item Gebrek aan modulariteit
          \begin{itemize}
            \item Kleine wijzingen hebben grote gevolgen
            \item Beter: wijzigingen leiden altijd kleine gevolgen
          \end{itemize}
    \item Mogelijke wijzigingen
          \begin{itemize}
            \item Andere voorstelling (datum omkeren naar jaar, maand, dag)
            \item Uitbreiden (adres toevoegen aan hogeschool)
          \end{itemize}
  \end{itemize}
\end{frame}


\begin{frame}
  \frametitle{Stand van Zaken}
  \structure{Problemen}
  \begin{itemize}
    \item Leesbaarheid
    \item Metadata
    \item Modulariteit
  \end{itemize}
  \vskip4mm
  \structure{Oplossing?}
  \begin{itemize}
    \visible<2>{\item ADTs: Abstract Data Types}
  \end{itemize}
\end{frame}


%%% Local Variables: 
%%% mode: latex
%%% TeX-master: "adt"
%%% End: 
