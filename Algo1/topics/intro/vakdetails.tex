\section{Introductie}

\frame{ \tableofcontents[currentsection] }

\begin{frame}
  \frametitle{Studiebelasting}
  \begin{columns}
    \column{.5\textwidth}
    \begin{center}
      \begin{tikzpicture}
        \pgfkeys{/mylib/segpie/.cd, inner radius=1.5, outer radius=2.5}
        \only<1>{
          \segpie[fill=white]{0}{74}{74}
        }
        \only<2>{
          \segpie[fill=red]{0}{11}{74}
          \segpie[fill=white]{12}{74}{74}
        }
        \only<3>{
          \segpie[fill=red]{0}{11}{74}
          \segpie[fill=blue]{12}{29}{74}
          \segpie[fill=white]{30}{74}{74}
        }
        \only<4>{
          \segpie[fill=red]{0}{11}{74}
          \segpie[fill=blue]{12}{29}{74}
          \segpie[fill=green]{30}{74}{74}          
        }
      \end{tikzpicture}
    \end{center}
    \column{.5\textwidth}
      \begin{itemize}
        \item<1-> 3 studiepunten
        \item<1-> 25 uur per studiepunt
        \item<1-> 12 weken
        \item<1-> $\Rightarrow$ 75 uren in totaal
        \item<2-> {\color{red} 1 uur/week theorie}
        \item<3-> {\color{blue} 1.5 uur/week oefeningen}
        \item<4-> {\color{green} 45 uur zelfstudie}
      \end{itemize}
  \end{columns}
\end{frame}

% \begin{frame}
%   \frametitle{Puntenverdeling}
%   \begin{columns}
%     \column{.5\textwidth}
%     \begin{itemize}
%       \item {\color{red} 40\%: opdrachten}
%              \begin{itemize}
%                \item {\color{red} Toetsen}
%              \end{itemize}
%              \vskip4mm
%       \item {\color{blue} 60\%: examen}
%              \begin{itemize}
%                \item {\color{blue} 10\% mondeling contactexamen}
%                \item {\color{blue} 50\% schriftelijk contactexamen}
%              \end{itemize}
%     \end{itemize}
%     \column{.5\textwidth}
%     \begin{center}
%       \begin{tikzpicture}
%         \pgfkeys{/mylib/segpie/.cd, inner radius=1.5, outer radius=2.5}
%         \segpie[fill=red]{0}{3}{10}
%         \segpie[fill=blue]{4}{9}{10}
%       \end{tikzpicture}
%     \end{center}
%   \end{columns}
%   \vskip5mm
%   {\tiny (Offici\"ele informatie te vinden op studiewijzer)}
% \end{frame}

\begin{frame}
  \frametitle{Plaatsing}
  \begin{center}
    \begin{tikzpicture}[vak/.style={rectangle,draw=black,fill=blue!25,thin,minimum size=12mm},
                        arr/.style={->,thick},
                        sem/.style={left color=blue!75, right color=blue!25,drop shadow},
                        semc/.style={rotate=90,fill=blue!75,rectangle,drop shadow},
                        lang/.style={circle,fill=red!50,circular drop shadow},
                        langarr/.style={ultra thick,->,red!50}]
      \path[use as bounding box] (-3,0) rectangle (3,5);
      \node[semc] at (-2.25,3.75) {S1};
      \node[semc] at (-2.25,1.75) {S2};
      \node[semc] at (-2.25,-0.25) {S3};
      \shade[sem] (-2,4.75) rectangle (2, 3.25);
      \shade[sem] (-2,2.75) rectangle (2, 1.25);
      \shade[sem] (-2,0.75) rectangle (2, -0.75);
      \node[vak] (bop)    at (-1,4)   {BOP};
      \node[vak] (algo1)  at (1,4)    {Algo1};
      \node[vak] (oop)    at (-1,2)   {OOP};
      \node[vak] (algo2)  at (1,2)    {Algo2};
      \node[vak] (ooo)    at (-1,0)   {OOO};
      \draw[arr] (algo1) -- (bop);
      \draw[arr] (bop) -- (oop);
      \draw[arr] (oop) -- (ooo);
      \draw[arr] (algo1) -- (algo2);
      \draw[arr] (bop) -- (algo2);

      \only<2>{
        \node[lang] (java) at (-4,2.5) {Java};
        \node[lang] (javascript) at (3.75,4) {JavaScript};
        \draw[langarr] (java) to [bend left=45] (bop);
        \draw[langarr] (java) to [bend left=45] (algo2);
        \draw[langarr] (java) to [bend left=45] (oop);
        \draw[langarr] (java) to [bend left=45] (ooo);
        \draw[langarr] (javascript) to [bend right=45] (algo1);
      }
    \end{tikzpicture}
  \end{center}
\end{frame}

%%% Local Variables: 
%%% mode: latex
%%% TeX-master: "intro"
%%% End: 
