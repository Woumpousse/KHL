\documentclass{../khlslides}


\title[Recursie]{Recursie}
\author{Fr\'ed\'eric Vogels}


\pgfkeys{/tikz/flowchart/node/.style={rectangle,fill=blue,opacity=0.5,text opacity=1,drop shadow,inner sep=2mm}}
\pgfkeys{/tikz/flowchart/arrow/.style={-latex,flowchart/arrowline}}
\pgfkeys{/tikz/flowchart/arrowline/.style={thick}}


\newcommand{\lcurly}{{\tt{\char '173}}}
\newcommand{\link}[2]{\href{#1}{\beamergotobutton{#2}}}
\newcommand{\PLACEHOLDER}[1]{\ensuremath{\langle}\textrm{\textit{#1}\ensuremath{\rangle}}}


\pgfkeys{
  /wof/sequence point/.cd,
  placement/.initial=above,
  rotation/.initial=0,
  /wof/sequence link/.cd,
  label/.initial={},
  /wof/round/.cd,
  name prefix/.initial=,
  /tikz/.cd,
  sequence point/.style={blue!50,thick,fill=white},
  sequence point label/.style={font=\tiny,black},
  sequence link/.style={blue!50,thick},
  active/.style={fill=red,thick,red}
}
\newcommand<>{\seqpoint}[3][]{
  \only#4{{
  \pgfkeys{/wof/sequence point/.cd,#1}
  \pgfkeys{/wof/sequence point/placement/.get=\placement}
  \pgfkeys{/wof/sequence point/rotation/.get=\rotation}
  \draw[sequence point] (#2) circle (.05)
                        node[sequence point label,\placement,rotate=\rotation] {#3};
}}}
\newcommand<>{\seqlink}[3][]{\only#4{{
  \pgfkeys{/wof/sequence link/.cd,#1}
  \pgfkeys{/wof/sequence link/label/.get=\seqlinklabel}
  \draw[sequence link] (#2) -- (#3) node[midway,sloped,yshift=1mm,font=\tiny,black] {\seqlinklabel};
}}}


\newcommand<>{\dude}[1]{
  \only#2{
    \draw #1 -- ++(0.1,0.2) -- +(0.1,-0.2) +(0,0) -- ++(0,.2) -- +(-.1,0) +(0,0) -- +(.1,0) +(0,0) -- ++(0,0.1) ++(0,0.1) circle (.1);
  }
}

\begin{document}

\begin{frame}
  \titlepage
\end{frame}

\begin{frame}[plain]
  \begin{tikzpicture}[overlay,remember picture]
    \node at (current page.center) {\includegraphics[width=\paperwidth]{watals.jpg}};
  \end{tikzpicture}
\end{frame}

\begin{frame}
  \frametitle{Wat als\dots}
  \begin{itemize}
    \item Je krijgt een taak toegewezen
    \item Je bent lui
    \item Je kan jezelf klonen
    \item Je kan de taak \emph{deels} delegeren aan je kloon
          \begin{itemize}
            \item Je kloon zou het niet appreci\"eren zou hij al je werk voor jou moeten doen (je kloon is immers exact even lui als jij)
            \item Je moet minstens zelf \emph{iets} doen van je taak, hoe weinig ook
          \end{itemize}
  \end{itemize}
\end{frame}

\begin{frame}
  \frametitle{Taak \#1}
  \begin{center}
    Sommeer alle getallen van 1 t.e.m.\ 10
  \end{center}
  \begin{center}
    \begin{tikzpicture}
      \path[use as bounding box] (0,0) rectangle (11,2);

      \foreach[count=\i] \c in {10,9,...,1} {
        \node[inner sep=0mm] (node \c) at (\i, 1) {\c};
      }

      \foreach \c in {10,9,...,2} {
        \pgfmathparse{int(\c - 1)}\let\d\pgfmathresult
        \node (add \c) at ($ (node \c) ! .5 ! (node \d) $) {+};
      }

      \coordinate (clone 10) at ($ (node 10.north west) + (0,.3) $);
      \coordinate (clone 9) at ($ (node 9.north west) + (0,.3) $);
      \coordinate (clone 8) at ($ (node 8.north west) + (0,.3) $);
      \coordinate (clone 7) at ($ (node 7.north west) + (0,.3) $);
      \coordinate (clone 6) at ($ (node 6.north west) + (0,.3) $);
      \coordinate (clone 5) at ($ (node 5.north west) + (0,.3) $);
      \coordinate (clone 4) at ($ (node 4.north west) + (0,.3) $);
      \coordinate (clone 3) at ($ (node 3.north west) + (0,.3) $);
      \coordinate (clone 2) at ($ (node 2.north west) + (0,.3) $);
      \coordinate (clone 1) at ($ (node 1.north west) + (0,.3) $);


      \only<2>{
        \dude{($ (clone 10) $)}
        \draw[|-|] ($ (node 10.north west) + (0,0.1) $) -- ($ (node 1.north east) + (0,0.1) $);
      }

      \only<3>{
        \coordinate (clone 9) at ($ (node 9.north west) + (0,.3) $);
        \dude{(clone 10)}
        \dude{(clone 9)}
        \draw[|-|] ($ (node 10.north west) + (0,0.1) $) -- ($ (node 9.north west) + (0,0.1) $);
        \draw[|-|] ($ (node 9.north west) + (0,0.1) $) -- ($ (node 1.north east) + (0,0.1) $);

        \node[/khl/note,anchor=north west] (note clone 9) at ($ (node 9.north west) + (0,-1) $) {
          \parbox{6cm}{Je laat je kloon sommeren van 1 t.e.m.\ 9, en je voegt zelf nog 10 toe}
        };
        \draw[/khl/note arrow] (note clone 9) -- ($ (clone 9) + (.2,0) $);
      }

      \only<4>{
        \dude{(clone 10)}
        \dude{(clone 9)}
        \dude{(clone 8)}
        \draw[|-|] ($ (node 10.north west) + (0,0.1) $) -- ($ (node 9.north west) + (0,0.1) $);
        \draw[|-|] ($ (node 9.north west) + (0,0.1) $) -- ($ (node 8.north west) + (0,0.1) $);
        \draw[|-|] ($ (node 8.north west) + (0,0.1) $) -- ($ (node 1.north east) + (0,0.1) $);

        \node[/khl/note,anchor=north west] (note clone 8) at ($ (node 9.north west) + (0,-1) $) {
          \parbox{6cm}{Je kloon is even lui als jij, dus doet hetzelfde}
        };
        \draw[/khl/note arrow] (note clone 8) -- ($ (clone 8) + (.2,0) $);
      }

      \only<5>{
        \dude{(clone 10)}
        \foreach \c in {7,...,9} {
          \dude{(clone \c)}
          \pgfmathparse{int(\c + 1)}\let\d\pgfmathresult
          \draw[|-|] ($ (node \d.north west) + (0,0.1) $) -- ($ (node \c.north west) + (0,0.1) $);
        }

        \draw[|-|] ($ (node 7.north west) + (0,0.1) $) -- ($ (node 1.north east) + (0,0.1) $);
      }

      \only<6>{
        \dude{(clone 10)}
        \foreach \c in {6,...,9} {
          \dude{(clone \c)}
          \pgfmathparse{int(\c + 1)}\let\d\pgfmathresult
          \draw[|-|] ($ (node \d.north west) + (0,0.1) $) -- ($ (node \c.north west) + (0,0.1) $);
        }

        \draw[|-|] ($ (node 6.north west) + (0,0.1) $) -- ($ (node 1.north east) + (0,0.1) $);
      }

      \only<7>{
        \dude{(clone 10)}
        \foreach \c in {5,...,9} {
          \dude{(clone \c)}
          \pgfmathparse{int(\c + 1)}\let\d\pgfmathresult
          \draw[|-|] ($ (node \d.north west) + (0,0.1) $) -- ($ (node \c.north west) + (0,0.1) $);
        }

        \draw[|-|] ($ (node 5.north west) + (0,0.1) $) -- ($ (node 1.north east) + (0,0.1) $);
      }

      \only<8>{
        \dude{(clone 10)}
        \foreach \c in {4,...,9} {
          \dude{(clone \c)}
          \pgfmathparse{int(\c + 1)}\let\d\pgfmathresult
          \draw[|-|] ($ (node \d.north west) + (0,0.1) $) -- ($ (node \c.north west) + (0,0.1) $);
        }

        \draw[|-|] ($ (node 4.north west) + (0,0.1) $) -- ($ (node 1.north east) + (0,0.1) $);
      }

      \only<9>{
        \dude{(clone 10)}
        \foreach \c in {3,...,9} {
          \dude{(clone \c)}
          \pgfmathparse{int(\c + 1)}\let\d\pgfmathresult
          \draw[|-|] ($ (node \d.north west) + (0,0.1) $) -- ($ (node \c.north west) + (0,0.1) $);
        }

        \draw[|-|] ($ (node 3.north west) + (0,0.1) $) -- ($ (node 1.north east) + (0,0.1) $);
      }

      \only<10>{
        \dude{(clone 10)}
        \foreach \c in {2,...,9} {
          \dude{(clone \c)}
          \pgfmathparse{int(\c + 1)}\let\d\pgfmathresult
          \draw[|-|] ($ (node \d.north west) + (0,0.1) $) -- ($ (node \c.north west) + (0,0.1) $);
        }

        \draw[|-|] ($ (node 2.north west) + (0,0.1) $) -- ($ (node 1.north east) + (0,0.1) $);
      }

      \only<11->{
        \dude{(clone 10)}
        \foreach \c in {1,...,9} {
          \dude{(clone \c)}
          \pgfmathparse{int(\c + 1)}\let\d\pgfmathresult
          \draw[|-|] ($ (node \d.north west) + (0,0.1) $) -- ($ (node \c.north west) + (0,0.1) $);
        }
        \draw[|-|] ($ (node 1.north west) + (0,0.1) $) -- ($ (node 1.north east) + (0,0.1) $);
      }

      \only<11>{
        \node[/khl/note,anchor=north east] (note clone 1) at ($ (node 1.north west) + (0,-1) $) {
          \parbox{6cm}{\raggedright Deze kloon moet 1 t.e.m. 1 optellen. Dit is eenvoudig genoeg dat hij het zelf wel doet. Klonen is meer werk dan dat.}
        };
        \draw[/khl/note arrow] (note clone 1) -- ($ (clone 1) + (.2,0) $);
      }

      \only<12>{
        \draw[-latex] ($ (clone 1) + (0.1,0.55) $) -- +(-.7,0) node[midway,above] {\tiny 1};
      }

      \only<13>{
        \draw[-latex] ($ (clone 2) + (0.1,0.55) $) -- +(-.7,0) node[midway,above] {\tiny 3};
      }

      \only<14>{
        \draw[-latex] ($ (clone 3) + (0.1,0.55) $) -- +(-.7,0) node[midway,above] {\tiny 6};
      }

      \only<15>{
        \draw[-latex] ($ (clone 4) + (0.1,0.55) $) -- +(-.7,0) node[midway,above] {\tiny 10};
      }

      \only<16>{
        \draw[-latex] ($ (clone 5) + (0.1,0.55) $) -- +(-.7,0) node[midway,above] {\tiny 15};
      }

      \only<17>{
        \draw[-latex] ($ (clone 6) + (0.1,0.55) $) -- +(-.7,0) node[midway,above] {\tiny 21};
      }

      \only<18>{
        \draw[-latex] ($ (clone 7) + (0.1,0.55) $) -- +(-.7,0) node[midway,above] {\tiny 28};
      }

      \only<19>{
        \draw[-latex] ($ (clone 8) + (0.1,0.55) $) -- +(-.7,0) node[midway,above] {\tiny 36};
      }

      \only<20>{
        \draw[-latex] ($ (clone 9) + (0.1,0.55) $) -- +(-.7,0) node[midway,above] {\tiny 45};
      }

      \only<21>{
        \draw[-latex] ($ (clone 10) + (0.1,0.55) $) -- +(-.7,0) node[midway,above] {\tiny 55};
      }
        
    \end{tikzpicture}
  \end{center}
\end{frame}

\end{document}

%%% Local Variables: 
%%% mode: latex
%%% TeX-master: t
%%% End: 
