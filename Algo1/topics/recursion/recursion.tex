\documentclass{../khlslides}
\usepackage{eurosym}

\title[Recursie]{Recursie}
\author{Fr\'ed\'eric Vogels}


\pgfkeys{/tikz/flowchart/node/.style={rectangle,fill=blue,opacity=0.5,text opacity=1,drop shadow,inner sep=2mm}}
\pgfkeys{/tikz/flowchart/arrow/.style={-latex,flowchart/arrowline}}
\pgfkeys{/tikz/flowchart/arrowline/.style={thick}}


\newcommand{\lcurly}{{\tt{\char '173}}}
\newcommand{\link}[2]{\href{#1}{\beamergotobutton{#2}}}
\newcommand{\PLACEHOLDER}[1]{\ensuremath{\langle}\textrm{\textit{#1}\ensuremath{\rangle}}}


\pgfkeys{
  /wof/sequence point/.cd,
  placement/.initial=above,
  rotation/.initial=0,
  /wof/sequence link/.cd,
  label/.initial={},
  /wof/round/.cd,
  name prefix/.initial=,
  /tikz/.cd,
  sequence point/.style={blue!50,thick,fill=white},
  sequence point label/.style={font=\tiny,black},
  sequence link/.style={blue!50,thick},
  active/.style={fill=red,thick,red}
}
\newcommand<>{\seqpoint}[3][]{
  \only#4{{
  \pgfkeys{/wof/sequence point/.cd,#1}
  \pgfkeys{/wof/sequence point/placement/.get=\placement}
  \pgfkeys{/wof/sequence point/rotation/.get=\rotation}
  \draw[sequence point] (#2) circle (.05)
                        node[sequence point label,\placement,rotate=\rotation] {#3};
}}}
\newcommand<>{\seqlink}[3][]{\only#4{{
  \pgfkeys{/wof/sequence link/.cd,#1}
  \pgfkeys{/wof/sequence link/label/.get=\seqlinklabel}
  \draw[sequence link] (#2) -- (#3) node[midway,sloped,yshift=1mm,font=\tiny,black] {\seqlinklabel};
}}}


\newcommand<>{\dude}[1]{
  \only#2{
    \draw #1 -- ++(0.1,0.2) -- +(0.1,-0.2) +(0,0) -- ++(0,.2) -- +(-.1,0) +(0,0) -- +(.1,0) +(0,0) -- ++(0,0.1) ++(0,0.1) circle (.1);
  }
}

\begin{document}


\begin{frame}
  \titlepage
\end{frame}

\input{summation.tex}

\begin{frame}
  \frametitle{Cash Betalen}
  Je moet \EUR{$n$} betalen, met in je portefeuille
  \begin{overprint}
    \onslide<1-2>
    \begin{itemize}
      \item $x$ centen van \EUR{1}
    \end{itemize}

    \onslide<3-4>
    \begin{itemize}
      \item $y$ centen van \EUR{2}
    \end{itemize}

    \onslide<5-6>
    \begin{itemize}
      \item $x$ centen van \EUR{1}
      \item $y$ centen van \EUR{2}
    \end{itemize}

    \onslide<7-8>
    \begin{itemize}
      \item $x$ centen van \EUR{1}
      \item $y$ centen van \EUR{2}
      \item $z$ briefjes van \EUR{5}
    \end{itemize}
  \end{overprint}
  \vskip4mm
  Kan je exact betalen?
  \vskip5mm
  \begin{overprint}
    \onslide<1>
    \code{cash1.js}
    \onslide<2>
    \code{cash1b.js}
    \onslide<3>
    \code{cash2.js}
    \onslide<4>
    \code{cash2b.js}
    \onslide<5>
    \code{cash12.js}
    \onslide<6>
    \code{cash12b.js}
    \onslide<7>
    \code{cash125.js}
    \onslide<8>
    \code{cash125b.js}
  \end{overprint}
\end{frame}

\begin{frame}
  \frametitle{Cash Betalen}
  \begin{center}
    \Huge We beginnen opnieuw, maar met recursie
  \end{center}
\end{frame}

\begin{frame}
  \frametitle{Cash Betalen}
  Je moet \EUR{$n$} betalen, met in je portefeuille
  \begin{overprint}
    \onslide<1-2>
    \begin{itemize}
      \item $x$ centen van \EUR{1}
    \end{itemize}

    \onslide<3-4>
    \begin{itemize}
      \item $y$ centen van \EUR{2}
    \end{itemize}

    \onslide<5-6>
    \begin{itemize}
      \item $x$ centen van \EUR{1}
      \item $y$ centen van \EUR{2}
    \end{itemize}

    \onslide<7-8>
    \begin{itemize}
      \item $x$ centen van \EUR{1}
      \item $y$ centen van \EUR{2}
      \item $z$ briefjes van \EUR{5}
    \end{itemize}
  \end{overprint}
  \vskip4mm
  Kan je exact betalen?
  \vskip5mm
  \begin{overprint}
    \onslide<1>
    \code{cash1.js}
    \onslide<2>
    \code{cashr1.js}
    \onslide<3>
    \code{cash2.js}
    \onslide<4>
    \code{cashr2.js}
    \onslide<5>
    \code{cash12.js}
    \onslide<6>
    \code[width=.9\linewidth]{cashr12.js}
    \onslide<7>
    \code{cash125.js}
    \onslide<8>
    \code[width=.9\linewidth]{cashr125.js}
  \end{overprint}
\end{frame}

\begin{frame}
  \frametitle{Cash Betalen: Optimisatie}
  \only<1>{
    \code[width=.9\linewidth]{cashr125.js}
  }
  \only<2-5>{
    \code[width=.9\linewidth]{cashr125b.js}
  }
  \only<6-7>{
    \code[width=.95\linewidth]{cashr125c.js}
  }
  \only<7>{
    Pas deze code aan zodat {\tt cash} nagaat
    of je \EUR{$n$} kan vormen
    met centjes van \EUR{2} en
    briefjes van \EUR{5}, \EUR{20} en \EUR{100}.
  }
  \begin{tikzpicture}[overlay,remember picture]
    \only<3>{
      \node[/khl/note,anchor=south west] (negative check note) at ($ (negative check) + (.5,.5) $) {
        \parbox{6cm}{\raggedright
          E\'enmaal negatief kunnen we niet meer exact betalen
        }
      };
      \draw[/khl/note arrow] (negative check note) -- (negative check);
    }

    \only<4>{
      \node[/khl/note,anchor=south] (five note) at ($ (five) + (.5,.5) $) {
        \parbox{6cm}{\raggedright
          We beginnen bij de briefjes van 5 om hopelijk sneller tot ons doel te geraken
        }
      };
      \draw[/khl/note arrow] (five note) -- (five);
    }

    \only<5>{
      \node[/khl/note,anchor=south east] (zero 1 note) at ($ (zero 1) + (.5,.5) $) {
        \parbox{6cm}{\raggedright
          We gaan de briefjes/centen in strikte volgorde af. Als we \'e\'enmaal
          besluiten geen gebruik meer te maken van briefjes van \EUR{5},
          dan komen we hier niet op terug.
        }
      };
      \draw[/khl/note arrow] (zero 1 note) -- (zero 1);
      \draw[/khl/note arrow] (zero 1 note) -- (zero 2);
      \draw[/khl/note arrow] (zero 1 note) -- (zero 3);
    }
  \end{tikzpicture}
\end{frame}

\end{document}

%%% Local Variables: 
%%% mode: latex
%%% TeX-master: t
%%% End: 
